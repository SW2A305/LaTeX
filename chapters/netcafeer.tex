\chapter{Netcafeer}\label{chap:netcafeer}

En netcafé er et sted, hvor interesserede, primært computerspillere, kan samles for at snakke, samt spille de
nyeste spil. Disse netcafeer plejede at eksistere i stor stil, rundt omkring i landet, såvel som i andre
lande. Men i takt med, at folk har fået hurtigere internetforbindelser, og bedre adgang til de nyeste spil, er
eksistensgrundlaget for disse netcafeer så godt som forsvundet, og mange er lukket ned.

Ikke desto mindre er der tale om et område, hvor der er behov for en del management, og på visse områder
minder det om den problemstilling projektet omhandler.


\section{Administration af en netcafé}\label{sec:administration-af-en-netcafe}

En netcafé består typisk af et kioskområde, samt et kundeområde, samt muligvis et administrationsområde,
utilgængeligt for kunderne.

I kioskområdet sælges proviant til kunderne, men det er også hér, tiden administreres. Det varierer lidt
mellem netcafeerne, hvordan dette håndteres. Nogen tilbyder forudbetaling, andre betales bagefter. Derudover
er der betaling per time, eller en flat--rate for et bestemt antal timer, eventuelt i forbindelse med et
event.

I kundeområdet er der typisk opstillet borde med kraftige spillecomputere, behagelige stole og andet udstyr,
som en såkaldt gamer kunne tænkes at få behov for, mens der spilles. Det er også dette område, der oftest
giver den største udfordring når det kommer til administration, for alle computerne skal holdes opgraderede,
både med software og hardware, så det hele kører som det skal.

Et godt management system til netcafeer vil derfor være i stand til at håndtere alt hvad der sker i
kioskområdet, fra salg af slik, sodavand og andet, til håndtering af kundernes tid ved computerne. Men det
skal også gerne kunne stå for distribution af software til maskinerne, når der kommer nye spil/programmer,
inklusive licenshåndtering, samt holde styr på alderen af det forskellige hardware.

Sidstnævnte punkt, med at holde styr på hardware, vil ikke være kritisk, hvis netcafeen har ekstra hardware på
stedet, så en mus, et tastatur eller lignende hurtigt vil kunne skiftes ud, hvis nødvendigt.


\section{Eksisterende systemer}\label{sec:eksisterende-systemer}

Her beskrives to af de eksisterende systemer til håndtering af en netcafé. Det ene er gratis og open source, det andet har en pris, som er fastsat ud fra antallet af klientcomputere der er i kundeområdet, i den pågældende netcafé.


\subsection{CyberCafePro}\label{subsec:cybercafepro}

TBA\ldots


\subsection{Smartlaunch}\label{subsec:smartlaunch}

TBA\ldots

\section{Afgrænsning} \label{IntAfgrænsning}
Ud fra indsamlet data drages konkluderes det at sejlklubber i stor grad har meget til fælles med andre klubber dedikerede til udøvelse af en given hobby. Det ses i analysen af de klubber som rapporten har set nærmere på, at der er et generelt behov for administration af en ressource som klubben har til rådighed. En stor del af de mindre fritidsklubber er drevet af en større del af frivillige, hvilket har en betydning for den tidsmæssige investering der kan sættes i projektet, således ville et effektivt system til håndtering af opgaver som kontingentbetaling, medlemsinformation eller lignende være til gavn for klubbens generelle arbejdskraft.

Der konkluderes således at der ses på et generelt problem. Hertil tilstræbes en overordnet løsning der løser så mange problemer som muligt, og derefter en specialisering til hver enkelt klubs specielle krav. Da en specialiseret løsning til hver enkelt nævnte klub, samt utallige unævnte, ikke er tidsmæssigt muligt, fokuseres der på en overordnet løsning med specialisering i sejlklubber, som så kan videreudvikles til at indeholde yderligere specialer foruden de generelle kriterier.