\chapter{Sportsklubber}
For at få et overblik over de mest populære typer sportsklubber som er, og som kunne tænkes at have gavn af et management system, er der lavet en kort analyse af de forskellige typer af sportsklubber med henblik på management systemer. 
\section{Fodbold}
% Der bør skrives noget meta tekst af en eller anden art for at introducere hvorfor hulen det her er vigtigt.
I en fodboldklub er der mange administrative og organisatoriske opgaver, som kunne have gavn af et management system. Eksempler på sådanne administrative opgaver kunne være:
\begin{itemize}
\item Holde styr på alle baner, hvilket hold som spiller/træner hvor og hvornår.
\item Medlemshåndtering; alder, spillestatus (skader, niveau ol.).
\item Holdsammensætning og taktik.
\item Medlemsbetaling.
\item Kørsel til og fra arrangementer og lignende.
\item Pointgivning ved lokale sportsarrangementer.
\end{itemize}
Små lokale klubber kan også have vask af trøjer og andet udstyr gående på skift blandt medlemmerne, hvor forældre hurtigt vil kunne se hvis tur det er via et management system. 

\section{Badminton}
En badmintonklub har ligesom ved fodbold, nogle baner som klubben har til rådighed. Dog skiller badminton sig ud ved at det ofte foregår i lokale sportshaller, hvor andre sportsklubber også har bane, f.eks. håndbold. Dog vil følgende ting være typisk for hvad en badmintonklub har af administrative opgaver:
\begin{itemize}
\item Håndtering af medlemmer.
\item Udskrivning af regninger.
\item Arrangering af sportsarrangementer
\end{itemize}

\section{Tennis}
Tennis har som andre sportsgrene administrative opgaver som skal klares. Ved tennis er det, for det meste, udendørsbaner, og der er ikke andre sportsklubber, som deler bane med dem. Dog kan almindelige mennesker ofte låne/leje banerne, så hvis man kan se hvornår banerne bruges, så behøver man ikke gå forgæves efter en tennisbane.
Andre administrative opgaver kunne være følgende:
\begin{itemize}
\item Håndtering af medlemmer.
\item Udskrivning af regninger.
\item Arrangering af sportsarrangementer
\end{itemize}

\section{Skydning}
I en skytteklub kan der foregå mange forskellige typer skydning og alle aldersgrupper kan være med. Der kan forekomme indtil flere administrative opgaver, som der i mindre klubber kan forekomme at gøres manuelt. Sådanne administrative opgaver kan være følgende:
\begin{itemize}
\item Administration af skydebaner og evt. reservation
\item Medlemshåndtering inkl. udskrivning af regninger
\item Kørsel til og fra stævner og andre arrangementer
\item Håndtering af våbenlicenser internt i foreningen
\end{itemize}
I nogle skytteforeninger kan man også få sit våben opbevaret i deres våbenskab, sammen med skytteforeningens egne våben. Da våbenene godt kan blive blandet sammen, så kan det godt være svært at holde styr på, hvem der ejer hvilket våben.

\section{Golf}
Golf skiller sig ud fra de tidligere nævnte sportsklubber ved at golf foregår på meget store græsarealer. Ligeledes er der på de græsarealer et bestemt antal huller, og klubben kan have flere baner af ofte 9, eller 18 huller. 

Da der er flere grupper af spillere på samme bane samtidigt, skal klubben i stedet holde styr på hvem der starter hvornår på 1. hul. Herudover, kan det være centralt for klubben at have en måde at registrere om gruppen vil have golfbiler med, og endda caddier, hvis det er noget golfklubben også tilbyder.

Det vil sige, at der i stedet for ressourceplanlægning, er mere tale om skemalægning. Altså hvem der spiller hvornår, og med tilvalg såsom golfbiler eller caddier.
\begin{itemize}
\item Golfbiler?
\item Banerne?
\end{itemize}

\section{Haller}
Sportshaller lægger lokale til mange forskellige sportsklubber. Fodbold bruger omklædningen og de udendørsbaner til træning, badminton bruger de indendørs baner som de skal dele med f.eks. håndbold, indendørs fodbold, indendørs hockey og mange flere. Sportshaller har også ofte lokale arrangementer, f.eks. foredrag, ungdomsklub osv. 
Haller har også ofte en kiosk eller cafe, hvor sportsudøvere eller arrangementsgæster kan komme og få noget at spise og drikke. 
Haller kan også have et motionscenter, hvor idrætsklubber og almindelige personer kan købe adgang. 
Eksempler på administrative opgaver kunne være:
\begin{itemize}
\item Administration af baner og omklædning.
\item Administration af andre typer arrangementer.
\item Information fra kiosken.
\item Informationsdeling til sportsklubber og andre interesserede.
\end{itemize}
