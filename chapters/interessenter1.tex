\chapter{Interessenter}
\section{Dansk sejlunion}
Dansk sejlunion er et forbund, som blev dannet i 1913. 
%[http://www.sejlsport.dk/mere/dansk-sejlunion/historie] 
Deres mission er at være det nationale samlingspunkt for alle sejlere. Dansk sejlunion er tilsluttet Danmarks Idrætsforbund, International Sailing Federation og andre lignende organisationer inden for sejlsport. 
Dansk sejlunion tilbyder også services i form af rådgivning og aktiviteter til klubber, sejlklasser og andre samarbejdspartnere.
%[http://www.sejlsport.dk/mere/dansk-sejlunion/strategi-og-politik/vision-og-vaerdier]

\section{Andre fritidsklubber}
I denne sammenhæng er ''Andre fritidsklubber´´ fritidsklubber som har samme lignende funktion som en sejlklub. Der kan drages paralleller til en skytteforening; en skytteforening har skydebaner som kan adminstreres og hvis man ikke selv medbringer våben, så kan man ofte også låne et, hvor der ligeledes kan ske adminstration. I skydning er konkurrencerne ofte delt op efter typen af skydning, alder og skydestil (andet ord), hvilket medfører at medlemmerne med fordel kunne skrives ind i et system som automatisk kan holde styr på dem og se hvilke konkurrencer de kvalificerer sig til.

\section{Medlemmer}
Interessentgruppen ''medlemmer´´ omhandler medlemmer af sejlklubber. Det er meget forskelligt fra sejlklub til sejlklub hvordan medlemmer kategoriseres eller om de overhovedet gøres dette. F.eks. så har ''Sejlklubben Sundet´´ kategoriseret deres medlemmer således: Voksen-, Bådejer-, Gaste-, Mini-kølbåd-, Ungdoms-, Passiv- og Støttemedlem.
%[http://www.sundet.dk/vedtaegter/Vedtaegter%20for%20Sundet%2027nov12.pdf]
Til sammenligning har Vestre Baadelaug kategoriseret deres medlemmer efter følgende: Aktive-, passive- og æresmellemer
%[http://www.aalborglystbaadehavn.dk/UserFiles/file/VB_filer/Statiske_filer/Vedtaegter_2008-9_web.pdf].

