\chapter{User stories}\label{User_stories}
Til udarbejdelse af en kravspecifikation for programmet er der blevet brugt user stories.
Ud fra disse user stories dannes den egentlige kravspecifikation for programmet.
Under udviklingen kan det også bruges som en checkliste, så der ikke glemmes funktionaliteter.

\section{Medlemmer}
Som \textbf{bruger} kan jeg se sejlklubbens kalender.
\newline
Som \textbf{bruger} kan jeg ændre eller slette mine reservationer, hvis jeg ombestemmer mig.
\newline
Som \textbf{bruger} vil jeg kunne logge ind i programmet.
\newline
Som \textbf{bruger} vil jeg have en personlig side med overblik over mine sejlture.
\newline
Som \textbf{bruger} vil jeg gerne kunne se hvilke andre medlemmer der er i sejlklubben, for at kunne aftale fælles sejlture med dem.
\newline
Som \textbf{administrator} vil jeg gerne være i stand til at se detaljeret information om de medlemmer, der er i sejlklubben.
\newline
Som \textbf{administrator} kan jeg importere medlemmer fra andre systemer, så de kan blive genbrugt.
\newline
Som \textbf{administrator} vil jeg have en speciel administratordel af programmet.
\newline
Som \textbf{administrator} vil jeg kunne angive rettighedsniveau for medlemmerne.

\section{Både}

Som \textbf{administrator} vil jeg være i stand til at tilføje og fjerne både fra systemet.
\newline
Som \textbf{administrator} vil jeg kunne markere både som værende operationelle.

\section{Bådudlån}

Som \textbf{bruger} kan jeg reservere både.
\newline
Som \textbf{administrator} kan jeg se hvor mange ture hver bruger har taget, så jeg kan kræve det korrekte beløb fra dem.

\section{Undervisning}
Som \textbf{elev} kan jeg se informationer omkring sejlerskolen, så jeg kan følge mine fremskridt, se fremtidige lektioner osv. 
\newline
Som \textbf{elev} vil jeg kunne se hvilke lektioner jeg har deltaget i og hvilke jeg ikke har.
\newline
Som \textbf{underviser} vil jeg gerne være i stand til at kontakte de elever, som er med i sejlerskolen.
\newline
Som \textbf{underviser} kan jeg se og ændre brugers fremskridt i sejlerskolen, så brugerne bliver opdateret.
\newline
Som \textbf{underviser} vil jeg kunne se en liste over mine elever.
\newline
Som \textbf{underviser} vil jeg have en oversigt over mine lektioner samt deltagere til disse.

\section{Begivenheder}
Som \textbf{bruger} vil jeg kunne se hvilke begivenheder der er i klubben samt tilmelde mig.
\newline
Som \textbf{administrator} vil jeg være i stand til at oprette/fjerne/redigere begivenheder.
