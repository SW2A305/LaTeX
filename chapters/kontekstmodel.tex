\chapter{Struktur af rapporten}\label{chap:struktur-af-problemanalyse}

I dette afsnit, beskrives strukturen af rapporten. Strukturen er baseret på beskrivelsen af et informationssystem fra \citet{Laudon1999}. Komponenterne i et informationssystem er illustreret i
\myref{fig:kontekstmodel}.

\begin{figure}[htbp]
  \centering
  \includegraphics{images/kontekstmodel/metode.png}
  \caption[Metode for Kontekstmodellen]{Illustration af elementerne i et informationssystem. Kilde:
  \protect\citet{Laudon1999}}
  \label{fig:kontekstmodel}
\end{figure}


\section{Informationssystem}\label{sec:Informationssystem}

Et informationssystem bruges til at effektivisere en arbejdsproces og hjælper med at holde fokus, så arbejdet bliver gjort tilfredsstillende. 
For at udvikle et godt informationssystem, skal man indsamle viden om mennesker, organisation
og teknologi. 
Når informationssystemet er udviklet består de af tre processer: Indsamling af data, behandling af dataene
og formidling af dataene. 
De tre elementer bruges til at finde ud af, hvad der skal tages hensyn til under problemløsningsdelen, og vil blive beskrevet nedenunder.

\subsection{Mennesker}\label{subsec:mennesker}

Elementet \textit{mennesker} handler om personer/persongrupper, som har en interesse i, at en given
problemstilling løses. Det kan være brugeren af det program, der bliver lavet, samt andre, som får gavn af en
løsning. Man undersøger bl.a. brugerens evner, da programmet skal laves på en sådan måde, at brugeren har den
fornødne kunnen, til at kunne betjene programmet. Brugerens behov undersøges også, så man får alle de
funktioner med, som er nødvendige for at programmet er brugbart.


\subsubsection{Organisation}\label{subsec:organisation}

Elementet \textit{organisation} undersøges hvor og hvordan problemet opstår, derudover undersøges der hvilke
regler og værdier organisationen har, for at kunne tage disse med til problemløsningen.


\subsection{Teknologi}\label{subsec:Teknologi}

Elementet \textit{teknologi} omhandler de teknologier, som anvendes til at løse en informationssystemsrelevant
problemstilling. Ofte vil denne teknologi være en computer,
en server og/eller internettet. Dette er altså de byggeblokke hvorpå systemet bygges.\fxnote{Omskrives så det passer med SOTA}

\section{Problemløsning}\fxnote{Mangler problemløsningsdel}

\section{Rapportens opbygning}\label{sec:rapportens-opbygning}

Først bliver menneskedelen behandlet, i form af en interessentanalyse, herefter vil organisationselementet
blive berørt. Derefter bliver der skrevet om teknologier, hvorefter de tre elementer munder ud i en
problemafgrænsning og en problemformulering. 
Til sidst skrives der om problemløsningsdelen. \fxnote{Udvide problemløsningsdel i dette afsnit.}


