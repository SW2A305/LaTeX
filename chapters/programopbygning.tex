\chapter{Programopbygning}

I dette kapitel beskrives programmets opbygning. 

\begin{figure}[h]
\centering
\tikzstyle{lille} = [rectangle, minimum width=2cm, minimum height=1.0cm,text centered, draw=black, fill=blue!30]
\tikzstyle{invi} = [draw, rectangle, minimum height=2cm, minimum width=2cm]
\tikzstyle{line} = [draw]
\tikzstyle{arrow} = [thick,->,>=stealth]
\begin{tikzpicture}[node distance = 1.5cm]
%noderne (objekterne) laves
\node (uixaml) [lille] {UI XAML};
\node (cb) [lille, below of=uixaml] {CB};
\node (invi1) [invi,draw=none,below of=cb] {};
\node (model) [lille, below of=invi1] {Model};
\node (idal) [lille, right of=invi1] {iDal};
\node (sqldal) [lille, right=0.5cm of idal] {SQLite Dal};
\node (efdal) [lille, below of=sqldal] {EF Dal};
\node (sql) [lille, right=0.5cm of sqldal] {SQLite};
\node (ef) [lille, below of=sql,align=center] {Entity\\ Framework};
%pointers laves
\draw [line] (uixaml) -- (cb);
\draw [line] (cb) -| (idal);
\draw [dashed] (cb)-- (model);
\draw [line] (model) -| (idal);
\draw [line] (idal) -- (sqldal);
\draw [line] (idal) to [bend right] (efdal);
\draw [line] (sqldal) -- (sql);
\draw [line] (efdal) -- (ef);
\end{tikzpicture}
\caption{Programopbygning: Figuren viser sammenhængen mellem de forskellige komponenter i programmet.}
\label{img:Program_flow}
\end{figure}

På \myref{img:Program_flow}, ses den overordnede struktur af programmet.\fxnote{Skriv noget om her at vi har valgt at bruge en database, og at man kan læse mere om det i kapitel(Database)? Vi snakker om dal og alt muligt her, uden at vi har fortalt noget som helst om det til læseren (Søren)}
Nederst på figuren ses modelboksen, som repræsenterer modellaget, her findes de modeller, der anvendes i programmet.\fxnote{Lidt sjov sætning, evt. punktum efter 'modellaget'}
Modellerne er forbundet til Dal'et (Data Abstraction Layer), hvilket er den generiske kontakt fra programmet ud mod databasen. \fxnote{Troels: Her lyder det som om at IDal og dal blandes sammen, med det neden under}	
Fra den generiske IDal findes der forbindelser ud til et SQLite Dal og et Entity Framework Dal, som hver i sær har forbindelse til de databaser, de hører til. \fxnote{Troels: IDal har ingen forbindelser, det definerer bare hvad hvert dal skal implementere}\fxnote{Tristan: Ved ikke om vi skal have Entity Framework med her, da vi jo helt har fjernet det som en komponent i programmet?}
Ideen ved det generiske IDal er, at der let kan skiftes mellem forskellige databaser, hvilket vil blive uddybet i \myref{chap:database}.
Over IDal og Model findes Code-Behind og UI'en. \fxnote{Troels: Er dette en beskrivelse af flowchartet? eller er det en beskrivelse af vores programs opbygning?} \fxnote{UI? Skal vi ikke bruge nogle ord her :D ? (Søren)}
Code-Behind og UI er forbundet, da UI-filerne kun indeholder grafiske elementer, og Code-Behind filerne indeholder det kode, som UI'en har brug for til at fungere.
Code-Behind filerne er også i direkte kontakt med IDal \fxnote{Troels: gennem IDal interfacet til det aktive Dal} for at kunne tilgå data i databasen.
Yderligere har Code-Behind også kendskab til modellaget, illustreret med en stiplet linje.
Code-Behind har behov for kendskab til modellerne, da der flere steder i koden har været brug for at instantiere modellerne for at håndtere data. 