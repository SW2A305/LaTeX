\chapter{Programopbygning}

I dette kapitel beskrives programmets opbygning. 

\subsection{Programopbygning} \label{img:programOpbygning}
\begin{figure}[h]
\centering
\tikzstyle{lille} = [rectangle, minimum width=2cm, minimum height=1.0cm,text centered, draw=black, fill=blue!30]
\tikzstyle{invi} = [draw, rectangle, minimum height=2cm, minimum width=2cm]
\tikzstyle{line} = [draw]
\tikzstyle{arrow} = [thick,->,>=stealth]
\begin{tikzpicture}[node distance = 1.5cm]
%noderne (objekterne) laves
\node (uixaml) [lille] {UI XAML};
\node (cb) [lille, below of=uixaml] {CB};
\node (invi1) [invi,draw=none,below of=cb] {};
\node (model) [lille, below of=invi1] {Model};
\node (idal) [lille, right of=invi1] {iDal};
\node (sqldal) [lille, right=0.5cm of idal] {SQLite Dal};
\node (efdal) [lille, below of=sqldal] {EF Dal};
\node (sql) [lille, right=0.5cm of sqldal] {SQLite};
\node (ef) [lille, below of=sql,align=center] {Entity\\ Framework};
%pointers laves
\draw [line] (uixaml) -- (cb);
\draw [line] (cb) -| (idal);
\draw [dashed] (cb)-- (model);
\draw [line] (model) -| (idal);
\draw [line] (idal) -- (sqldal);
\draw [line] (idal) to [bend right] (efdal);
\draw [line] (sqldal) -- (sql);
\draw [line] (efdal) -- (ef);
\end{tikzpicture}
\caption{Figuren viser sammenhængen mellem de forskellige komponenter i programmet.}
\label{Program_flow}
\end{figure}

På \myref{img:Program_flow}, ses den overordnede struktur af programmet. Ud fra figuren kan der ses de forskellige elementer programmet er opbygget af.
Nederst på figuren ses boksen \textit{Model}, denne repræsenterer modellaget.
Heri findes de modeller, som bliver brugt i programmet dette er nærmere forklaret i \myref{chap:klasser}.
Modellaget har en direkte forbindelse til det generiske IDal(Data Abstraction Layer), hvilket kan ses på figuren som \textit{IDal} boksen.
Dette har en direkte forbindelse ud til persistenslaget, dette er nærmere forklaret i \myref{chap:database}.
Yderligere ses en forbindelse fra boksen \textit{IDal} til boksen \textit{Code-Behind}. 
Denne boks repræsenterer programmets bagvedliggende kode, som gennem IDak interfacen kan tilgå persistenslaget. 
Det ses ydermer fra figuren at der er en direkte forbindelse til UI'et repræsenteret med boksen \textit{UI XAML}.
Koden har yderligere kendskab til modellaget, illustreret med en stiplet linje, for at kunne instantiere modellerne så da data kan håndteres.
Dette er den visuele del af programmet som brugeren kan se, når brugeren laver en aktion på programmet kaldes Code-Behind'en for at udføre dette, heraf den direkte forbindelse mellem de to. 
En nærmere forklaring af disse bokse følger i \myref{Programmets_brugergraenseflade}
