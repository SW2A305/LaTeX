\chapter{Programopbygning}

I dette kapitel vil programmets opbygning blive beskrevet ud fra UML-diagrammet, som kan findes i \myref{UML_diagram}.
Dataene indeholdt i klasser kan ses ud fra UML-diagrammet. De fleste af dataene indeholdt i klasserne giver sig selv,
men nogle vil blive uddybet.
Sidst i kapitlet vil der blive argumenteret for dataene der skal være på medlemmerne.


\section{Klasser}

For at holde styr på medlemmer i sejlklubben, er der lavet en overordnet person klasse. Klassens ``PersonId'' er et
nummer som kan bruges til at skelne de enkelte personer fra hinanden. ``BoatDriver'' er en boolsk værdi som afgør om
personen har et førerbevis eller ej. ``SailClubMember'' klassen arver fra person klassen og har et ``MemberId'' og et
``Username'' og ``PasswordHash''. De to sidstnævnte skal bruges til logge på i programmet. Klassen har også et
``Position'' medlem. Denne afgør hvilke slags medlem man er. Medlemstyper er som følger:

\begin{itemize}
\item SupportMember
\item Student
\item Member
\item Teacher
\item Admin
\end{itemize}

Programmet indeholder også en båd klasse. Dataene indeholdt i denne klasser giver sig selv ud fra UML-diagrammet. Der er
også en overordnet klasse for sejlture som er abstrakt. Denne arver klasserne skoletur og almindelig tur fra. Der skal
for hver sejltur, registreres hvilken båd der bliver brugt samt tidspunkt for afgang og ankomst. Vejrforholdene skal
registres og der kan tilføjes de kommentarer der muligvis kan være. Når der er tale om en almindelig tur, registreres
der hvem kaptajnen er og de øvrige besætningsmedlemmer bliver tilføjet til en liste. Der skal desuden registreres,
hvornår man regner med at være tilbage fra turen og hvad formålet med turen er og hvor man sejler henne. Dette skal
gøres så det er lettere at finde båden og besætningen, f.eks. i tilfælde af at en ulykke forekommer.
Ved en skoletur er der registret et skolehold, som er defineret i en klasse. Skoleholds klassen indeholder lærer og
elever. Man kan tilføje yderligere elever til en skoletur, som normalt ikke er med samt gæster som ikke er elever. Der
registreres også hvem der deltager i turen, da der kan være elever fra det normale hold, som ikke møder op. Til
sejlturene er der tilknyttet en skadesraport, som skal, lige som navnet angiver, bruges til registre hvis der kommer
skader på båden under sejladsen. 


\section{Medlemsdata}

``PersonId'' og ``MemberId'' bliver som nævnt brugt til at skelne de forskellige personer og medlemmer fra hinanden,
ligesom ``FirstName'' og ``LastName'' bliver brugt til identificering. Adresse dataene er almindelige for en klub at
have på sine medlemmer, men bliver ikke brugt til noget i programmet. Ligeledes bliver ``DateOfBirth'', ``Gender'',
``PhoneNumber'' og ``Email'' ikke brugt. ``BoatDriver'' fortæller som sagt hvorvidt medlemmet har førerbevis, og
bestemmer i programmet om vedkommende medlem har lov til at være ``Captain'' på en sejltur. ``Position'' afgør hvad
medlemmet har adgang til i programmet, f.eks. har et medlem der er ``Student'' adgang til undervisningsfunktionerne i
programmet, hvorimod et almindeligt ``Member'' ikke har. ``Username'' og ``PasswordHash'' bruges som sagt til at logge
på.
