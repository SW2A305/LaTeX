\chapter{Programopbygning}

I dette kapitel beskrives programmets opbygning. 

\begin{figure}[h]
\centering
\tikzstyle{lille} = [rectangle, minimum width=2cm, minimum height=1.0cm,text centered, draw=black, fill=blue!30]
\tikzstyle{invi} = [draw, rectangle, minimum height=2cm, minimum width=2cm]
\tikzstyle{line} = [draw]
\tikzstyle{arrow} = [thick,->,>=stealth]
\begin{tikzpicture}[node distance = 1.5cm]
%noderne (objekterne) laves
\node (uixaml) [lille] {UI XAML};
\node (cb) [lille, below of=uixaml] {CB};
\node (invi1) [invi,draw=none,below of=cb] {};
\node (model) [lille, below of=invi1] {Model};
\node (idal) [lille, right of=invi1] {iDal};
\node (sqldal) [lille, right=0.5cm of idal] {SQLite Dal};
\node (efdal) [lille, below of=sqldal] {EF Dal};
\node (sql) [lille, right=0.5cm of sqldal] {SQLite};
\node (ef) [lille, below of=sql,align=center] {Entity\\ Framework};
%pointers laves
\draw [line] (uixaml) -- (cb);
\draw [line] (cb) -| (idal);
\draw [dashed] (cb)-- (model);
\draw [line] (model) -| (idal);
\draw [line] (idal) -- (sqldal);
\draw [line] (idal) to [bend right] (efdal);
\draw [line] (sqldal) -- (sql);
\draw [line] (efdal) -- (ef);
\end{tikzpicture}
\caption{Programopbygning: Figuren viser sammenhængen mellem de forskellige komponenter i programmet.}
\label{img:Program_flow}
\end{figure}

På \myref{img:Program_flow}, ses den overordnede struktur af programmet.
Ud fra figuren kan der ses de forskellige elementer programmet består af.
Nederst på figuren ses boksen \textit{Model}, som repræsenterer modellaget.
Heri findes de modeller, som bliver brugt i programmet, de er nærmere forklaret i \myref{chap:klasser}.
Modellaget har en forbindelse til det generiske Data Abstraction Layer interface (IDAL), hvilket kan ses på figuren \textit{IDAL} boksen.
Dette har en forbindelse til persistenslaget, som er nærmere forklaret i \myref{chap:database}.
Yderligere ses en forbindelse fra boksen \textit{IDAL} til boksen \textit{Code-Behind}. 
Denne boks repræsenterer programmets bagvedliggende kode, som gennem IDAL interfacet kan tilgå persistenslaget. 
Det ses ydermere på figuren, at der er en forbindelse til brugergrænsefladen (GUI) repræsenteret med boksen \textit{GUI}.
Koden har yderligere kendskab til modellaget, illustreret med en stiplet linje, for at kunne instantiere modellerne så data kan håndteres.
Dette er den del af programmet, som brugeren kan se, når brugeren laver en aktion i programmet kaldes Code-Behind'en for at udføre denne, heraf forbindelsen mellem de to. 
En nærmere forklaring af \textit{GUI} og \textit{Code-Behind} boksene følger i \myref{Programmets_brugergraenseflade}.
