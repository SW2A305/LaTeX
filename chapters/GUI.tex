\chapter{Grafisk brugergrænseflade} \label{chap:GUI}

I forbindelsen med udviklingen af systemet, skal man sørge for at designe det så brugerne kan finde rundt i de forskellige funktioner. Da sejlklubben Sundets medlemmer kan bestå af en meget mangfoldig gruppe af mennesker, er det svært at indskrænke brugerne i en enkelt gruppe. Brugerne af systemet konkluderes derfor at kunne have meget forskellige evner inden for brugen af IT.

\section{Designet} \label{sec:Designet}

Programmets brugergrænseflade er designet ud fra principperne ifølge. \citep{gui1} 

\begin{itemize}
	\item Minimalisme
	\item Konsistente
\end{itemize}

Det betyder at der skal være så få forstyrrelser på skærmbilledet som muligt. Det skal være enkelt og simpelt at navigere rundt og finde de funktioner i programmet man skal bruge. Der skal ikke være mange forskellige farver, og generelt skal programmet følge samme struktur til at navigere rundt igennem hele programmet. Herudover skal der ikke være en dyb menustruktur helst ikke et dybere hierarki end 3.

Disse forskellige principper eller guidelines, er forsøgt implementeret i systemet.

\section{Implementation}\label{sec:Implementation}

Menustrukturen består af tabs, som er store og lette at se. De forskellige funktioner i programmet er delt op i deres tilhørende tabs, og man kan altid gå ind i en ny tab uanset hvor man befinder sig i programmet. Dette betyder at strukturen ikke har et hierarki, da uanset hvor man befinder sig, kan der navigeres ind i alle programmets funktioner. Der er valgt blå nuancer, da det er en sejlklub, og dermed virkede som en logisk beslutning. Grunden til der ikke er flere farver rundt omkring i programmet, skyldes altså den minimalistiske, samt den konsistente, tankegang.

\fxnote{Her skal tilføjes nogle screenshots, så man kan se hvordan det helt konkret ser ud. Dette synes jeg dog skal vente til programmet når et mere færdigt stadie.}

Når brugergrænsefladen er lavet, skal den dermed også testes. Der vil blive forklaret hvordan dette vil blive gjort i et senere kapitel. \fxnote{Tilføj en kilde til kapitlet omhandlende test af programmets GUI}