\chapter{Interessenter}

I følgende afsnit vil der blive set nærmere på sejlklubber, herunder hvordan de bliver håndteret, hvilket arbejde det
indebærer, samt hvor der eventuelt er plads til forbedring. I henhold til dette ses der også på andre organisationer
som minder om sejlklubber, her er fokus sat på klubber, der som sejlklubben, er en hobby for en gruppe af mennesker.
Dette gøres for at se om sådanne klubber har et generelt behøv, eller om det specifikt er sejlklubber.


\section{Fritidsklubber} \label{Fritidsklub}

I forbindelse med sejlklubber er der blevet set på en række af diverse fritidsklubber. Det undersøges for hver klub, om
de har opgaver, hvor et EDB system kunne lette arbejdsbyrden for, de primært frivillige, hjælperne i fritidsklubberne.
Fritidsklubber benytter ofte frivillig som arbejdskraft både som trænere, hjælpere, bestyrelse eller lignende
administrative roller. Alt efter klubbens størrelse er nogle af disse roller nogle gange betalte. Disse roller indebærer
typisk administrativt arbejde, hvilket kan variere fra klub til klub og vil blive beskrevet nærmere i afsnit
\ref{Fritidsklub}\fxwarning{Henviser til nuværende afsnit!?} og  \ref{chap:netcafeer}. Nogle administrative opgaver såsom planlægning af træning/undervisning er
udeladt fra det specifikke afsnit, da dette er generelt for alle fritidsklubber. Der vil  blive set på mere
administrative opgaver i klubben, som håndtering af ressourcer og lignende, da disse kan variere alt efter hvad der skal
håndteres.

\fxnote{Kan være der her skal inkluderes en liste af mere generelle ting?(Medlemskontigent, undervisning osv. og så fjerne det fra de enkelte afsnit, på den måde undgås at dette blot skrives på variende måder}
%\subsection{Sejlklub}\label{subsec:sejlklub}
%Dette projekt handler om management af en sejlklub, hvilket er en forening som har mange administrative opgaver, som er
%besværlige at håndtere manuelt. Der vil derfor, i denne rapport, blive fokuseret på de funktioner, der ønskes af et
%elektronisk management system, i en sejlklub. Funktionerne er som følger:
%
%\begin{itemize}
%\item Af- og tilmelding af undervisning
%\item Reservering af både
%\item Oversigt over hvem der skylder penge for lån af båd
%\end{itemize}
%\fxnote{Afsnittet skal skrives om!!!}
%


\subsection{Fodbold} \label{Fodbold}

% Der bør skrives noget meta tekst af en eller anden art for at introducere hvorfor hulen det  %her er vigtigt.
I en fodboldklub er der mange administrative og organisatoriske opgaver, som kunne have gavn af et elektronisk system
til håndtering af administrative opgaver frem for et manuelt system. Eksempler på sådanne
administrative opgaver kunne være:

\begin{itemize}
\item Holde styr på baner, hvilket hold som spiller/træner hvor og hvornår
\item Medlemshåndtering; alder, spillestatus (skader, niveau ol.)
\item Holdsammensætning og taktik
\item Medlemsbetaling
\item Kørsel til og fra arrangementer og lignende
\item Pointgivning ved lokale sportsarrangementer
\end{itemize}

Små lokale klubber kan også have vask af trøjer og andet udstyr gående på skift blandt medlemmerne, hvor forældre hurtigt vil kunne se hvis tur det er via et administrations system. 

\subsection{Badminton og tennis}

En badmintonklub har ligesom ved fodbold, nogle baner som klubben har til rådighed. Dog skiller badminton sig ud ved, at det ofte foregår i lokale sportshaller, hvor andre sportsklubber også har bane, f.eks. håndbold. Tennis er her meget ens, den største forskel værende, at de typisk har deres egne baner samt dette foregår udendørs.\fxnote{Det foregår vel ikke nødvendigvis udenfor?} Disse baner kan typisk lejes, hvilket også skal håndteres. Administrationen indebærer blandt andet følgende opgaver:

\begin{itemize}
\item Håndtering af medlemmer
\item Udskrivning af regninger
\item Arrangering af sportsarrangementer
\item Udlejning af baner
\end{itemize}

%\subsection{Tennis}
%Tennis har som andre sportsgrene administrative opgaver som skal klares. Ved tennis er det, for %det meste, udendørsbaner, og der er ikke andre sportsklubber, som deler bane med dem. Dog kan %almindelige mennesker ofte låne/leje banerne, så hvis man kan se hvornår banerne bruges, så %behøver man ikke gå forgæves efter en tennisbane.
%Andre administrative opgaver kunne være følgende:
%\begin{itemize}
%\item Håndtering af medlemmer.
%\item Udskrivning af regninger.
%\item Arrangering af sportsarrangementer
%\end{itemize}


\subsection{Skydning}

I en skytteklub kan der foregå mange forskellige typer skydning og alle aldersgrupper kan være med. Der kan forekomme optil flere administrative opgaver, som der i mindre klubber kan forekomme at gøres manuelt. Sådanne administrative opgaver kan være følgende:

\begin{itemize}
\item Administration af skydebaner og evt. reservation
\item Medlemshåndtering inkl. udskrivning af regninger
\item Kørsel til og fra stævner og andre arrangementer
\item Håndtering af våbenlicenser internt i foreningen
\end{itemize}

I nogle skytteforeninger kan man også få sit våben opbevaret i deres våbenskab, sammen med skytteforeningens egne våben. Da våbenene godt kan blive blandet sammen, så kan det være svært at holde styr på, hvem der ejer hvilket våben.


\subsection{Golf}

Golf skiller sig ud fra de tidligere nævnte sportsklubber ved at golf foregår på meget store græsarealer. Ligeledes er der på de græsarealer et bestemt antal huller, og klubben kan have flere baner af ofte 9 eller 18 huller. 

Da der er flere grupper af spillere på samme bane samtidigt, skal klubben i stedet holde styr på hvem der starter hvornår på 1. hul. Herudover, kan det være centralt for klubben at have en måde at registrere om gruppen vil have golfbiler med, og endda caddier, hvis det er noget golfklubben også tilbyder.

Det vil sige, at der i stedet for ressourceplanlægning, er mere tale om skemalægning. Altså hvem der spiller hvornår, og med tilvalg såsom golfbiler eller caddier.


\subsection{Haller}

Sportshaller lægger lokale til mange forskellige sportsklubber. Fodbold bruger omklædningen og de udendørsbaner til træning, badminton bruger de indendørs baner som de skal dele med f.eks. håndbold, indendørs fodbold, indendørs hockey og mange flere. Sportshaller har også ofte lokale arrangementer, f.eks. foredrag, ungdomsklub osv. 
Haller har også ofte en kiosk eller cafe, hvor sportsudøvere eller arrangementsgæster kan komme og få noget at spise og drikke. 
Haller kan også have et motionscenter, hvor idrætsklubber og almindelige personer kan købe adgang. 
Eksempler på administrative opgaver kunne være:

\begin{itemize}
\item Administration af baner og omklædning.
\item Administration af andre typer arrangementer.
\item Information fra kiosken.
\item Informationsdeling til sportsklubber og andre interesserede.
\end{itemize}


\section{Netcaféer}\label{chap:netcafeer}

En netcafé er et sted, hvor interesserede, primært computerspillere, kan samles for at snakke, samt spille de
nyeste spil. Disse netcafeer plejede at eksistere i stor stil, rundt omkring i landet, såvel som i andre
lande. Men i takt med, at folk har fået hurtigere internetforbindelser, og bedre adgang til de nyeste spil, er
eksistensgrundlaget for disse netcafeer så godt som forsvundet, og mange er lukket ned.

Ikke desto mindre er der tale om et område, hvor der er behov for en del administration, og på visse områder
minder det om den problemstilling projektet omhandler.


\subsection{Administration af en netcafé}\label{sec:administration-af-en-netcafe}

En netcafé består typisk af et kioskområde, samt et kundeområde, samt muligvis et administrationsområde,
utilgængeligt for kunderne.

I kioskområdet sælges proviant til kunderne, men det er også hér, tiden administreres. Det varierer lidt
mellem netcafeerne, hvordan dette håndteres. Nogen tilbyder forudbetaling, andre betales bagefter. Derudover
er der betaling per time, eller en flat--rate for et bestemt antal timer, eventuelt i forbindelse med et
event.

I kundeområdet er der typisk opstillet borde med kraftige spillecomputere, behagelige stole og andet udstyr,
som en såkaldt gamer kunne tænkes at få behov for, mens der spilles. Det er også dette område, der oftest
giver den største udfordring når det kommer til administration, for alle computerne skal holdes opgraderede,
både med software og hardware, så det hele kører som det skal.

Et godt administrations system til netcafeer vil derfor være i stand til at håndtere alt hvad der sker i
kioskområdet, fra salg af slik, sodavand og andet, til håndtering af kundernes tid ved computerne. Men det
skal også gerne kunne stå for distribution af software til maskinerne, når der kommer nye spil/programmer,
inklusive licenshåndtering, samt holde styr på alderen af det forskellige hardware.

Sidstnævnte punkt, med at holde styr på hardware, vil ikke være kritisk, hvis netcafeen har ekstra hardware på
stedet, så en mus, et tastatur eller lignende hurtigt vil kunne skiftes ud, hvis nødvendigt.


\section{Afgrænsning}

Ud fra indsamlet data konkluderes der, at sejlklubber i stor grad har meget til fælles, med andre klubber dedikerede til udøvelse af en given hobby. Det ses i analysen, af de klubber som rapporten har set nærmere på, at der er et generelt behov for administration af en ressource som klubben har til rådighed. En stor del af de mindre fritidsklubber er drevet af en større del af frivillige, hvilket har en betydning for den tidsmæssige investering der kan sættes i projektet. Således ville et effektivt system til håndtering af opgaver som kontingentbetaling, medlemsinformation eller lignende være til gavn for klubbens generelle arbejdskraft.

Der konkluderes således at der ses på et generelt problem. Der tilstræbes en overordnet løsning, der løser så mange problemer som muligt, og derefter en specialisering til hver enkelt klubs specielle krav. Da en specialiseret løsning til hver enkelte klub, ikke er tidsmæssigt muligt, fokuseres der på en overordnet løsning med specialisering i sejlklubber, som så kan videreudvikles til at indeholde yderligere specialer foruden de generelle kriterier.

