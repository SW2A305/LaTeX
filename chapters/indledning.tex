\chapter{Indledning}\label{chap:indledning}

Alle organisationer har administrative opgaver, hvad end den er stor eller lille, men størrelsen afgør tiden der skal bruges på opgaverne. Når der er tale om en lidt større organisation, kan det let blive uoverskueligt at gøre det hele manuelt. For at optimere arbejdsprocessen ved administrative opgaver, kan der bruges et elektronisk system. Denne rapport handler om udviklingen af et sådant system.

I denne rapport fokusres der på de administrative opgaver, som en sejlklub har, og der undersøges hvilke funktioner, en sejlklub vil kunne drage nytte af at have i et elektronisk system. Udover dette, vil der kort bliver undersøgt, hvad gavn andre slags sportsforeninger kan have af et elektronisk system, som kan administrere administrative opgaver.

\fxnote{Ide til noget der kan undersøges/skrives om i rapport - skal lige vendes med gruppen: Hvor meget tid kan der spares på at få et elektronisk system? Er der flere som vil have mod på at melde sig som frivillig, hvis administrationen foregik elektronisk?}

For bedre at kunne overskue omfanget af funktioner, der skal implementeres i et elektronisk system, fokuseres der på én klub, nærmere bestemt sejlklubben ``Sundet'', som har til huse i København. Der vil blive foretaget et interview med en repræsentant for klubben, som også er gruppens vejleder.

Når undersøgelserne er færdige, vil der så vidt muligt blive udviklet et system, som kan varetage de funktioner, som ``Sundet'' har brug for.
