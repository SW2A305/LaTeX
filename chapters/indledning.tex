\chapter{Indledning}\label{chap:indledning}

% Alle organisationer har administrative opgaver, hvad end den er stor eller lille, men størrelsen afgør tiden der skal bruges på opgaverne. Når der er tale om en lidt større organisation, kan det let blive uoverskueligt at gøre det hele manuelt. For at optimere arbejdsprocessen ved administrative opgaver, kan der bruges et elektronisk system. Denne rapport handler om udviklingen af et sådant system.
% I denne rapport fokusres der på de administrative opgaver, som en sejlklub har, og der undersøges hvilke funktioner, en sejlklub vil kunne drage nytte af at have i et elektronisk system. Udover dette, vil der kort bliver undersøgt, hvad gavn andre slags sportsforeninger kan have af et elektronisk system, som kan administrere administrative opgaver.
% \fxnote{Ide til noget der kan undersøges/skrives om i rapport - skal lige vendes med gruppen: Hvor meget tid kan der spares på at få et elektronisk system? Er der flere som vil have mod på at melde sig som frivillig, hvis administrationen foregik elektronisk?}
% For bedre at kunne overskue omfanget af funktioner, der skal implementeres i et elektronisk system, fokuseres der på én klub, nærmere bestemt sejlklubben ``Sundet'', som har til huse i København. Der vil blive foretaget et interview med en repræsentant for klubben, som også er gruppens vejleder.
% Når undersøgelserne er færdige, vil der så vidt muligt blive udviklet et system, som kan varetage de funktioner, som ``Sundet'' har brug for.

%%% INDLEDNING
%EMNE INTRODUKTION
%%FORMÅL MED PROJEKT (SOFTWARE)
%ANDRE SPORTSKLUBBER
%Rapportopdeling
%%ANALYSE
%%LØSNIN 	G

Denne rapport er den skriftlige del af et projekt, udført af en gruppe studerende på AAU, nærmere bestemt er det et P2 projekt på Software studiet. Formålet med projektet er i følge projektoplægget \textit{``[..] at opnå færdigheder i problemorienteret projektarbejde i en gruppe samt viden om sammenhænge mellem problemdefinition, modeldannelsers rolle i forståelse og konstruktion af programmer, og programmer som løsning på et problem i en problemstilling kontekst.''} Helt konkret skal der produceres et program i C\#, som skal virke som en hel eller delvis løsning på det problem opstillet i det emne, som er valgt. 

Projektets emne er ``Administration af skolebåde og bådudlån i en sejlklub'', dette er et meget specifikt emne, derfor vil vi undersøge muligheden for at udvide det til andre sports- og fritidsklubber. 

I moderne tider er computere og smartphones, i højere grad blevet en integreret del af langt størstedelen af personers hverdag. 
Dog er det ikke altid, at foreninger og fritidsklubber anvender disse nye værktøjer på en effektiv og optimal måde. 
Dette er det problem, som gruppen heri rapporten vil undersøge, samt forsøge at bidrage til en løsning. 

Rapporten er opdelt i to hoveddele, først undersøges problemet i problemanalysen og dernæst udarbejdes en løsning i løsningsdelen. 
Problemanalysen har til mål, at opnå et overblik over hvilke interessenter, deres organisation og den teknologi der allerede findes. 
Dette ender ud i en problemformulering som leder projektet ind i løsningsdelen. 
Her skal skabelsen af selve programmet dokumenteres, herunder vil der være en programspecifikation som vil være de krav der er opstillet, på baggrund af analysen. \fxnote{Her skal tilføjes lidt ang. hvad løsningsdelen kommer til helt konkret at indeholde.}
