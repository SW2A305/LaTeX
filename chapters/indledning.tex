\chapter{Indledning}
%Selve opgave formuleringen:
%En større sejlklub råder over ialt 6 sejlbåde, som benyttes af klubbens medlemmer til både undervisning og
%rekreative formål. Undervisningen foregår i faste hold, hvor hvert hold undervises på en bestemt hverdags-
%aften (ma-ti, to-fr). Deltagere kan melde afbud, og gæster (elever fra andre hold) kan besætte ledige pladser.
%En elev skal have gennemført 12 skolesejladser i løbet af en sæson for at kunne rykke op (1. år) eller blive
%indstillet til prøve (2. år). Når bådene bruges rekreativt, kan et medlem reservere en ledig båd udenfor un-
%dervisningstidspunkterne til dag-, weekendture, kapsejladser eller sommerferie. Låneren betaler en mindre
%afgift for brugen af båden. Inden en båd tager afsted, skal der udfyldes et skema med besætning, forventet
%destination og varighed. Tilbagekomst skrives i skemaet, hvor man også kan tilføje bemærkninger om f.eks.
%bådens tilstand og reservationsbehov.
Der findes mange organisationer som dagligt skal holde styr på på medlemmer og på materiel. Et større puslespil skal gå op for at for alle tingene til at gå op i en højere enhed, så alt fungerer og at der ikke bliver glemt noget.

Dette kunne være en sejlklub, som administerer et antal både, hvor medlemmerne af klubben har mulighed for at deltage i undervisning og låne bådene til reakreative formål. I denne rapport vil der fokuseres på netop sejlklubber, og på deres administration af deres materiel og medlemmer. Vi vil undersøge; hvordan hierakiet internt i sejlklubben er, hvordan tilmeldning til undervisning og lån af både foregår, hvilke kriterier der skal opfyldes for at stige i hierakiet i klubben og hvordan forskellige sejlklubber styrer deres administration af alle disse oplysninger.  

Informationer omkring hvordan en sejlklub administeres, vil blive uddraget fra et interview med en repræsentant fra sejlklubben Vestre Baadelaug i Aalborg.

Efter vores undersøgelser om hvordan administrationen foregår, vil der arbejdes på at kode et system, som kan samle alle oplysningerne som en sejlklub bruger, og gøre det mere overskueligt for medlemmerne og administrationen i klubben at holde styr på alle oplysningerne.