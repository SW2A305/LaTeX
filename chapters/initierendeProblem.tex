\section{Initierende Problem}\fxnote{Mange ster deri bliver der gået ud fra sejlkubber, hvilket leder læseren til at tro at vi ikke vil kigge på andre. Overvej omformulering til bare generelle fritidklubber. Samt evt. kilder? --Troels}
Sejlklubber og andre fritidsklubber benytter ofte frivillige i klubben til at hjælpe med administrativt arbejde. Det er
ikke altid lige let at finde frivillige til at hjælpe med klubbens administrative poster, hvilket kan være tidskrævende.
I en sejlklub er der logger over sejlture som skal håndteres, eventuel sejlerskole skal holdes styr på, klubbens interne
begivenheder og andre funktioner relateret til klubben. Sådanne opgaver foregår ofte på papir. Dette kan medføre at
arbejdet kan blive uoverskueligt, tager længere tid end nødvendigt og det bliver lettere at miste informationer.

Denne problemstilling er ikke unikt knyttet til sejlklubber, men et generelt dilemma med administrativt arbejde i
fritidsklubber, specielt klubber der bruger samme faciliteter til både udlejning og undervisning, så som sejlklubbens
fartøjer.

Den følgende analyse i rapporten vil bygges ud fra følgende initierende problem:

\textit{Er det muligt at lave en softwareløsning, der kan gøre frivillige i fritidsklubbers administrative arbejde
nemmere, som stadig er let at benytte uden nødvendigvis at have meget erfaring med anvendelse af computere, og i så fald
hvordan?}

I analysen vil disse problemstillinger blive undersøgt:
\begin{itemize}
\item Hvilke fritidsklubber har administrative opgaver, i form af udlejning og skemalægning for begivenheder mm., der
skal håndteres af frivillig arbejdskraft?
\item Hvilken information skal diverse fritidsklubber håndtere?
\item Er det muligt at organisere de frivilliges arbejde med en softwareløsning?
\item Findes der andre løsninger på markedet til at løse problemet, i så fald, hvorfor benyttes de ikke?
\end{itemize}