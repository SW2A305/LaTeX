\section{Initierende Problem}

Fritidsklubber benytter ofte frivillige i klubben til at hjælpe med administrativt
arbejde.  I en sejlklub er der flere
logbøger over sejlture, som skal håndteres, eventuel skal der også holdes styr på en sejlerskole, klubbens
interne begivenheder og andre funktioner relateret til klubben. Sådanne opgaver foregår ofte på papir. Dette
kan medføre, at arbejdet kan blive uoverskueligt, tager længere tid end nødvendigt og risikoen for at miste
informationer bliver større.

Denne problemstilling er ikke unikt knyttet til sejlklubber, men et generelt dilemma med administrativt
arbejde i fritidsklubber, specielt klubber der bruger samme faciliteter til både udlejning og undervisning, så
som sejlklubbens fartøjer.

Rapportens analyse vil bygges ud fra følgende initierende problem:

\textit{Er det muligt at lave en softwareløsning, der kan gøre frivillige i fritidsklubbers administrative
arbejde nemmere, som stadig er let at benytte uden nødvendigvis at have meget erfaring med anvendelse af
computere, og i så fald hvordan?}

I analysen vil disse problemstillinger blive undersøgt:

\begin{itemize}
  \item Hvilke fritidsklubber har administrative opgaver, såsom af udlejning, skemalægning for begivenheder mm., der
        skal håndteres af frivillig arbejdskraft?
  \item Hvilken information skal diverse fritidsklubber håndtere?
  \item Er det muligt at organisere de frivilliges arbejde med en softwareløsning?
  \item Findes der andre løsninger på markedet til at løse problemet? I så fald, hvorfor benyttes de ikke?
\end{itemize}