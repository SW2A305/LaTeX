\chapter{Perspektivering}

I dette afsnit vil der gives en beskrivelse af forskellige funktioner, som der er blevet vurderet nyttige, hvis projektet skulle videreudvikles. 
Der vil desuden blive diskuteret generelle overvejelser af programmet, som kunne have været udformet anderledes. 

Det har i gruppen været diskuteret, at en SMS-service ville være smart at bruge. 
Denne service kunne bruges til at sende notifikationer på dagen eller dagen før et medlems reservering af en båd finder sted. 
Det kunne også tænkes, at hvis man vil reservere en båd i et tidsrum, hvor båden allerede er reserveret, kan man placere sig selv i kø til båden.
Det betyder at hvis medlemmet, der har den reelle reservering pludselig aflyser, vil den næste i køen, få en sms om at båden nu er fri.
Her kunne der evt. bruges en svarmulighed, så medlemmet, ved at svare på smsen, kan meddele om de vil overtage reserveringen eller ej, dvs. uden at skulle åbne systemet op. 
Et lignende system kan konstrueres ved e-mail i stedet for SMS.

Som programmet er nu, fungerer det ikke online, og det vil sige, at man skal ned i sejlklubben for at kunne reservere en båd, tilmelde sig begivenheder,og generelt bruge programmet. 
En måde at ændre dette på er at lægge databaseforbindelsen online, så der kan laves ændringer i databasen uden at logge på i sejlklubben. 
En online løsning vil gøre brugen af programmet mere fleksibelt, samt hjælpe medlemmerne og gøre brugen af klubbens faciliteter lettere at tilgå.

Som nævnt i analysen vil programmet med nogle ændringer, kunne bruges i andre sportsklubber. 
Et eksempel kunne være en sportshal, som vil gøre det muligt at reservere hallen til diverse arrangementer, melde sig til begivenheder der måtte foregå ol. 
Dog vil dette, som nævnt tidligere, kræve ændringer i programmets funktionaliteter, da løsningen fremstillet til dette projekt, er lavet specifikt til Sejlklubben Sundet.

Der blev i starten af projektperioden diskuteret internt i gruppen, at et system som dette ville tjene sig bedre som værende en hjemmeside.
Grundet manglende kendskab til at udvikle en hjemmeside ved hjælp af C\#, blev dette dog fravalgt og et almindeligt program blev udviklet i stedet. \fxnote{Almindeligt? Jeg ved ikke hvad man ville kalde det, men almindeligt skal ikke bruges. (Søren) - Nikolaj: et program til Windows? Troels: skrivebordsapplikation?}
Ved en løsning udviklet som en hjemmeside, ville det være muligt at anvende mobile enheder, så som smartphones og tablets samt computere med andre styresystemer end Windows, hvilket anses for at være en fordel. \fxnote{Troels: Hjemmeside lyder meget 00'erne, webservice eller website lyder væsenlig bedre. Det er også et dansk ord} 
