\chapter{Perspektivering}

\cbstart

I dette afsnit vil der blive beskrevet forskellige funktioner som der er blevet vurderet nyttige hvis projektet fortsat
skulle udvikles på. Der vil desuden blive diskuteret generelle overvejelser af programmet, som kunne have været udformet
anderledes. 

Der har på gruppen været diskuteret at en SMS-service ville være smart at bruge. Denne service kunne bruges til at sende
notifikationer på dagen eller dagen før et medlems booking af en båd finder sted. Det kunne også tænkes ind at hvis man
vil booke en båd i et tidsrum hvor båden allerede er booket, kan man melde sig på som næste booker. Det betyder at hvis
medlemmet der har den reelle booking pludselig aflyser, vil den næste booker, dermed få en sms om at nu er båden altså
fri til at blive booket i det tidsrum. Her kunne der evt. bruges en svar mulighed, så medlemmet ved at svare på smsen
kan melde om de vil overtage bookingen eller ej, dvs. uden at skulle åben systemet op. Et lignende system kan
konstrueres ved e-mail notifikationer.

Som programmet er nu fungerer ingenting online, og det vil sige man skal ned i klubben for at kunne booke en båd, eller
tilmelde sig begivenhederne i klubben, samt for at tilgå alle de andre funktioner i programmet. En måde at ændre dette på kunne være
at lægge database forbindelsen online, så der kan laves ændringer i databasen uden at logge på nede i sejlklubben. Dette
ville gøre programmet meget mere fleksibelt, samt hjælpe medlemmerne og gøre brugen af klubbens faciliteter lettere at
tilgå.

Som nævnt i analysen vil programmet med nogle ændringer, kunne bruges i andre sportsklubber. Et eksempel kunne være en
sportshal som vil gøre det muligt at booke hallen til diverse arrangementer, melde sig til begivenheder der måtte foregå
og lignende. Dog vil dette som nævnt tidligere kræve ændringer i programmets funktionaliteter, da løsningen fremstillet
til projektet her, er specifikt lavet en sejlklub med sejlerskole, med udgangspunkt i sejlklubben Sundet.

Endeligt er det blevet konkluderet at et system som dette fremstillet til projektet her, ville tjene sig bedre som
værende en hjemmeside. På denne måde kan tilgå funktionerne via ens webbrowser, og tilgå databasen igennem denne
hjemmeside. Hermed undgår man at brugerne skal installere programmet, som de ellers ville være nødt til hvis det
nuværende program brugte en online database forbindelse, som nævnt tidligere. Men da det var angivet af projektoplægget
at man skulle skrive projektet her i C\#, var dette ikke en mulighed, selvom gruppen klart så dette som værende den
bedste løsning.

\cbend