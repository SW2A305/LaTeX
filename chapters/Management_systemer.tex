\section{Management systemer}\label{sec:management-systemer}


Et system, som med fordel kunne anvendes til det formål, er et management system. Et management system er et framework af processer og procedurer, som bruges til administration af frivillige organisationer og virksomheder.  
Eksempler på management systemer findes under afsnittet ``\myref{chap:teknologi-analyse}''. 
Mere præcist er SailingClubManager et management system, hvilket er yderligere dokumenteret i "\nameref{bilag:scm}  (\ref{bilag:scm})". 
Det management system har en grafisk brugergrænseflade (GUI) i web browseren. 
Dette er i kontrast til en konsol applikation, eller en nativ GUI. 
\newline 

Ved at anvende et management system kan en klub eller forening, forudsat at det er succesfuldt, mindske den tid, der bruges på papirarbejde. Særligt fordi at en lang række idrætsforeninger drives at frivillige. 
Ifølge \cite{Frivilligrapporten} er 26\% af de arbejdsopgaver som findes for frivillige sekretariatsarbejde og administrativt arbejde. Med et godt management system kunne de admnistrative opgaver derved bringes ned på mindre tid, og derved kunne de frivillige bruge deres tid på mere interessante ting for organisationen, eller bruge den ekstra tid på noget andet.

%Et management system er et system, som håndterer en række data. Dette kan være mange forskellige slags data, men i denne
%rapport, er der fokus på håndtering af informationer for foreninger, nærmere bestemt sportsforeninger/klubber.

%Sportsforeninger har brug for en måde at håndtere forskellige informationer om deres klub. Det kan f.eks. være
%medlemslister, hvornår der skal kræves kontingent, herunder hvem der har betalt og hvor meget der skal betales, booking
%af forskellige ting; baner, både, haller osv. Det kan være uoverskueligt at holde styr på f.eks. 1000 medlemmer og om
%hvorvidt, de hver især har betalt kontingent eller ej, hvis der bruges et manuelt system i klubben. Et online
%bookingsystem kan også være nyttigt, f.eks. i en tennisklub, for booking af baner. I nogle klubber booker man ved at
%skrive sig op på en tavle, eller man møder op og håber på, der en ledig bane. Her kunne et online bookingsystem være
%behjælpeligt, så medlemmerne slipper for at spilde tid på, at tage hen til klubben, hvis ikke der er en ledig bane.
%\newline
%Mange sportsforeninger er afhængige af frivillige, for at kunne køre økonomisk rundt.\fxnote{Lidt talesprogsagtigt?} Et
%elektronisk system vil nok give flere personer mod på at være frivillige, da det ikke er indbydende at skulle håndtere
%alt det økonomiske manuelt.\fxnote{!Påstand! - der kunne måske laves et spørgeskema for at underbygge dette?}


