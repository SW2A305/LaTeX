\chapter{Interview}\label{bilag:interview}
Som en del af informationsindsamlignen til projektet, er der blevet lavet et kort interview med Jacob Nørbjerg, tidligere skolechef i sejlklubben Sundet. Ideen i interviewet var at få viden omkring hvordan en sejlklub administreres direkte fra en person, som havde haft med administrationen at gøre på nærmeste hold. Dog kan der være forældede informationer i interviewet, da Jacob Nørbjerg ikke længere er skolechef og ikke har været det i et par år. Udover selve interviewet med Jacob Nørbjerg, blev der udleveret dokumenter, som beskriver nogle bestemte administrative strukturer i sejlklubben nærmere, disse dokumenter refereres der også til i den følgende tekst.

På baggrund af interviewet med Jacob Nørbjerg findes der nu informationer om hvordan en større sejlklub, Sundet, med udlån af både og sejlskole administreres. Interviewet var delt op i to sektioner, den første sektion med fokus på, hvordan administrationen fandt sted da Jacob var klubleder, og den anden sektion med fokus på problemerne ved administrationen. Administrationen af sejlklubbens medlemmer foregik på et hjemmelavet IT-system, som holdte styr på informationer omkring medlemmerne, som navn, adresse, telefonnummer, medlemsnummer, om medlemmet var båd ejer eller ej. Systemet kunne desuden håndtere indbetaling fra medlemmerne, for fx lån af værktøj til vedligeholdelse af bådene, og organisationen af bådene når de lå på land, altså plads uddeling. IT-systemet var Access baseret, en udgivelse fra Microsoft, brugt til at skabe brugervenlige databaser uden programmerings viden.
Ud fra både interviewet og de udleverede dokumenter, blev den del af administrationen, som ikke foregår elektronisk forklaret. Ud fra dette kunne det ses, at sejlklubben har faste procedurer i henhold til bådulån og skolesejlads, som alt sammen foregår på papir. I en logbog noteres udlån af bådene, med informationer vedrørende hvem der sejler, hvorhen, forventet tilbagekomst, og noter om båden, som alt sammen skal udfyldes af personen som låner båden. Desuden skal en havari log udfyldes, også på papir, vedrørende eventuelle skader på fartøjet som følge af turen. En sejladsprotokol skal også udfyldes i tilfælde af at der har været undervisning, med informationer vedrørende fører, dato for sejladsen, hvilket elever der var med og eventuelt hvilke gæster som var med. Og med hensyn til reservationer fandtes der lister med fartøjerne, hvor det efter først til mølle princippet galdt om at skrive sig på først. Alle de informationer kunne kun udfyldes på papir i sejlklubbens klubhus, og det var derfor nødvendigt for medlemmer og elever at tage til klubhuset for at undersøge om hvilke både der var ledige, og om de var sat på et hold som skulle ud og sejle osv. 

Problemerne ved administrationen var netop, at informationerne vedrørende hvilke både der var ledige og hvornår der var undervisning alt sammen befandt sig i klubhuset, og ikke var tilgængeligt fra nettet. Desuden var det et større arbejde når der skulle udskrive regninger, da listen med udlån skulle gåes igennem for at se hvilke personer der havde lånt både, og hvor mange gange de havde lånt bådene,  det samme var gældende for krydstjekning af antallet af undervisningstimer for eleverne. Alle former for tilmeldinger og reservationer foregik også på papir, hvor først til mølle principet galdt uden mulighed for at tjekke i forvejen om hvad der var frit/havde frie pladser.

Ud fra interviewet kan det konkluderes, for sejlklubben Sundet, at de mangler et bredere system, som ville kunne indeholde alle de informationer de i øjeblikket skriver ned i diverse protokoller og logfiler. Problemet med den måde de gør tingene på, er at de tvinger folk til at tage ned til klubhuset for at finde de informationer de søger, og samtidigt bliver arbejdet med at krydstjekke informationerne fra diverse papirer et større arbejde, i forbindelse med udskrivning af regninger og elevernes undervisningstimer.  