\chapter{Interview}\label{chap:Interview}
\fxnote{Det bliver nok lidt svært at skrive det her afsnit uden ''vi´´, men man må vel ikke bruge vi i en akademisk rapport?}
Til informationsindsamling til vores analyse valgte vi, i samarbejde med gruppe A304 og 405a, at interviewe kasséren,
Peter, fra Vester Baadelaug og en afløser, Dorte, en anden kassér i Sejlklub Limfjorden. Begge klubber ligger i
Aalborg Lystbådhavn og derfor foretages der et samlet interview. En vigtig forskel mellem de to klubber er at Vester
Baadelaug har flere typer af både, mens Sejlklubben Limfjorden kun har sejlbåde. Selve interviewet blev udført af
repræsentanter fra gruppe 405a, hvor vi andre to grupper havde udarbejdet spørgsmål som blev sendt med repræsentanterne.
Det vi forventer at få ud af interviewet er, at få skabt et overblik over hvordan administrationen i en sejlklub rent
praktisk fungerer, og finde ud af om der nogle problematiske områder indenfor administrationen, som vi kunne fokusere
vores projekt på. I det følgende afsnit findes et resume over de punkter vi fandt vigtigst fra interviewet.

Ud fra vores interview fandt vi frem til at der i Aalborg er 4 kommuneejede havne, hvor sejlklubberne lejer havnene
gratis fra kommunen, hvor de så til gengæld skal stå for at vedligeholde havnen. Dette er interessant for os at vide, så
vi kan danne os et overblik over hvor mange potentielle kunder der ville kunne være til et administrationssyten. 

Derudover spurgte vi til størrelserne af klubberne Vester Baadelaug og Sejlklubben Limfjorden, og de havde hhv. omkring
380 og 150 medlemmer. Desuden har klubberne tilsammen ca. 450 bådpladser til rådighed. Videre derfra spurgte vi ind til
ombygning af medlemskaberne og fandt at der var forskel mellem de to klubber. Vester Baadelaug fører et system hvor det er
ét medlemskab pr. båd, hvorimod Sejlklubben Limfjorden er et medlemskab pr. person. En anden forskel ved de to klubber
var at de har forskellige måder at beregne prisen pr. bådplads. Vester Baadelaug tager efter størrelsen af pladsen i
kvadratmeter, mens Sejlklubben Limfjorden tager efter størrelsen af båden i kvadratmeter. Det blev desuden udtrykt af de
interviewede personer, at pladsuddelingen til medlemmerne var et stort og besværligt arbejde, da der skulle tages
hensyn til mange forskellige faktorer.

Desuden hæftede vi os ved måden hvorpå medlemsdata i Vester Baadelaug blev opdateret. En gang om året blev de nuværende
informationer sendt ud til medlemmerne, og i tilfælde af at noget skulle ændres skrev medlemmerne tilbage, og kasseren
ændrede så i medlemsdataet. Da forhørte os om et system hvor medlemmerne selv kunne opdatere deres informationer,
afviste kasseren dette, han ønskede ikke at medlemmerne selv skulle ændre deres informationer, men han medgav dog at det
medførte \textit{"en helveds masse arbejde"}. Et andet punkt som kasseren fra Vester Baadelaug mente besværliggjorde
administrationen en smule, var nummereringen af medlemmerne, da et lavere nummer betød man havde været medlem i længere
tid, og derved havde førsteret foran de højere numre. Netop den besværlige administration er et problem for alle
sejlklubber mener kasseren Peter. Han sagde at det var problematisk at finde folk, som har interesse og evner til at
administrere sejlklubberne. Den anden kasser, Dorte, udtrykte også at det var problematisk at bruge Sejlklubben
Limfjordens IT-system som afløser. Der blev udtrykt et ønske fra begge kassérer om at slå administrationen af de to
klubber sammen for at gøre det lettere. Dog havde de ikke kunne finde et system som ville kunne administrere to havne
som lå på kryds og på tværs af hinanden og samtidigt skulle kunne indeholde forskellige indstillinger for medlemmerne af
de to forskellige klubber. Vi forhørte os så om hvor meget administrationstid der var for kassererne, og de skød på
omkring 700 timer om året.

Et nyt punkt for os var dog at bådene blev i nogen grad brugt som både kolonihavehuse og endda som faste boliger. Dog
blev det nævnt at ikke alle havne tillader fast bopæl på bådene. 
 
Ud fra vores interview blev det altså slået fast at administrationen af en sejlklub er et stort arbejde, og giver
problemer for dem der står for administrationen. Desuden fik vi et indblik i Vester Baadelaugs og Sejlklub Limfjordens
interne opbygning omkring deres medlemmer og hvordan medlemsdata bliver administreret. Vi fik også et indblik i de IT-systemer 
de to klubber anvender, og hvad de godt kunne tænke sig i forhold til et nyt system.
