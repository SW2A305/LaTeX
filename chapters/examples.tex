\chapter{Eksempler}

{\itshape Dette dokument er primært ment som en hjælp til at se de forskellige muligheder
i rapporten.}

\section{Akronymer}
For lettere at håndtere akronymer, benyttes en pakke, hvor disse defineres én gang, og bruges så i
dokumentet ved hjælp af \textbackslash ac{acronym}. Så vil systemet sørge for, at skrive det fuldt
ud første gang man eksempelvis skriver \ac{AAU} eller \ac{KOT}.

Skulle det derefter ske, at man har behov for at snakke om \ac{AAU} igen, vil den bruge forkortelsen,
og dermed selv holde styr på det. Dette sikrer os bedre mod fejl. :)

\section{Ændringsregistreringer}
For at lette kommunikation med vejleder kan vi registrere ændringer på forskellige måder.

\subsection{Hele afsnit}
\cbstart Man kan markere, at et helt afsnit er blevet ændret eller tilføjet, ved hjælp af kommandoerne
\textbackslash cbstart og \textbackslash cbend, som markerer henholdsvist start og slut på sidebaren.\cbend

Det er også muligt at markere, at noget er blevet slettet, med \textbackslash cbdelete, hvilket er gjort i
dette \cbdelete afsnit.

\subsection{Detaljerede ændringer}
Der er også mulighed for at vise mere detaljerede ændringer. Eksempelvis \added[id=TB]{hvis man tilføjer tekst},
eller \deleted[id=TB]{sletter tekst}. Eller for den sags skyld \replaced[id=TB]{en kombination af de to}{a
combination of the two}.

\section{Matematik}
Matematiske formler er lette at indsætte\ldots

\begin{equation} \label{eq:example}
  f\left(x\right) = \dfrac{a \cdot b}{c}
\end{equation}

\section{ToDo--typer}
Der er fire forskellige typer af noter, og brugen heraf bestemmes af gruppen i fællesskab, så der er enighed om,
hvad der benyttes til hvilket.

\subsection{Generelt}
Først og fremmest er der noterne almindelig\fxnote{Peger på et bestemt sted i teksten.}
og \fxnote*{Fremhæver et tekststykke også.}{med modtagertekst}.

Ligeledes er der advarsler\fxwarning{Dette er en advarsel!} og fejl\fxerror{Dette er en fejl!}. Sidstnævnte
vil desuden forhindre dokumentet i at kompilere, hvis det markeres som værende \emph{final}.

Disse tre typer kan naturligvis også bruges i \emph{stjerneformen}, til at fremhæve et bestemt stykke
tekst som denne tilknyttes.

\subsection{Avanceret}
Til de lidt mere krævende noter kan det være nødvendigt med mere tekst, hvilket kræver et environment.

\begin{anfxnote}{Dette er en opsummering.}
Dette kan benyttes til at få nogle længere noter ind i teksten, hvis der eksempelvis er noget der skal
overvejes, hvor det bare ikke er nok men en lille notits.

Foruden denne anfxnote er der også \emph{anfxwarning}, \emph{anfxerror} og \emph{anfxfatal}, ligesom ovenfor.
\end{anfxnote}


\section{Referencer}
Det er muligt at referere til forskellige ting, eksempelvis kan nævnes \myref{eq:example}.

For at oprette en såkaldt \emph{label}, som senere kan refereres til, benyttes \verb|\label{key}|. Det er god
stil indenfor \LaTeX at benytte prefixes til de forskellige labels, eksempelvis:
\begin{table}[h]
  \begin{center}
    \begin{tabular}{ll}
      \toprule[1.5pt]
      \texttt{chap:}   & chapter              \\ 
      \texttt{sec:}    & section              \\ 
      \texttt{subsec:} & subsection           \\ 
      \texttt{fig:}    & figure               \\ 
      \texttt{tab:}    & table                \\ 
      \texttt{eq:}     & equation             \\ 
      \texttt{lst:}    & code listing         \\ 
      \texttt{itm:}    & enumerated list item \\ 
      \texttt{ex:}     & example              \\
      \bottomrule[1.5pt]
    \end{tabular}
  \end{center}
  \caption{Reference prefixes} \label{tab:reference_prefixes}
\end{table}


\section{Eksempelbokse}

\begin{example}{Overskrift}
  \label{ex:an_example}
  \lipsum[1]
\end{example}



\section{Programmeringskode}

