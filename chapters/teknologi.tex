\chapter{Teknologi}\label{chap:teknologi-analyse}



For at kunne udvikle et godt produkt, der skal kunne bruges i en bådklub, er det vigtigt, at se på hvilke
produkter der allerede er på markedet, altså hvad er ``state of the art''. I denne forbindelse er der fundet
forskellige produkter, som har nogle af de features der efterspørges i et system til en bådklub. 
\newline
Der er f.eks.
et program udviklet af Anderson Software, der hedder BoatCloud.\citep{BoatCloud} BoatCloud består afsnit 3
applikationer, StackTrack, VesselValet og Service Request. StackTrack benyttes, når medlemmerne selv sejler
deres både, hvorimod VesselValet, er når der skal tjenere og passagerer med ombord på bådene. BoatCloud
applikationerne er derfor mere designet til en bådklub, som passer på medlemmer af klubbens egne både, fremfor
klubbens egne både. I applikationerne kan medlemmerne melde at de vil sejle på et bestemt tidspunkt. Medlemmet
kan få klubben til at vaske båden, tanke den, og fylde den op med diverse snacks, alt sammen registreres
igennem denne webbaserede applikation. Applikationen tager imod alle disse bestillinger i realtime, og 24
timer i døgnet. Herudover bliver der sendt e-mails ud til medlemmerne når de har lavet en bestilling. Man kan
logge ind som administrator, og her kan man se alle reservationer der er lavet, samt have mulighed for at
se yderligere detaljer om hver enkelt reservation.

En anden applikation der findes er Sailing Club Manager \citep{SailClub}. Denne applikation er også
webbaseret, og gør det muligt for en bådklub at tilføje hvilke både der er, samt hvor det er fortøjet i
havnen. Man kan i en kalender lave events og bookings, samt medlemstilmeldelse til forskellige events. Applikationen kan også bruges som kontaktmedium for klubben til deres medlemmer. Klubben kan sætte et
e-mail system op, samt tilføje et template som bliver sendt med hver enkelt e-mail. Applikationen kan endvidere
holde styr på økonomien, man kan tilføje en bankkonto, samt  holde styr på fakturaer, og endda sende
dem og håndtere det online. Man kan tilføje medlemsskaber, med forskellige oplysninger omkring medlemmerne der
har netop dette medlemsskab i klubben.

Ud fra denne undersøgelse er flg. features altså fundet i applikationerne:

\begin{itemize}
	\item Medlemmer kan booke afgange.
	\item Medlemmer kan melde sig på events i bådklubben.
	\item Medlemmer kan betale igennem hjemmesiden.
	\item Medlemmerne kan bede klubben om at gøre deres egen båd klar, med forskellige aftaler klubben håndterer
        før den aftalte tid.
	\item Bådklubben kan vise hvilke både der er i klubben.
	\item Bådklubben kan oprette events, og afsætte hvilke medlemmer der kan melde sig på eventet.
	\item Bådklubben kan oprette forskellige medlemsskaber og opkræve betalinger igennem hjemmesiden.
	\item Bådklubben kan sende e-mails med klubbens egne templates gennem hjemmesiden, herunder påmindelser om en
        bookning nogen tid før.
\end{itemize}

I bilag \myref{bilag:scm} findes der screenshots af nogle af de features der findes hos SailingClubManger.

Ud fra teknologi- og organisationsanalysen kan det uddrages at sejlklubber med fordel kan anvende et IT-system til at administere deres administrative opgaver. Der findes desuden en liste med udvalgte features fra andre administrationsprogrammer i dette afsnit, som med fordel kunne tages med videre til problemløsningsdelen. 
Endvidere kan der, ud fra de to nævnte afsnit, tænkes på et overordnet struktur for et program, som kunne hjælpe sejlklubber og andre fritidsklubber med at varetage deres administrative opgaver.  
