\chapter{Teknologianalyse}\label{chap:teknologi-analyse}

I det følgende kapitel findes en definition af management systemer og en analyse af de programmer, som på nuværende tidspunkt er state of the art inden for administration af sejlklubber.

\section{Management system}\label{subsec:management-systemer}
Projektgruppen definerer management system til at være et program, som håndterer og digitaliserer administrativt arbejde.
Med administrativt arbejde i forbindelse med en sejlklub menes der f.eks. håndtering af reservationer, opkrævning af penge, kommunikation osv.

Eksempler på management systemer findes i \myref{sec:sota}.

\section{State of the art}\label{sec:sota}

For at kunne udvikle et godt produkt, der skal kunne bruges i en sejlklub, er det vigtigt, at se på hvilke produkter der allerede er på markedet, altså hvad der er ``state of the art''. 
I denne forbindelse er der fundet forskellige produkter, som har nogle af de features, der efterspørges i et system til en bådklub.


\subsection*{BoatCloud}

BoatCloud er udviklet af Anderson Software \citep{BoatCloud}.
BoatCloud består af 3 applikationer, StackTrack, VesselValet og TicketTracker.
StackTrack benyttes, når medlemmerne selv sejler deres både, hvorimod VesselValet, er når der skal tjenere og passagerer med ombord på bådene. 
BoatCloud applikationerne er derfor mere designet til en sejlklub, som passer på medlemmernes både, fremfor at klubben har både til udlejning. 
I applikationerne kan medlemmerne meddele, at de vil sejle på et bestemt tidspunkt. 
Medlemmet kan få klubben til at vaske båden, tanke den, og fylde den op med diverse snacks, alt sammen registreres igennem denne webbaserede applikation, dette foregår i TicketTracker. 
Applikationen tager imod alle disse bestillinger 24 timer i døgnet. 
Herudover bliver der sendt e-mails ud til medlemmerne, som bekræftelse når de har lavet en bestilling.
Man kan logge ind som administrator, og her kan man se alle reservationer, der er lavet, samt have mulighed for at se yderligere detaljer om hver enkelt reservation.


\subsection*{Sailing Club Manager}

En anden applikation der findes er Sailing Club Manager \citep{SailClub}. 
Denne applikation er også webbaseret, og gør det muligt for en bådklub at tilføje både, samt hvor disse er fortøjet i havnen. 
Man kan i en kalender lave begivenheder og reservationer, desuden er der medlemstilmeldelse til forskellige begivenheder. Applikationen kan også bruges som kontaktmedium for klubben til deres medlemmer. 
Klubben kan sætte et e-mail system op, samt tilføje et template, som bliver sendt med hver enkelt e-mail.
Applikationen kan endvidere holde styr på økonomien, man kan tilføje en bankkonto, samt holde styr på fakturaer, og sende dem og håndtere det online. 
Man kan tilføje medlemsskaber, med forskellige oplysninger omkring medlemmerne, der har netop dette medlemsskab i klubben.

\subsection*{ForeningLet}

ForeningLet er en applikation, som adskiller sig fra de andre ved ikke at henvende sig specifikt til én type klub, desuden er det en dansk applikation. 
Applikationen er web-baseret og har mange nyttige funktioner for en sejlklub som Sundet. 
Applikationen har en medlemsdatabase, et regnskabssystem samt en kontaktfunktion, som kan benytte både e-mail og SMS. 
Der kan også opkræves betalinger fra medlemmerne gennem systemet. 
Applikationen har også en reservationsfunktion, hvilket klubber som Sejlklubben Sundet kunne anvende til at administrere udlejning af deres både. 
Desuden kan der oprettes begivenheder i applikationen, som medlemmerne også kan tilmelde sig gennem applikationen. ForeningLet giver også foreningen mulighed for at oprette deres egen hjemmeside samt deres egen online butik. 

Ud fra undersøgelsen af disse tre programmer er flg. funktioner fundet i applikationerne:

\begin{itemize}
  \item Medlemmer kan reservere afgange.
  \item Medlemmer kan melde sig på begivenheder i bådklubben.
  \item Medlemmer kan betale igennem hjemmesiden.
  \item Medlemmerne kan bede klubben om at gøre deres egen båd klar, med forskellige aftaler klubben håndterer
        før den aftalte tid.
  \item Bådklubben kan vise hvilke både der er i klubben.
  \item Bådklubben kan oprette begivenheder, og afsætte hvilke medlemmer der kan melde sig på begivenheden.
  \item Bådklubben kan oprette forskellige medlemsskaber og opkræve betalinger igennem hjemmesiden.
  \item Bådklubben kan sende e-mails med klubbens egen skabelon gennem hjemmesiden, herunder påmindelser om en
        kommende reservation.
\end{itemize}
\section{Konklusion af teknologianalyse}

Ud fra state of the art afsnittet kan der konkluderes, at der findes programmer, som administrerer udlån/udlejning af både.
Specielt ForeningLet har denne funktion, hvilket netop ville være en funktion som Sejlklubben Sundet ville kunne benytte sig af.
Flere af funktionerne, som findes på tværs af de tre programmer, anses for at være relevante for dette projekt, og vil derfor blive taget med videre til problemløsningsdelen.
Dog er der ikke nogen af disse tre programmer, der har undervisningsfunktioner, hvilket Jacob Nørbjerg gjorde udtryk for kunne være brugbart i \myref{bilag:interview-transkribering}. Derfor menes der, at der her findes en mangel indenfor de produkter, der allerede findes.
Netop manglen af undervisningsfunktioner kunne være en grund til, at Sejlklubben Sundet ikke anvender et af disse produkter.
Et lignende produkt med undervisningsfunktioner vil derfor kunne formodes at være en bedre løsning for Sejlklubben Sundet.

