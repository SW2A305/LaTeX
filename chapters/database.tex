\chapter{Persistenslag}\label{chap:database}

Dette kapitel omhandler forsøget på at opnå persistens mellem kørsler af programmet.
Det ønskes, at programmet kan lagre sit data, udenfor RAM, for at kunne bruge det igen senere. 
Det var et fælles ønske fra gruppens side, at der blev brugt tid på at tilføje dette element til programmet. 

\section{Persistensløsninger}

Der findes flere løsninger til denne problemstilling. 
Der kan vælges en bestemt database ud fra de eksisterende databaser, eller dataet kan gemmes på flade filer.

\subsection{Flade filer}
En simpel løsning på persistensproblemet ville være at gemme alt data på en flad fil ved programlukning, og læse filen igen ved programmets opstart.
Denne løsningen bliver dog hurtig upraktisk, da flade filer ikke har samme evne til at skalere til et større system som egentlige databaser.

\subsection{Databaser}
Efter research, herunder rådgivning fra OOP lektor Rene Hansen, stod valget mellem to databaser: Entity Framework og SQLite.

\subsubsection*{Entity Framework}
Den første af de to, \ac{EF}, er en Object-Relational Mapping (ORM) til .NET Frameworket.
Et ORM skaber et billede af objekterne i hukommelsen i databasen \citep{object_relational_mapping}. 
Herunder var rådgivningen fra Rene Hansen, at anvende Code-First udvikling.
Med Code-First designes objekterne først, i form af modelklasser, hvorefter databasen og tabellerne i denne genereres ud fra sammenkoblingen mellem de forskellige modeller. 
Denne sammenkobling sker i såkaldte migrations, som opretter, ændrer og sletter tabeller, afhængigt af ændringerne i de benyttede modeller.
Fordelen ved denne løsning er, at mange tekniske valg træffes af programmet, så programmøren ikke behøver at have et kendskab til \ac{SQL}, for at kunne benytte databasen og dermed opnå persistens.
Dette gør \ac{EF} til en attraktiv løsning for udviklere, der gerne vil hurtigt i gang med at programmere en løsning, uden først at sætte sig ind i databaseprogrammering. \citep{entity_framework}

\subsubsection*{SQLite}
Den anden kandidat er SQLite, som er et \ac{RDBMS}, hvilket betyder, at det benytter \ac{SQL} til de forskellige former for datahåndtering, såsom INSERT til at indsætte data, og SELECT til at hente data fra databasen.
SQLite er en letvægts embedded database, der er velegnet til mindre programmer, som har behov for lokal persistens.
At en database er embedded betyder, at den ikke kører i et separat program, men inkluderes som et bibliotek i programmet.
SQLite implementerer det meste af SQL-standarden, men udelader de dele, der af SQLite-udviklerne er bedømt til at være for tunge eller unødige til at have med.
Målet med SQLite er at bibeholde letvægten, da systemet også benyttes på mange mobile enheder, såsom Android \citep{sqlite_android}.
SQLite opfattes af dets udviklere, som værende den mest distribuerede SQL database i verden \citep{sqlite_most_deployed}. \citep{sqlite}

Begge databaser blev implementeret under projektarbejdet, grundet komplikationer med førstevalget af database, Entity Framework. 


\section{Database Abstraction Layer}
Et \ac{DAL} bruges til at separere programmet fra databasen.
Dette gør det muligt at udskifte det underliggende persistenslag, samt at lave lag, der simulerer persistens, ved udelukkende at gemme data i hukommelsen, så længe programmet kører.
Man udstiller et interface til programmet, som så anvendes, og derfor ikke en direkte forbindelse. 
Hver tabel som findes i den database programmet skal anvende, skal have sin egen \ac{DAL}, som er logikken der forbinder til databasen.
Der findes fire basale operationer til en database, kaldt CRUD: Create, Read, Update, Delete. 
Hver af disse operationer skal DAL'et udstille, og yderligere metoder kan tilføjes efter behov.

\section{Den valgte løsning}

Efter anbefaling fra OOP lektor Rene Hansen, valgte gruppen at fokusere på \acl{EF} med Code-First, grundet manglende kendskab til \ac{SQL}.
Det blev yderligere bestemt, at der skulle designes et \ac{DAL}, med henblik på at øge muligheden for at teste systemet.

\subsection{Design af \ac{DAL}}

Der blev udviklet et generisk \ac{DAL}-interface, som hver af de konkrete \ac{DAL}-interfaces nedarver fra.
Interfacet definerer otte metoder, som beskrevet i \myref{tab:ourExtCrud}.

\begin{table}[h]
    \colorlet{shadecolor}{gray!40}
    \rowcolors{1}{white}{shadecolor}
    \begin{tabular}{p{2cm}|p{13cm}}
    \textbf{Metode}   & \textbf{Forklaring}       \\ \hline
    \texttt{Create}   & Opretter et eller flere nye elementer i databasen.                                           \\ \hline
    \texttt{Update}   & Opdaterer et eller flere elementer i databasen.                                              \\ \hline
    \texttt{Delete}   & Fjerner et eller flere elementer fra databasen.                                              \\ \hline
    \texttt{GetAll}   & Findes i to udgaver, en som henter alle elementer, og en hvor der kan medsendes et prædikat. \\ \hline
    \texttt{GetOne}   & Henter et element i databasen ud fra det angivne itemId.                                    \\ \hline
    \texttt{LoadData} & Findes i to udgaver, de henter de indre referencetype objekter til elementet eller elementerne givet. \\ \hline
    \end{tabular}
    \caption{Forklaring af hver metode i det generiske DAL interface.}
    \label{tab:ourExtCrud}
\end{table}

De konkrete interfaces defineres for hver modelklasse og implementeres i specifikke klasser, der kommunikerer med et persistenslag.
For at gøre det lettere at udskifte \ac{DAL}-implementation, blev en statisk hjælpeklasse, \texttt{DalLocator}, oprettet, hvis opgave består i at returnere en implementation af et \ac{DAL}-interface. Hermed skal der kun foretages ændringer i en enkelt fil, hvis et nyt DAL tages i brug.
På \myref{img:Program_flow} kan man se hvordan \ac{DAL} indgår i programmets struktur.

\subsection{Første implementation --- \acl{EF}}\label{subsec:Pwoblem}

\acl{EF} blev implementeret for modellerne, og systemet blev bygget herpå.
Under programmets udvikling stod det klart, at det ikke var helt problemfrit at benytte \ac{EF}.

Migrations kom i vejen for hinanden og gav konflikter, der krævede, at gruppens medlemmer regelmæssigt skulle slette hver tabel og kalde kommandoen \texttt{Update-Database}.
Denne kommando får \ac{EF} til at gennemløbe alle migrations og udføre handlingerne i disse.

Systemet reagerede også sporadisk, hvor det ét sted fungerede som det skulle, mens det fejlede lydløst, altså uden en fejlbesked, andre steder.
Dette til trods for, at de to områder var programmeret ens.

\subsection{Anden implementation --- SQLite}

Gruppen aftalte at forsøge en implementation af SQLite i stedet for.
Det var på dette tidspunkt, at det udviklede \acl{DAL} viste sin styrke, da hele persistenslaget kunne udskiftes uden konflikter i koden.

Den implementerede løsning er relativt simpel i sin konstruktion, især i forhold til \acl{EF}.
Dette blev gjort for at sikre, at det ikke ville tage mere end et par dage at implementere og teste.

SQLites enkeltbruger-design har dog også givet problemer, idet databasen til tider angiver, at den er låst af en anden proces.
Disse problemer kan forårsages af flere ting, og virker til at opstå tilfældigt, så det er svært at forudse, hvor og hvornår det vil ske.

\subsection{Ekstra implementation --- Mock-data}

Som en backup--løsning, i tilfælde af, at SQLite pludseligt fejler, er en mock--implementation designet.
Denne implementation simulerer persistens, men gemmer kun i RAM mens programmet kører, og slettes derefter.

\subsubsection*{Mock-data-persistens}

Idet persistens er vigtigt for gruppen, blev der udviklet et persistenslag til mock-data-implementationen.
Dette persistenslag benytter ligeledes SQLite, men gemmer kun data når programmet lukkes, og henter det igen når programmet starter op.
Denne løsning er skrøbeligt overfor eventuelle computernedbrud, idet data i så fald ikke ville blive gemt i databasen.
Det ville være en mulighed at gemme data hver gang der trykkes på ``Log ud''--knappen, eller måske at have en decideret ``Gem''--knap. 