\section{Windows Presentation Foundation}
Windows Presentation Foundation (WPF) er en grafisk brugergrænseflade på Windowsbaseret applikationer. 
WPF gør brug af  bl.a. vektorbaseret rendering af grafik, databinding, til nem redigering af data igennem den grafiske brugergrænseflade, og har også inkluderet Extensible Application Markup Language (XAML), hvilket er en nem måde at skabe WPF-grafik på.\citep{wpf} 

I dette projekt bruges WPF til at skabe den grafiske brugergrænseflade. Valget stod mellem WPF og Windows Forms (WinForms), hvilket er forgængeren til WPF til generering af grafik til Windows. 
Valget faldt på WPF, da XAML, som WinForms ikke har, gør det let at hurtigt lave en grafisk brugergrænseflade. 
Microsoft har også informeret om, at der ikke længere tilføjes nye funktioner til WinForms, men kun udelukkende rettelser af fejl.\citep{winforms}

En helt anden mulighed, som blev diskuteret, var at bruge en hjemmeside, som ville have den fordel, at den kan køre på stort set alle enheder, som har adgang til internettet via en browser. En hjemmeside, som gør brug af C\#, vil kunne laves lidt på samme måde som ved WPF; Med HTML (HyperText Markup Language) og CSS (Cascading Style Sheets) som frontend og C\#-klasser som backend. 
En hjemmeside blev fravalgt, da der blev vurderet at WPF var nemmere at oprette end en hjemmeside og at en grafisk brugergrænseflade ikke er det essentielle i opgaven. 

\subsection{Ofte anvendte WPF-controls}
Der anvendes ofte nogle få WPF-controls til at bygge brugergrænsefladen.
Disse vil blive beskrevet her.

\subsubsection{ComboBox}
I WPF er en ComboBox det element som andre steder omtales som en dropdown menu. 
Dens indhold kan instilles enten i XAML-koden eller i Code-Behind koden.
Hvis dette indhold skal være dynamisk, vil Code-Behind ofte være anvendt.
Eksempelvis hvis man kun vil have de medlemmer af en liste som opfylder et givent prædikat. 

\subsubsection{Button}
En Button er en knap, knappen svarer til en method i Code-Behind koden. 

\subsubsection{DataGrid}

\subsubsection{Checkbox}

\subsubsection{Stackpanel}

\subsubsection{TextBlock}

\subsubsection{TextBox}

\subsubsection{ListBox}

\subsubsection{UserControls}


