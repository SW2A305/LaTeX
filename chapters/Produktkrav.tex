\chapter{Produktkrav}

Igennem problemanalysen blev det gjort klart, at der findes et problem ved fritidsklubber, særligt Sejlklubben Sundet, når det kommer til management af deres ressourcer samt dokumenthåndtering. 
Problemet kan løses ved et management system, til at håndtere dokumentation for sejlklubben og gøre det overskueligt. 
I følgende afsnit undersøges der, hvilke krav et managements system for Sejlklubben Sundet skal have. 


\section{Funktionelle krav} \label{sec:funktionelleKrav}

Den nødvendige dokumentation indebærer logbøger, bådreservationer, besætningslisten, tilmeldinger til begivenheder m.m. 
Som grundlag for at alt dette kan fungere, kræves der en metode til at sætte sig selv på en aktivitet. 
For at gøre dette muligt kræves der således to ting: En database af medlemmer, samt en form for begivenhedskalender. 
Foruden dette kræves der information om de forskellige begivenheder og hvem der er tilmeldt. 
Der kræves også et system til udlejning af både samt undervisning.

Produktet bør have følgende funktioner i prioriteret rækkefølge:
\begin{itemize}
  \item Medlemshåndtering
  \item Brugerlogin
  \item Oversigt over undervisning
  \item Bådreservation
  \item Logbog
  \item Vise begivenhedsinformation
\end{itemize}


%Dette afsnit skal uddybes meget mere, har skrevet et eksempel på hvad jeg mener afsnittet skal indeholde

\section{Uddybning af funktionelle krav}

Der ligger mere bag de ovenstående elementer end \myref{sec:funktionelleKrav} giver udtryk for, disse vil her blive set nærmere på fra en datalogisk synsvinkel.

\subsection{Medlemmer}

For at kunne integrere medlemmer i programmet, skal der være en overordnet klasse, der danner base for alle
medlemmer. 
Da der også kan komme gæster i klubben, er det relevant at lave en personklasse over dette.
Yderligere specifikationer kan så laves igennem subklasser af personklassen og på denne måde få lavet forskellige
instanser af f.eks. elever, undervisere, administratorer osv. 
Da Sundet allerede har en database med deres medlemmer, skal der tages højde for at importere dette data over i en anden database\fxnote{Menes der vores database eller hvad?}, så Sundet ville kunne have denne mulighed, hvis behovet skulle opstå.

\subsection{Login system}

For at medlemmerne skal kunne få en personlig oplevelse i programmet, er det vigtigt at have et loginsystem, som vil
være relevant ved f.eks. bådreservation og begivenheder. Hvis medlemmet skal kunne se hvilke begivenheder de deltager i, eller kunne reservere en båd, er det smart at reservationen kan kobles på objektet af medlemmet. 
Dette kan registreres på en nem måde, ved at medlemmet er logget ind på deres personlige bruger i systemet.


\subsection{Begivenhedsinformation}

Begivenhederne skal også gemmes i en database, således de kan tilgås og gemmes fra runtime til runtime af programmet. 
Derfor er det også et krav at håndtere objekter af begivenheder i en database. 
Her skal der være information omkring den pågældende begivenhed, såsom dato, start tidspunkt, og andre informationer relevante for medlemmernes deltagelse til
begivenheden.

\subsection{Begivenhedskalender}

Når begivenhederne kan findes i databasen, skal den kunne forbinde dem med kalenderen, som også skal indbygges i
programmet. Her skal det være muligt for programmet at vise, hvornår der sker begivenheder og opsætte begivenhederne i den rigtige kronologiske rækkefølge så det er nemt at finde rundt i for klubbens medlemmer.

\subsection{Bådreservation}

Der skal oprettes objekter af bådene, som findes i klubben og på denne måde reservere disse objekter, på den pågældende dag. 
Bådene, samt reservationerne, skal også gemmes i databasen, så de kan findes fra runtime, til runtime af programmet.
Foruden dette skal bådene også have en log omkring de forskellige sejladser, der er blevet foretaget i bådene. Disse logs er blevet omtalt og uddybet i \myref{bilag:sundet}

\subsection{Oversigt og kommunikation for undervisning}

Det skal være muligt for underviserne og eleverne at kunne holde styr på deres undervisningsdage, se hvilke typer
undervisning de har haft hvornår, samt at se hvilke typer undervisning de mangler for at kunne bestå. Underviserne skal også i den forbindelse kunne se hvilke elever der er på de forskellige hold, og samtidig være i stand til at skrive hvem der er deltaget hvornår.
