\chapter{Produktkrav}

Igennem problemanalysen blev det gjort klart, at der findes et problem ved fritidsklubber, særligt Sejlklubben Sundet, når det kommer til management af deres ressourcer samt dokumenthåndtering. 
Problemet kan løses ved hjælp af et management system, til at håndtere dokumentation for sejlklubben og gøre det overskueligt. 
I følgende afsnit undersøges der, hvilke krav et managements system for Sejlklubben Sundet skal have. 


\section{Funktionelle krav} \label{sec:funktionelleKrav}

Den nødvendige dokumentation indebærer logbøger, bådreservationer, besætningslisten, tilmeldinger til begivenheder m.m. 
Som grundlag for at alt dette kan fungere, kræves der en metode til at sætte sig selv på en aktivitet. 
For at gøre dette muligt kræves der således to ting: En database af medlemmer, samt en form for begivenhedskalender. 
Foruden dette kræves der information om de forskellige begivenheder og hvem der er tilmeldt. 
Der kræves også et system til udlejning af både samt undervisning.

Produktet bør have følgende funktioner i prioriteret rækkefølge:
\begin{itemize}
  \item Medlemshåndtering
  \item Brugerlogin
  \item Oversigt over undervisning
  \item Bådreservation
  \item Logbog
  \item Vise begivenhedsinformation
\end{itemize}


%Dette afsnit skal uddybes meget mere, har skrevet et eksempel på hvad jeg mener afsnittet skal indeholde

\section{Uddybning af funktionelle krav}

Der ligger mere bag de ovenstående elementer end \myref{sec:funktionelleKrav} giver udtryk for, disse vil her blive set nærmere på.

\subsection{Medlemmer}

Det er et krav at man kan udnytte medlemmerne i klubben, så man kan registrere hvem der er tilmeldt de forskellige reservationer eller lignende. 
Systemet skal kunne kende forskel på medlemmer, så man ikke kan have flere af det samme medlem på en reservation eller en begivenhed.
Forskellige informationer skal kunne findes på medlemmerne i programmet, så brugerne kan finde de rigtige medlemmer og markere om de mødte op.

\subsection{Login system}
Det skal være muligt at registrere hvem der er logget ind i programmet, for på denne måde at give forskellig adgang til diverse funktionaliteter i programmet.

\subsection{Oversigt over undervisning}

Det skal være muligt for elever ved sejlerskolen at se hvornår deres næste lektion finder sted.
Herudover skal de også kunne se hvor langt de er kommet i forbindelse med deres uddannelse, hvilke milepæle de har udført, mm.
Underviserne skal have mulighed for at oprette nye lektioner og angive hvilke læremål eleverne udførte i den pågældende lektion. 
Underviserne skal også kunne oprette, slette og redigere hold på skolen hvis nogle skulle melde sig ud, eller ved sæsonstart.

\subsection{Bådreservation}

Medlemmerne i klubben har muligheden for at reservere klubbens både, og skal derfor kunne gøre dette i programmet.
Derfor skal der være en måde hvorpå man kan tjekke om en båd er booket i en given tidsperiode, og samtidig en måde hvorpå man kan reservere båden samt slette en reservation.
Herudover skal det være muligt af afmelde sig en booking hvis man ikke deltager alligevel, og dermed åbne op for nye besætningsmedlemmer.


\subsection{Logbog}

I forbindelse med en sejltur, skal der kunne gemmes en logbog over turen med informationer, som beskrevet i \myref{subsec:bådudlån}.
Der skal her være registreret på hvilken båd turen foregik, så de korrekte logbøger kan findes for en given båd.

\subsection{Vise begivenhedsinformation}

Endeligt skal man kunne gemme begivenheder i klubben, med en dato og en beskrivelse af begivenheden, samt slette disse igen, hvis de skulle blive aflyst. 
Det skal være muligt at til- og afmelde sig begivenheden, foruden at kunne se hvilke medlemmer der er tilmeldt.
Desuden skal der kunne vise en liste over begivenheder sorteret efter dato, så man kan danne et overblik over hvornår der sker hvad i klubben.
