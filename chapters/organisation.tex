\chapter{Organisation}\label{chap:organisation}

I dette kapitel vil der blive set på sejlklubber som en organisation.
Der vil undersøges, hvordan medlemmer kategoriseres, hvordan undervisningen organiseres, samt hvordan bådudlån og andre administrative opgaver håndteres. 
Undersøgelsen omhandler tre forskellige klubber: Sejlklubben Sundet og Bådklubben Valby, som begge er beliggende i København og Aalborg Sejlklub, som har til huse i Aalborg. 
Der vil primært blive fokuseret på Sejlklubben Sundet, da der er mest information til rådighed om dem.
Dette i form af interviewet med Jacob Nørbjerg, som blev nævnt i \myref{chap:interessent-analyse-ved-sejlklubber}, samt en mere informativ hjemmeside, i forhold til de to andre klubber. 
En bredere forståelse, for den organisatoriske opbygning, kunne være opnået ved, at flere repræsentanter fra Sejlklubben Sundet blev interviewet, men dette har der ikke været mulighed for.
Aalborg sejlklub og Bådklubben Valby er med i undersøgelsen for sammenligningens skyld, men har ikke været tilgængelige for interviews. 
Denne mangel på information om de to andre klubber, gør at betingelserne for en sammenligning ikke er optimale, men dette anses i denne sammenhæng for værende acceptabelt.

\section{Medlemmer}\label{sec:organisation-medlemmer}

Det er meget forskelligt fra sejlklub til sejlklub, hvordan medlemmer kategoriseres, hvis dette overhovedet bliver gjort.
Sejlklubben Sundet kategoriserer deres medlemmer således: Voksen--, bådejer--, gaste--, mini-kølbåd--, ungdoms--, passiv-- og støttemedlem. 
Desuden har de en særlig æresmedlemskategori.
Denne status opnås ved at have ydet en mangeårig, ekstraordinær indsats for klubben og sejlsporten generelt.\citep{sundet_vedtaegter}

Til sammenligning har Aalborg sejlklub kategoriseret deres medlemmer som følger: aktive, passive, junior-- og æresmedlemmer \citep{aalborg_sejlklub_vedtaegter}.

Bådklubben Valby kategoriserer som A--, B-- og C--medlemmer, som indikerer hvilken bådtype de har \citep{badklubben_valby_love}.

Det formodes, på baggrund af \myref{bilag:interview-transkribering}, at fælles for medlemmerne er, at de gerne vil have informationer fra klubberne, så de ved hvilke begivenheder, der finder sted. 
Det kan være kapsejladser, foredrag o.l.


\section{Sejlerskole}\label{sec:sejlerskole}

Sejlklubben Sundet og Aalborg Sejlklub har en sejlerskole, hvor man kan opnå et duelighedsbevis til båd.
Sejlklubben Sundet kalder deres udgave af duelighedsbeviset et førerbevis og har yderligere krav.
Bådklubben Valby adskiller sig fra de to andre, ved ikke at have en decideret sejlerskole, men de har kurser, man kan tage hen over vintersæsonen, for at opnå et duelighedsbevis \citep{baedklubben_valby_duelighedsbevis}.

I Sejlklubben Sundets sejlerskole er der undervisning i hverdagene fra kl. 18 til 21, i månederne maj, juni, august og september. 
Uddannelsen varer to år, og man lærer at sejle i bådtyperne Drabant og Gaffelrigger. 
I løbet af en sæson sejler man typisk 18 gange. 
Er man forhindret i at møde op, skal man ringe og melde afbud.
Hvis man ikke møder op tre gange, uden at have meldt afbud, mister man sin plads i sejlerskolen.
Undervisningsforløbet afsluttes med en praktisk prøve, hvor man opnår sit førerbevis, i så fald man består \citep{Sundet}.
Aalborg sejlklubs undervisningsforløb minder om Sejlklubben Sundets \citep{aalborg_sejlklub_sejlerskole}.


\section{Bådudlån og andre administrative opgaver}\label{subsec:bådudlån}

Man kan i Sejlklubben Sundet, leje både til sejlads, når de ikke er i brug til undervisning. 
Prisen for leje af bådene er højere i weekenderne end i hverdagene, men hvis en af skolens elever er med på sejladsen, er det gratis at leje en båd. 
For at kunne leje en båd, skal der være mindst én bådfører med.
Hver gang der skal sejles ved Sejlklubben Sundet, er der meget data, der skal skrives ned, blandt andet i en logbog, af forskellige årsager.
Det er vigtigt at vide, hvem der er med på båden tildels for at sikre, at besætningen er i stand til at føre båden, men også af sikkerhedsmæssige årsager, hvis der skulle ske en ulykke. 
En liste, over hvilken information der skal skrives ned, er udformet og kan ses i \myref{bilag:sundet}. 
På Sejlklubben Sundets website ses det, at forskellige værktøjer, bl.a. en doodle, er taget i brug, i et forsøg på at øge brugervenlighed og interaktion mellem deltagere \citep{SundetUdlaan}. 
Som Jacob Nørbjerg fortalte i interviewet, skulle man før i tiden ned i klubhuset, for at se om der var en ledig båd, man kunne reservere, men dette kan nu gøres gennem et offentligt bookingsystem fra Supersaas \citep{Booking}.
Jacob Nørbjerg giver i interviewet også udtryk for, at det er svært at få viden omkring klubbens foretagender hjemmefra. 
Nuværende praksis er, at man skal ned i klubben for at se og melde sig på diverse begivenheder, såsom 24-timers sejladser, eller for at se om der er en ledig plads på en sejltur.

Håndtering af information omkring klubbens medlemmer foregår på en lokal computer i Sejlklubben Sundets klubhus. 
Endvidere bruges denne computer til at udprinte girokort til betaling af bådlån. 
Da informationerne er gemt lokalt, betyder det, at de frivillige, som er ansvarlige for udprintningen af girokort, skal ned i klubhuset for at udføre denne opgave.


\section{Konklusion af organisationsanalyse}\label{sec:organisation-konklusion}

Sejlklubben Sundet har mange administrative opgaver, som håndteres manuelt og til dels på en ustruktureret måde. 
Som Jacob Nørbjerg giver udtryk for i interviewet, er det svært at holde sig opdateret med, hvad der sker i klubben.
Ud fra dette kan der konkluderes at en softwareløsning, hvori der implementeres en kalender, som kan holde styr på alle begivenhederne, vil være et godt supplement til administrationen af klubben. 
For at give eleverne bedre mulighed for at kunne af-- og tilmelde sig, så de har mindre risiko for at miste deres plads i sejlerskolen, kunne der med fordel implementeres en sådan funktion i en softwareløsning. 
Det nuværende bookingsystem er åbent for offentligheden og er derfor redigerbart af alle.
Dette er problematisk, derfor vil en forbedret udlånsfunktion med fordel kunne udvikles.

Da der har været flest informationer til rådighed for Sejlklubben Sundet, som dermed også har haft de bedste konkretiserede krav til funktioner, vil der fremover blive fokuseret på at lave en løsning til netop dem. 
Dog skal det nævnes, at systemet der udvikles til Sejlklubben Sundet, vil kunne bruges af de andre sejlklubber, enten ved at ændre eller slette funktioner. 

Det lader til at f.eks. Sejlklubben Sundet har brug for det projektgruppen definerer som et management system, denne findes i \myref{chap:teknologi-analyse}.