\chapter{Organisation}\label{chap:organisation}


I dette afsnit vil der blive set på diverse interessenter med relevans for bådklubber. Der vil for
interessenternes vedkommende blive set på hvad de hver især kunne få gavn af i henhold til klubben. På
samme tid vil der også blive set på eksempler på andre organisationer der har lignende behov som en bådklub,
eksempelvis jagtklubber. Det er værd at notere at hver enkelt klub har forskellige behov, der eksisterer både
små lokalklubber hvor nogle af de interessenter der vil være omtalt ikke eksistere samt større klubber der
omfatter alle punkter, disse primært lokaliseret for Danmarks vedkommende i hovedstadsområdet.


\section{Dansk sejlunion}

Dansk sejlunion er et forbund, som blev dannet i 1913.
%[http://www.sejlsport.dk/mere/dansk-sejlunion/historie] 

Deres mission er at være det nationale samlingspunkt for alle sejlere. Dansk sejlunion er tilsluttet Danmarks
Idrætsforbund, International Sailing Federation og andre lignende organisationer inden for sejlsport. Dansk
sejlunion tilbyder også services i form af rådgivning og aktiviteter til klubber, sejlklasser og andre
samarbejdspartnere.
%[http://www.sejlsport.dk/mere/dansk-sejlunion/strategi-og-politik/vision-og-vaerdier]


%\section{Andre fritidsklubber}
%
%I denne sammenhæng er ``Andre fritidsklubber'' fritidsklubber som har samme lignende funktion som en sejlklub.
%Der kan drages paralleller til en skytteforening; en skytteforening har skydebaner som kan administreres og
%hvis man ikke selv medbringer våben, så kan man ofte også låne et, hvor der ligeledes kan ske adminstration. I
%skydning er konkurrencerne ofte delt op efter typen af skydning, alder og skydestil \fxnote{andet ord?}, hvilket
%medfører at medlemmerne med fordel kunne skrives ind i et system som automatisk kan holde styr på dem og se
%hvilke konkurrencer de kvalificerer sig til.


\section{Medlemmer}

Interessentgruppen ``medlemmer'' omhandler medlemmer af sejlklubber. Det er meget forskelligt fra sejlklub til
sejlklub hvordan medlemmer kategoriseres eller om de overhovedet gøres dette. F.eks. så har ``Sejlklubben
Sundet'' kategoriseret deres medlemmer således: Voksen-, Bådejer-, Gaste-, Mini-kølbåd-, Ungdoms-, Passiv- og
Støttemedlem.
%[http://www.sundet.dk/vedtaegter/Vedtaegter%20for%20Sundet%2027nov12.pdf]

Til sammenligning har Vestre Baadelaug kategoriseret deres medlemmer efter følgende: Aktive-, passive- og
æresmedlemmer.
%[http://www.aalborglystbaadehavn.dk/UserFiles/file/VB_filer/Statiske_filer/Vedtaegter_2008-9_web.pdf].

%\section{Gæster}:
%Gæster har ikke nogen primær funktion når det kommer til udlån, de bådklubber der tilbyder udlejning af både, forbeholder denne funktion til medlemmer, som har bestået et førerkursus accepteret af den givne bådklub.

%Forældre(irrelevant)?:


\section{Undervisere}

Undervisere: Undervisererne udnytter sejlerklubbens både i faste tidsrum til undervisning. Deres funktion i
klubben gør at der er specifikke tidspunkter hvorpå bådene ikke kan udlejes. Underviserne kunne benyttes til
det formål, at gøre udlejning af både lettere. Undervisere ville løbende kunne opdatere status for hvilke
medlemmer der er kvalificeret til at føre en båd, og derved hjælpe med det administrative aspekt af udlejning
af både.

\cbstart
\section{Research af klubber}\label{sec:research}

I følgende afsnit beskrives nogle sejlklubbers opbygning, hvordan en af sejlklubberne håndtere data, samt arbejdet der
ligger bag. Udfra dette kan der konkluderes på en generel opbygning af klubberne, og slutteligt hvilke funktioner de
forskellige medlemmer har i klubben.

Sejlklubben Sundet, som ligger i København Ø 2100 på Svaneknoppen 8, har flere både, af forskellig størrelse, som bruges
til undervisning af klubbens elever, såvel som udlån til klubbens medlemmer. Undervisningen foregår i sejlerskolen, med
undervisning om hverdagene. Uddannelsen varer 2 år, og man lærer at sejle både i Drabant 24, og
Gaffelrigger.\fxnote{måske der skal en fodnote ind om hvad de her to ting er?} Man kan i klubben som nævnt låne bådene
til sejlladser for et mindre beløb, men hvis en af skolens elever er med, er dette gratis at gøre. Det er dyrere at leje
i weekenderne end om hverdagene. I forbindelse med en sæson på sejlerskolen, er man ude og sejle op til 18 gange typisk
fra 18:00-21:00 en gang om ugen. Der sejles desuden kun i maj, juni, august og september. Når skolen er færdig og man
har bestået førerprøven er man iflg. Sundets reglementer Bådfører, og man får tildelt sit eget certifikat.

For at en besætning må sejle skal der være minimum én fører med i bådens besætning, altså en person med at førerbevis
opnået igennem en sejleruddannelse. Desuden bestemmes prisen for udlånet bl.a. efter besætning også, idet
tilstedeværelsen af en af klubbens elever gør, at udlånet er gratis.\citep{Sundet}

Ud fra dokumenterne modtaget af Jacob Nørbjerg\fxnote{How to write this??} ses der at der bruges et en lokal computer til at
holde informationer på klubbens medlemmer, samt til at opkræve kontingenter af klubbens medlemmer. Dette kan resultere i
at det ikke bliver gjort ofte, og/eller er til gene for hjælperne da de skal ned i klubben, for at kunne håndtere disse
emner.

Ved et kig på Sejlklubben Sundets hjemmeside \citep{SundetUdlaan} ses det, at forskellige værktøjer bl.a. en doodle er
taget i brug, i et forsøg på at øge brugervenlighed og interaktion mellem deltagere. Som Jacob giver udtryk for i
interviewet er det svært at få viden omkring klubbens foretagender hjemmefra. Dette har effekten at man man skal altså
ned i klubben for at melde sig på diverse events, såsom 24-timer sejladser, eller for at se om der er en ledig plads på
et sejllads hold. Mange informationer for klubbens medlemmer findes på klubbens opslagstavle. Det kan være svært lige at
overskue, og det er slet ikke sikkert man får set ordenligt på tavlen, og på denne måde kan medlemmerne helt undgå at
opdage der er et 24-timers sejllads, hvilket naturligvis er ærgerligt.

Hver gang der skal sejles ved sejlklubben Sundet, er der meget data der skal skrives ned, af forskellige årsager. Det er
som tidligere nævnt vigtigt at vide hvem der er med på båden, både for at sikre sig, besætningen er i stand til at føre
båden, men også af sikkerhedsmæssige årsager, hvis der skulle ske en ulykke. Herudover er sejlerklubben nødt til at vide
hvor mange sejladser eleverne har været med på, samt hvilke øvelser de har udført, for at kunne lade dem bestå
uddannelsen. En liste over disse data er udformet og kan findes i \myref{bilag:sundet}.

Sejlklubben bestemmer sin ledelse ved generalforsamlinger, her bliver der også valgt, regnskabsfører, øvrige
bestyrelsesmedlemmer, samt 1 revisor og revisorsuppleant. Kun æresmedlemmer eller aktive medlemmer over 18 år har
stemmeret ved generalforsamlingen. Stemmeretten udøves ved personligt fremmøde til forsamlingen. Medlemmerne kan senest
seks uger før generalforsamlinger sende forslag til diskussioner, såsom vedtægtsændringer, eksklusioner af medlemmer
ol. Det er altså demokratiske valg der håndterer større beslutninger i klubben.

En anden sejlklub, Aalborg Sejlklub som findes på Skydebanevej 40, Aalborg 9000, ved Marina Fjordparken. Der findes
mindre information på deres hjemmeside, end ved Sundet, men Aalborg Sejlklub har også en sejlerskole, som indeholder en
praktisk prøve, \fxnote{Jeg har sendt en mail for at høre hvad formålet med skolen er, for man sit førerbevis ? - Det
står ikke angivet på deres hjemmeside.} Aalborg Sejlklub administreres på samme lign. måde som Sundet gør. De har en
Ordinær Generalforsamling, hvor der stemmes for hvem der skal have de bestyrende roller. I Aalborg Sejlklub findes også
en næstformand. Ved Generalforsamlingen har alle medlemmer over 18 år stemmeret, og den udøves på samme måde som ved
Sundet, i form af personligt fremmøde.\citep{AalborgSejlklub}

En tredje klub, Bådklubben Valby, har stortset samme fordeling af medlemmer, og deres funktioner. De holder
Generalforsamling, hvor bestyrrende medlemmer vælges. Bådklubben har ikke en skole på samme måde, men de har dog stadig
kurser således man kan opnå sit duelighedsbevis hos dem.\citep{BaadklubbenValby}


\subsection{Konklusion af Organisationsanalyse}

De 3 klubber der er undersøgt har mange ting til fælles i deres organisation.

De har en bestyrrelsesformand, nogle gange en næstformand. Formanden er valgt fra generalforsamlingen, og har til job at
stå for dagsordnerne ved generalforsamlingerne, og fungerer som en leder for klubben. Herudover er der sekretærer, samt
nogle der står for klubbernes økonomi. Udover disse stillinger, er der de frvillige hjælpere, som hælper med diverse
opgaver i klubben, nogle af dem er sejlinstruktører, andre hjælper med at holde styr på bådene og holde dem i stand. Der
er også forskellige typer medlemmer i klubberne. Nogle af dem kan være elever i sejlerskolen, hvis klubben har sådan en,
andre er det der kaldes for A-medlemmer. Fælles for medlemmerne er at de betaler kontigenter for at være medlem, og for
at kunne benytte sig af klubbens faciliteter. Det er herudover meget forskelligt fra klub til klub, hvilke frynsegoder
der kan være ved at hjælpe til i klubben.

Efter denne undersøgelse er der altså dannet en forståelse af hvordan bådkluberne er opbygget, og hvordan de håndtere
forskellige opgaver i form af nedskrivninger i bøger og på opslagstavler.

%\fxnote{Dette er blot et foreløbt udkast til funktioner, det kan være der skal lidt mere information til for at kunne
% drage disse konklusioner. }
%
%\begin{itemize}
%	\item Medlemmer skal kunne leje både, og angive hvem af klubbens medlemmer besætningen står af.
%	\item Medlemmer bør kunne holde øje med lektioner og være i stand til at melde sig på lektionerne.
%	\item Der bør være en liste over medlemmer, så medlemmerne indbyrdes kan tage kontakt til hinanden.
%	\item Der bør være et kontaktmedium så medlemerne kan kontakte hinanden, eller spørge på et forum om diverse emner.
%	\item Det bør være muligt for formanden let at dele meddelelser med klubbens medlemmer.
%	\item Medlemmerne bør være i stand til at betale for deres kontigenter over nettet, for at nedsætte det manuelle
% arbejde fra klubbens frivillige.
%\end{itemize}
\cbend
