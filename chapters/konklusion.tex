\chapter{Konklusion}

\cbstart

I det initierende problem blev der fundet frem til, at der skulle laves en softwareløsning, som kan hjælpe frivillige i fritidsklubber til administration og som er let at benytte. 
Der blev afgrænset i afsnittet Fritidsklubber  til at arbejde med sejlklubber, fordi det blev vurderet at det var lettere at udarbejde et program, ud fra en bestemt type fritidsklub. Der blev yderligere afgrænset, i interessentanalysen, til at arbejde med Sejlklubben Sundet, for at gøre processen med at udarbejde et program endnu lettere.
I organisationsafsnittet blev Sejlklubben Sundet undersøgt bl.a. for at få afklaret deres administrative opgaver. 
I teknologianalysen blev mulige teknologier til en softwareløsning undersøgt og under ``State of the Art'' blev andre typer af bådsystemer undersøgt. 
Ud fra analysen, blev der skabt en problemformulering, som fokuserer på frivilliges arbejde i fritidsklubber og hvordan de kan hjælpes i deres arbejde.
Efter udarbejdelsen af problemformuleringen, blev kravene til produktet undersøgt, som skulle bruges til det videre forløb, med at udarbejde et program.


\cbend
