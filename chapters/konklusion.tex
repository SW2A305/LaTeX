\chapter{Konklusion} \fxnote{Troels: Er konklusion en del af problemløsningen, eller er det (og diskusion/perspektivering en del for sig?) Marc: vi kunne lave det til sin egen del under ''Reflektion'' da alle tre kapitler essentielt er det og ikke rigtig en løsning. Tristan: Hvis vi gør dét, skal vi huske at ændre de steder, der omtaler rapporten som bestående af to dele.}
I det initierende problem blev der undersøgt, hvorvidt der skulle laves en softwareløsning, som kan hjælpe frivillige i fritidsklubber til administration, og som er let at benytte. 
Der blev i afsnittet Fritidsklubber afgrænset til at arbejde med sejlklubber, fordi det blev vurderet, at hvis det var muligt at udarbejde et program til en enkelt type fritidsklub, kunne programmet også udvikles til andre fritidsklubber. \fxnote{Marc: I diskussionen nævnte vi at programmet ikke kunne overføres, mon ikke det skulle gøres klart i konklusionen at et program kan udvikles til andre klubber, vores er bare ikke kompatibelt?}
Der blev i interessentanalysen yderligere afgrænset til at arbejde med Sejlklubben Sundet, for at gøre udviklingsprocessen for programmet endnu lettere. \fxnote{Tristan: Lyder det ikke lidt træls? At vi har afgrænset for at gøre det så lidt udfordrende som overhovedet muligt?}
I organisationsafsnittet blev der fundet frem til hvilke funktioner, Sejlklubben Sundet søgte i et program.
De centrale værende reservering af både, sejlerskole organisering og håndtering af logbøger.\fxnote{Tristan: ``De centrale værende'' lyder lidt isoleret fra resten. Kan vi eventuelt omformulere det en smule?}
I teknologianalysen blev der fundet frem til, at der eksisterer softwareløsninger, som kan dække nogle af Sejlklubben Sundets behov, dog blev disse ikke benyttet, muligvis fordi ingen af dem havde undervisningsfunktionaliteter. 
Dette resulterede i en problemformulering, som fokuserede på at udarbejde en softwareløsning, som kunne dække Sejlklubben Sundets behov. 
Efter udviklingen af programmet blev der opstillet en række tests for at teste programmets brugervenlighed, samt for at se om programmet opfyldte de stillede krav.

På trods af, at løsningen ikke er optimal, konkluderes det hermed, at den er acceptabel. 
Den endelige vurdering af programmets brugbarhed kan dog kun gives af Sejlklubben Sundet.
Dette ændrer ikke på, at løsningens funktionaliteter opfylder de opstillede krav fra analysen. 

\fxnote{Skal der skrives noget om programstruktueren her ? (Søren)}
\fxnote{Skal der sættes myrefs ind over det hele ???(Søren)}
