\chapter{Konklusion}
I det initierende problem blev der fundet frem til, at der skulle laves en softwareløsning, som kan hjælpe frivillige i fritidsklubber til administration og som er let at benytte. 
Der blev afgrænset i afsnittet Fritidsklubber  til at arbejde med sejlklubber, fordi det blev vurderet at det hvis det var muligt at udarbejde et program til enkelt type fritidsklub, kunne programmet også udvikles til andre. 
Der blev yderligere afgrænset, i interessentanalysen, til at arbejde med Sejlklubben Sundet, for at gøre processen med at udarbejde et program endnu lettere.
I organisationsafsnittet blev der fundet frem til opbygningen og funktionerne for Sejlklubben Sundet.
I teknologianalysen blev der fundet frem til at der eksisterer softwareløsninger, dog blev disse benyttet af Sejlklubben Sundet eftersom ingen af dem havde undervisningsfunktioner. 
Dette resulterede i en problemformulering som fokuserede på at udarbejde en softwareløsning til Sejlklubben Sundet. 
Efter udviklingen af nævnte løsning blev der opstillet en række tests for at se om programmet stemte overens med krav opstillet i analysen.

På trods af at løsningen ikke er den optimale løsning er der konkluderet på at det er en acceptabel løsning.
 
Dog kan en egentlig konklusion om programmets brugbarhed kun gives af medlemmer af Sejlklubben Sundet. 

Dette ændrer ikke på at løsningen, med hensyn til funktionalitet, opfyldte de krav som blev opstillet i analysen. 

