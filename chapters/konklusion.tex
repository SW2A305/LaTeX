\chapter{Konklusion}
I det initierende problem blev der undersøgt, hvorvidt der skulle laves en softwareløsning, som kan hjælpe frivillige i fritidsklubber til administration og som er let at benytte. 
Der blev i afsnittet Fritidsklubber afgrænset til at arbejde med sejlklubber, fordi det blev vurderet at hvis det var muligt at udarbejde et program til en enkel type fritidsklub, kunne programmet også udvikles til andre fritidsklubber. 
Der blev i interessentanalysen yderligere afgrænset, til at arbejde med Sejlklubben Sundet, for at gøre udviklingsprocessen for programmet endnu lettere.
I organisationsafsnittet blev der fundet frem til hvilke funktioner, Sejlklubben Sundet eftersøgte i et program.
De centrale værende reservering af både, sejlerskole organisering og håndtering af logbøger.
I teknologianalysen blev der fundet frem til at der eksisterer softwareløsninger som kan dække nogle af Sejlklubben Sundets behov, dog blev disse ikke benyttet, eftersom ingen af dem havde undervisningsfunktionaliteter. 
Dette resulterede i en problemformulering som fokuserede på at udarbejde en softwareløsning som kunne dække Sejlklubben Sundets behov. 
Efter udviklingen af programmet blev der opstillet en række tests for at teste programmets brugervenlighed, samt for at se om programmet opfyldte de stillede krav.

På trods af at løsningen ikke er optimal, konkluderes det hermed at den er acceptabel. 
Dog kan den endelige vurdering af programmets brugbarhed kun gives af Sejlklubben Sundet.
Dette ændrer ikke på at løsningen, med hensyn til funktionalitet, opfyldte de krav som blev opstillet i analysen. 

