\chapter{Programmets brugergrænseflade}
I dette kapitel vil programmet grafiske brugergrænseflade blive beskrevet.

\section{Login brugergrænseflade}
\begin{wrapfigure}{r}{0.5\textwidth}
    \label{img:login_interface}
    \vspace{-20pt}
    \begin{center}
        \includegraphics[width=0.48\textwidth]{UI/Login_Window_Empty_Fields.png}
    \end{center}
    \vspace{-15pt}
    \caption{Login interface}
\end{wrapfigure}

Login brugergrænsefladen (Figur \ref{img:login_interface}) er opstartsvinduet, og giver brugerne mulighed for at logge ind.
Formålet med dette er, at programmet skal kende identiteten af den person som anvender programmet.
Login knappen (samt at trykke på ``Enter''/``Return''-knappen mens tekstfelterne er i fokus), udfører et login. 
Forudsagt at brugeroplysninger angivet er korrekte, eller vises en passende fejlbesked.
Man kan også logge ind som en gæste bruge, som har reducerede tilladelser. 
Den kan bruges, hvis man vil tjekke noget infomation hurtigt, og ikke vil til at logge ind, da det tager tid.
Dette udføres ved at trykke på ``Login som gæst''-knappen. 

Efter fuldført login åbens et passene hovedvindue, med funktionaliteter i henhold til tabel \ref{tab:permissions}.

% LaTeX tabel som viser alle brugerniveauer og deres muligheder efter login.
% http://bit.ly/1kToHZ3 to edit raw table
\begin{table}
    \colorlet{shadecolor}{gray!40}
    \rowcolors{1}{white}{shadecolor}
    \begin{tabular}{l|llllll}
    ~                        & Gæst & Støttemedlem & Medlem & Elev & Underviser & Administrator \\ \hline
    Personlig forside        & ~    & ~             & \ding{51}     & \ding{51}    & \ding{51}          & \ding{51}             \\
    Se begivenheder          & \ding{51}    & \ding{51}             & \ding{51}      & \ding{51}    & \ding{51}          & \ding{51}             \\
    Tilmeld begivenheder     & ~    & ~	             & \ding{51}      & \ding{51}    & \ding{51}          & \ding{51}             \\
    Opret begivenheder       & ~    & ~             & ~      & ~    & \ding{51}          & \ding{51}             \\
    Se sejlture              & \ding{51}    & \ding{51}             & \ding{51}      & \ding{51}    & \ding{51}          & \ding{51}             \\
    Opret sejltur            & ~    & ~             & \ding{51}      & \ding{51}    & \ding{51}          & \ding{51}             \\
    Se logbøger              & \ding{51}    & \ding{51}             & \ding{51}      & \ding{51}    & \ding{51}          & \ding{51}             \\
    Opret logbog             & ~    & ~             & \ding{51}      & \ding{51}    & \ding{51}          & \ding{51}             \\
    Svar på logbog           & ~    & ~             & ~      & ~    & ~          & \ding{51}             \\
    Se undervisningstimer    & ~    & ~             & ~      & \ding{51}    & \ding{51}          & \ding{51}             \\
    Opret undervisningstimer & ~    & ~             & ~      & ~    & \ding{51}          & \ding{51}             \\
    \end{tabular}
    \caption{Tabel over alle brugerniveauer og deres tilladte funktioner.}\label{tab:permissions}\fixme{Måske skal denne tabel lige laves lidt om, enten indholdet, eller hvordan det virker i programmet.}
\end{table}

\section{Primære brugergrænseflade}
Det primære vindue tilgåes via login vinduet, som er det der starter ved programstart, eller logud. 
Der findes 4 vinduer, forskellen mellem dem er hvilke tabs der er aktive. 
I hver tab findes der en funktionalitet, samlet set findes følgende tabs:
\begin{itemize}% Denne skulle måske relatere til tabel tab:permission
    \item Forside
    \item Undervisning
    \item Begivenheder
    \item Medlemmer
    \item Både
\end{itemize}

Via hver af disse tabs, vil der være adgang til programmet forskellige funktionaliteter.
Der er også en log ud knap, som bruges til at vende tilbage til loginvinduet, således en anden bruger kan anvende systemet.
Programmet er lavet til at køre i opløsningen 1024x720 pixels.
Denne opløsning er valgt for at understøtte alt fra bærbare, med opløsninger som 1366x768 pixels i et vindue, og op til FullHD (1920x1080 pixels) osv.
Programmet har et lyst farveskema, men farverne hvid og en lys blå som hovedfarver (\#87D4EE).

\section{UserControls}
Der anvendes UserControls til at kode både brugergrænsefladen og den tilhørende code behind.

\subsection{Forside}
\begin{figure}
    \label{img:frontpage}
    \vspace{-10pt}
    \begin{center}
        \includegraphics[scale=0.55]{UI/UserControl_FrontPage}
    \end{center}
    \vspace{-15pt}
    \caption{FrontPage-Usercontrol}
\end{figure}

\textbf{Formål}: 
Formålet med forsiden er at vise aktuel infomation på en overskulig måde for brugeren.
Det er den første side som medlemmer, og op, ser.
Fra forsiden kan man ændre og slette sine bookings, samt starte oprettelsen af en logbog.

\textbf{BrugerGrænseflade}: 
Brugergrænsefladen er forsiden består primært af to DataGrids, det venste viser de kommende sejlture og det højre viser dem personen manlger at udføre logbøger for. 
Under dem er der knapper, som bliver aktive efter en markering er udført ved, at trykke på en af rækkerne i det passende DataGrid.

\textbf{Code-Behind}: 
For at kun vise de sejlture hvor personen som er logget ind deltager i, anvendes der standard query operatorerne. 
I Listing \ref{fntpg-cb} er et udsnit af koden, nærmere bestemt den del som vælger de rigtige sejlture.
Det første udtryk finder de ture som er i fremtiden, hvor personen indgår i besætningslisten, ud fra alle turene.

Det andet udtryk finder de ture, hvor den nuværende person er kaptajn for, som er forventet at være returneret til havn og ikke har nogen logbog. 

Begge udtryk returnerer en IEnumerable som derefter assignes som DataGridenes ItemSource.

\begin{lstlisting}[frame=single, caption=Forsidens Code-Behind, label=fntpg-cb]
UpcommingTripsDataGrid.ItemsSource =
    sailTripList.Where(t => t.Crew.Select(p => p.PersonId).Contains(usrId))
        .Where(t => t.DepartureTime > DateTime.Now);

LogbookDataGrid.ItemsSource =
    sailTripList.Where(t => t.Captain.PersonId == usrId && t.ArrivalTime < DateTime.Now && t.Logbook == null);
\end{lstlisting}

\section{Vinduer}
\subsection{CreateBoatBookingWindow}

\begin{wrapfigure}{R}{0.5\textwidth}
    \label{img:boatBookWindow}
    \vspace{-20pt}
    \begin{center}
        \includegraphics[width=0.48\textwidth]{UI/CreateBoatbookingWindow.png}
    \end{center}
    \vspace{-20pt}
    \caption{CreateBoatbookingWindow}
    \vspace{-30pt}
\end{wrapfigure}

\textbf{Formål}: 
Dette vindue opretter, eller ændrer, en \textbf{RegularTrip} (En bådbooking).
Dette vindue anvendes fire stedder i programmet, fra \textbf{Boats}-UserControllen, \textbf{NewLecture}-vinduet (under Undervisning), og fra \textbf{FrontPage}-UserControllen eller \textbf{Boats}-UserControllen (Til at redigere en booking). 

\textbf{BrugerGrænseflade}: 
Vinduet er opbygget af et vertikalt StackPanel.
Først vælges en båd, her anvendes en ComboBox da der skal vælges en værdi fra en liste.
Herefter anvendes \textbf{DateTimePicker}-UserControllen til at vælge et start- og sluttidspunkt.
Der vises den nuværende besætning, samt muligheden for at ændre den ved at trykke på en Button med labellet ``Ændre Besætning''.
Når en besætning er valgt, kan der vælges en kaptajn ud fra besætningslisten.
Tilsidst er der en TextBox, hvori brugeren kan angive formålet med turen.
Derudover er der to Buttons, en til at gemme og en til at annulere.

\textbf{Code-Behind}: 
Til vinduet er der tre constructors, dette er således de fire stedder hvorfra vinduet åbnes, hver kan behande det på deres måde, da to af dem er ens. 
De tre constuctors er i Listing \ref{threeConstructors}.
Den første constuctor skaber et nyt vindue, herunder henter den bådede fra DataBasen og bliver kaldt fra Boats-UserControllen.
Den vælger også den båd, som er angivet i dens index parameter.
De to DateTimerPickere sættes også til det nuværende tidspunkt, dette gør det nemmere at vælge et passende tidspunkt for den kommende booking.

De to andre constructors kalder, den første da den er grundlæggende for at kunne bruge vinduet. 
Constructoren med forskiften ``CreateBoatBookingWindow(RegularTrip rt) : this(-1)'', ændrer teksten på gem knappen fra ``Gem Booking'' til ``Ændre Booking''. \fixme{Skal der være nutids-r på ændre(r)?}
Samt den ændre hvilken methode den kalder, således der ikke forsøges at oprette en ny sejltur, men derimod opdateres en ekstisterende. 
Fælles for de to metoder som kaldes af knappen er at de skal verificere gyldigheden af en tur. 
Dette udføres i ``CreateSailTrip()'' som returner en gyldig instans af SailTrip-klassen hvis turen er gyldig, eller null og en fejlbesked hvis den ikke er. 
Dette null bliver håndtereret af kaldermetoden, således der ikke opstår exceptions.

\begin{lstlisting}[frame=single, caption=De tre constuctoreres forskrifter, label=threeConstructors]
// Called to Initialize the window, from the other constructors and from the Boats-UserControl
public CreateBoatBookingWindow(int index)
{
    InitializeComponent();

    // Initialize ComboBoxes and Databaseforbindelsen
    [...]

    // Set DateTimerPickers to the current time.
    [...]
}

// Called to edit a trip from the FrontPage
public CreateBoatBookingWindow(RegularTrip rt) : this(-1)
{
    // Sets all the value from the input RegularTrip
    [...]

    // Change the text and behaveour of the buttons
    [...]
}

// Called from NewLecture
public CreateBoatBookingWindow(DateTime departure, DateTime arrival, Team currentTeam) : this(-1)
{
    // Sets values to match the parameters given
    [...]

    // Set the Captain to be the tracher
    [...]

    // Add a description of the class in the PurposeTextBox.
    [...]

    // Call the SaveFunction
    [...]
}
\end{lstlisting}

\subsection{CreateCrewWindow}

\begin{wrapfigure}{r}{0.5\textwidth}
    \label{img:login_interface}
    \vspace{-20pt}
    \begin{center}
        \includegraphics[width=0.48\textwidth]{Screenshots/CreateCrewWindow.png}
    \end{center}
    \vspace{-20pt}
    \caption{CreateCrewWindow}
    \vspace{-30pt}
\end{wrapfigure}

\textbf{Formål}: Dette vindue åbnes op to steder i programmet: \textbf{CreateLogbookWindow} og i \textbf{CreateBoatBookingWindow}. Det bruges når der skal laves en besætning til en RegularSailtrip.  

\textbf{BrugerGrænseflade}: Der er to datagrids som indeholder to lister. Listen til venstre består af SailClubMembers som bliver hentet ind fra databasen. Listen til højre indeholder det Crew som man er i gang med at udforme til enten RegularSailTrip, eller Logbook. Der er også tilføjet tekstfelter, så man kan skrive navnet på en gæst man tog med på sejlturen. Der er knapper som tilføjer de forskellige personer til Crewlisten. Øverst findes også et tekstfelt til at søge listen over medlemmer igennem. Når man har lavet sin liste, kan man trykke udfør, for at komme tilbage til vinduet der kaldte CreateCrewWindow.

\textbf{Code-Behind}: 
På \myref{AddGuestButton} kan man se koden der sker når man trykker på knappen med teksten Tilføj Gæst.
Der er blevet brugt et \textbf{regular expression} til at tjekke om den string brugeren angiver i de to tekstbokse for gæstens navne er lovlige. 
Det er blevet valgt man må bruge hele det danske alfabet samt mellemrum, så navne såsom: Lars Peter Vesterager, er mulige.
Hvis begge tekstbokse er lovlige, kommer man ind i det inderste if-statement, hvor der laves en ny person med det pågældende FirstName og LastName. 
Derefter tilføjes personen til Listen der vises i Datagridded til højre, og til sidst kaldes RefreshDatagrid, som man kan se på \myref{RefreshDatagrid}.

\begin{lstlisting}[frame=single, caption=Add Guest Buttton, label=AddGuestButton]
{
    if (Regex.IsMatch(FirstNameBox.Text, "^[A-ZÆØÅa-zæøå ]*$") && FirstNameBox.Text.Trim() != String.Empty)
    {
        if (Regex.IsMatch(LastNameBox.Text, "^[A-ZÆØÅa-zæøå ]*$") && LastNameBox.Text.Trim() != String.Empty)
        {
            var p = new Person();
            p.FirstName = FirstNameBox.Text;
            p.LastName = LastNameBox.Text;
            CrewList.Add(p);

            RefreshDatagrid(CurrentCrewDataGrid, CrewList);

            FirstNameBox.Clear();
            LastNameBox.Clear();
        }
    }
    else
    {
        MessageBox.Show("Ugyldigt navn. \nPrøv venligst igen");
    }
}      
\end{lstlisting}

Her modtages der et Datagrid, som skal have dets Itemssource Refreshet, og en ICollection, som er det data der skal sættes ind i Datagridded. 
Det gøres ved at assigne dets Itemssource til null, og derefter assigne det tilbage til den ICollection der blev sendt med. 

\begin{lstlisting}[frame=single, caption=Refresh Datagrid, label=RefreshDatagrid]
private void RefreshDatagrid(DataGrid Grid, ICollection<Person> list)
{
    Grid.ItemsSource = null;
    Grid.ItemsSource = list;
}
\end{lstlisting}

Knappen til at tilføje et medlem kan ses på \myref{AddMember}.
Inden det valgte medlem tilføjes til listen tjekkes der om medlemmet allerede findes i den anden liste. 
Dette gøres ved at anvende standard query operatorerne. 
Først et where med et lambda udtryk hvor der findes alle SailClubMembers i listen. 
Herefter castes disse om til SailClubMembers, for derefter at kunne tjekke om det valgte medlems SailClubMemberId er forskelligt fra de andre i listen. 
Hvis det hele er true, så bliver det valgte medlem tilføjet til listen, og RefreshDataGrid kaldes for at opdatere Datagridded.

\begin{lstlisting}[frame=single, caption=Add Member, label=AddMember]
private void AddButton_OnClick(object sender, RoutedEventArgs e)
{
    var currentPerson = (SailClubMember) MemberDataGrid.SelectedItem;

    if (
        CrewList.Where(x => x is SailClubMember)
            .Cast<SailClubMember>()
            .All(x => x.SailClubMemberId != currentPerson.SailClubMemberId))
        CrewList.Add(currentPerson);

    DataGridCollection.Filter = Filter; 
    RefreshDatagrid(CurrentCrewDataGrid, CrewList);
}
\end{lstlisting}

\subsection{CreateLogbookWindow}

\begin{wrapfigure}{r}{0.5\textwidth}
    \label{img:login_interface}
    \vspace{-20pt}
    \begin{center}
        \includegraphics[width=0.48\textwidth]{Screenshots/CreateLogbook.png}
    \end{center}
    \vspace{-15pt}
    \caption{CreateLogBookWindow}
    \vspace{-30pt}
\end{wrapfigure}

\textbf{Formål}: Dette vindue bruges når et medlem som har booket en båd, skal udfylde sejlturens logbog. Vinduet tilgås fra usercontrollen FrontPage. Medlemmet skal udfylde de forskellige felter der findes i vinduet og trykke udfør, for at gemme logbogen i databasen.

\textbf{BrugerGrænseflade}: 

\textbf{Code-Behind}:

\myref{HelloWorld}


