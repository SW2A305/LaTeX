\chapter{Programmets brugergrænseflade}


I dette kapitel vil programmet grafiske brugergrænseflade blive beskrevet.


\section{Navigering i programmet}

\begin{figure}[H]
\hspace*{-2cm}
\includegraphics{UI/UI2.pdf}
\label{img:programNavigation}
\vspace{-310pt}
\caption{Navigations skema for programmet}
\vspace{-20pt}
\end{figure}

På \myref{img:programNavigation} ses der hvordan navigeringen i programmet foregår.
Øverst ses login-vinduet, herfra logger brugeren ind med sine login informationer, eller en gæst kan logge ind ved at trykke på \textit{login som gæst}-knappen.
Brugeren ser nu hovedvinduet, hvilket er vinduet i midten med fem tabs i toppen.
Antallet af tabs varierer alt efter hvilken position brugeren har i systemet, hvilket kan ses på \myref{tab:permissions}. 
Fra forside tabben kan der navigeres til de andre tabs, som brugeren har tilgængelig.
Pilene illustrerer de dialogbokse som brugeren kan åbne fra de forskellige tabs.
Dialogboksene åbnes ved at trykke på knapper inde i tabsne, f.eks. under  \textbf{Bådreservation} kan opret booking dialogen åbnes ved at trykke på knappen \textit{Reserver Båden}. 
På samme måde åbnes de resterende dialoger ved tryk på knapper.
Flere steder i programmet anvendes de samme dialogvinduer, f.eks. fra forsiden findes en mulighed for at ændre på en reservation, hvilket åbner en mindre modificeret udgave af booking dialogen, hvorfra besætningsvælgerdialogen også kan åbnes.
Desuden findes logbogsdialogen også på forsiden.
Forskellen på logbogsdialogen på forsiden og den under bådreservation, er at dialogen under forsiden er lavet for at logbogen kan udfyldes, mens den under bådreservation er read-only.
\fxnote{Aner ikke om knapperne har de rigtige navne}
\fxnote{Billedet skal opdateres, mangler opret event,}


% LaTeX tabel som viser alle brugerniveauer og deres muligheder efter login.
% http://bit.ly/1kToHZ3 to edit raw table
\begin{table}
    \colorlet{shadecolor}{gray!40}
    \rowcolors{1}{white}{shadecolor}
    \begin{tabular}{l|llllll}
    ~                        & Gæst & Støttemedlem & Medlem & Elev & Underviser & Administrator \\ \hline
    Personlig forside        & ~    & ~             & \ding{51}     & \ding{51}    & \ding{51}          & \ding{51}             \\
    Se begivenheder          & \ding{51}    & \ding{51}             & \ding{51}      & \ding{51}    & \ding{51}          & \ding{51}             \\
    Tilmeld begivenheder     & ~    & ~	             & \ding{51}      & \ding{51}    & \ding{51}          & \ding{51}             \\
    Opret begivenheder       & ~    & ~             & ~      & ~    & \ding{51}          & \ding{51}             \\
    Se sejlture              & \ding{51}    & \ding{51}             & \ding{51}      & \ding{51}    & \ding{51}          & \ding{51}             \\
    Opret sejltur            & ~    & ~             & \ding{51}      & \ding{51}    & \ding{51}          & \ding{51}             \\
    Se logbøger              & \ding{51}    & \ding{51}             & \ding{51}      & \ding{51}    & \ding{51}          & \ding{51}             \\
    Opret logbog             & ~    & ~             & \ding{51}      & \ding{51}    & \ding{51}          & \ding{51}             \\
    Svar på logbog           & ~    & ~             & ~      & ~    & ~          & \ding{51}             \\
    Se undervisningstimer    & ~    & ~             & ~      & \ding{51}    & \ding{51}          & \ding{51}             \\
    Opret undervisningstimer & ~    & ~             & ~      & ~    & \ding{51}          & \ding{51}             \\
    \end{tabular}
    \caption{Tabel over alle brugerniveauer og deres tilladte funktioner.}\label{tab:permissions}\fixme{Måske skal denne tabel lige laves lidt om, enten indholdet, eller hvordan det virker i programmet.}
\end{table}

\section{Hovedvinduet}
Hovedvinduet tilgås via loginvinduet, som starter ved programstart eller ved at trykke på logudknappen inde fra hovedvinduet selv. 
Der findes fire udgaver af hovedvinduet, forskellen mellem dem er, hvilke tabs der er aktive.
Hver tab indeholder forskellige funktioner, samlet set findes følgende tabs:
\begin{itemize}% Denne skulle måske relatere til tabel tab:permission
    \item Forside
    \item Undervisning
    \item Begivenheder
    \item Medlemmer
    \item Både
\end{itemize}

Via hver af disse tabs, vil der være adgang til programmets forskellige funktionaliteter.
Der er også en logud knap, som bruges til at vende tilbage til loginvinduet, således en anden bruger kan anvende systemet.
Programmet er lavet til at køre i opløsningen 1024x720 pixels.
Denne opløsning er valgt for at understøtte alt fra bærbare, med opløsninger som 1366x768 pixels i et vindue, og op til FullHD (1920x1080 pixels) op op efter. 
Stort set alle computerskærme har en opløsning større end 1024x720 pixels\citep{resolutions}. 
Programmet har et lyst farveskema; med farverne hvid og en lys blå som hovedfarver (\#87D4EE).

\section{UserControls}
Der anvendes Usercontrols til at kode både brugergrænsefladen og den tilhørende Code-Behind.

\subsection{DateTimePicker}\label{subsec:DateTimePicker}

\begin{wrapfigure}{r}{0.5\textwidth}
    \label{img:DateTimePicker}
    \vspace{-20pt}
    \begin{center}
        \includegraphics[width=0.48\textwidth]{Screenshots/DateTimePicker.png}
    \end{center}
    \vspace{-15pt}
    \caption{DateTimePicker}
    \vspace{-30pt}
\end{wrapfigure}

\textbf{Formål}: 
Denne Usercontrol er lavet, da der ikke fandtes en tilfredsstillende løsning, som gjorde det muligt at vælge både dato og tidspunkt i samme control. 
Det er ofte nødvendigt at vælge både dato og tidspunkt samtidigt. 

\textbf{BrugerGrænseflade}: 
Den består af en DatePicker, og en TimePicker som findes i extended WPF toolkit.

\textbf{Code-Behind}: 
Der er blevet lavet en specialfremstillet getter og setter for Usercontrolen. 


\subsection{Forside}

\textbf{Formål}: 
Formålet med forsiden er at vise aktuel infomation på en overskuelig måde for brugeren.
Det er den første side i hovedvinduet, som man ser, med mindre man er gæst.
Fra forsiden kan man ændre og slette sine bookings, samt starte oprettelsen af en logbog.

\textbf{Brugergrænseflade}: 
Brugergrænsefladen på forsiden består primært af to DataGrids: Det venstre viser dine kommende sejlture, og det højre viser de sejlture personen mangler at udfylde logbøger for. 
Under dem er der knapper, som bliver aktive efter en markering er udført ved, at trykke på en af rækkerne i det tilhørende datagrid.

\textbf{Code-Behind}: 
For kun at vise de sejlture hvor personen, som er logget ind deltager i, anvendes der standard query operatorer. 
I listing \ref{fntpg-cb} er der et udsnit af koden, nærmere bestemt den del som vælger de korrekte sejlture.

Begge udtryk returnerer en IEnumerable, som derefter assignes som DataGridenes ItemSource.

\begin{lstlisting}[frame=single, caption=Forsidens Code-Behind, label=fntpg-cb]
//Sets the trips for the person currently logged in while only getting the ones in the furture to the DataGrid ItemsSource
UpcommingTripsDataGrid.ItemsSource =
    sailTripList.Where(t => t.Crew.Select(p => p.PersonId).Contains(usrId))
        .Where(t => t.DepartureTime > DateTime.Now);
//Sets trips created by the current user which happend in the past while also missing a logbook, to the other DataGrid ItemsSource 
LogbookDataGrid.ItemsSource =
    sailTripList.Where(t => t.Captain.PersonId == usrId && t.ArrivalTime < DateTime.Now && t.Logbook == null);
\end{lstlisting}

\subsection{Boat UserControl}

\textbf{Formål:}
I fanebladet Både kan man få et overblik over hvilke både som, der er til rådighed i sejlklubben inklusiv bådtype og status på båden. 
Man kan også booke en båd, se liste over logbøger for en valgt båd og ligeledes se en liste over kommende reservationer. 

\textbf{Brugergrænseflade:}
Efter valg af båd opdateres de resterende elementer i UserControlen. 
Her kan der ses et DataGrid med udfyldte logbøger samt et andet DataGrid med kommende reservationer på båden.
Det er muligt at vise og ændre felterne i begge DataGrids.
Som administrator kan man også svare på logbogens skadesrapport.
Foruden disse funktionaliteter kan man også herinde reserver den valgte båd samt har administratorer mulighed for at tilføje og redigere både.

\textbf{Code-Behind:}
Den bagvedliggende kode henter data fra databasen og opdaterer de grafiske elementer med dataet. 


\subsection{StudyTeacher UserControl}

\textbf{Formål:}
Denne UserControl bruges til give lærerer på sejlerskolen mulighed for at administrere skolehold og undervisningslektioner.

\textbf{Brugergrænseflade:}
Når der er valgt et hold, er der forskellige controls som opdateres.
Man kan se holdets lærer, elever og hvorvidt holdet er et 1. eller 2. års hold. 
Det er også muligt at se holdets lektioner, samt oprette og slette disse.
Når der trykkes på knappen for opretning af en ny lektion, åbnes et nyt vindue, hvor man indtaster start og sluttidspunkt.
Efter der er valgt en lektion kan der afkrydset hvad der er lært på lektionen, samt hvilke elever der har lært disse.
Der er også en knap som åbner et vindue, hvor man kan se hvilke læringsområder de enkelte elever har lært.
Når der laves et nyt hold, åbnes et vindue hvor man blot indtaster det ønskede navn for holdet.
Der kan slettes og forfremmes hold, hvilket betyder de får et bådførerbevis.

\textbf{Code-Behind:}
Når der oprettes en lektion, bliver der reserveret en båd på det pågældende tidspunkt.
Når der forsøges at forfremme et hold, tjekker Code-Behinden om eleverne har lært alt det de skal og ændrer herefter eleverne fra StudentMember til SailClubMember og sætter deres BoatDriver property til true.

Der findes også en undervisnings UserControl for elever. 
Denne kan ikke interageres med men viser blot hvilke læringsområde eleverne har lært, informationer om det hold de er på, samt hvornår næste lektion er.


\section{Vinduer}
Aller vinduer beskrevet i dette afsnit kan ses på \fxnote{Indset kilde til figur}

\subsection{Login brugergrænseflade}
 
\textbf{Formål}:
Login brugergrænsefladen har til formål at verificere en brugers identitet. 
Det er det første vindue brugeren ser når programmet åbnes, og det åbner hovedvinduet efter fuldført login.
 
\textbf{BrugerGrænseflade}: 
Brugergrænsefladen til loginvinduet er meget simpel og ligetil. 
Der er en TextBox til brugernavnet og en PasswordBox til kodeordet.
En PasswordBox er en TextBox, hvor hver af de indtastede karakterer vises som en sort prik, i stedet for de skrevne tegn, for at beskytte brugeren.
``Login som gæst''-knappen kræver ingen brugeroplysninger og åbner en let udgave af hovedvinduet, uden særlige tilladelser.
Til sidst er en TextBlock, som fortæller brugeren, hvad der er skrevet forkert, hvis der er skrevet noget forkert.

\textbf{Code-Behind}: 
TextBoxen, hvor brugernavnet indtastes, er case insensitive.
PasswordTextBoxen er case sensitive. 

\subsection{CreateBoatBookingWindow}
\textbf{Formål}: 
Dette vindue opretter, eller ændrer, en instans af RegularTrip.

\textbf{Brugergrænseflade}: 
Først vælges en båd, her anvendes en ComboBox.
Herefter anvendes \textbf{DateTimePicker}-UserControllen til at vælge et start- og sluttidspunkt.
Der vises den nuværende besætning, samt muligheden for at ændre den ved at trykke på en knappen ``Ændre Besætning'', som åbner CreateCrewWindow.
Når en besætning er valgt, kan der vælges en kaptajn ud fra besætningslisten.
Til sidst er der en TextBox, hvori brugeren kan angive formålet med turen.

\textbf{Code-Behind}: 
I Code-Behinden er der to constructors. 
Den ene bruges til at oprette en ny reservation, den anden bruges når der skal ændres på en eksisterende reservation. 
Efter tryk på ``Gem Reservation'', verificeres brugerens input fra alle de grafiske elementer i vinduet.

\subsection{CreateCrewWindow}

\textbf{Formål}: Dette vindue åbnes to steder i programmet: \textbf{CreateLogbookWindow} og i \textbf{CreateBoatBookingWindow}. 
Det bruges, når der skal laves en besætning til en RegularTrip.  

\textbf{Brugergrænseflade}: 
Der er to DataGrids, som hver indeholder en liste. 
Listen til venstre består af SailClubMembers, som bliver hentet ind fra databasen. 
Listen til højre indeholder det Crew, som man er i gang med at udforme til enten RegularTrip eller Logbook. 
Der er også tilføjet TextBoxe, så man kan skrive navnet på en gæst, man tog med på sejlturen. 
Der er knapper, som tilføjer de forskellige personer til Crewlisten. 
Øverst findes også et tekstfelt til at søge i listen over medlemmer. 
Når man har lavet sin liste, kan man trykke udfør, for at komme tilbage til vinduet, der kaldte CreateCrewWindow.

\textbf{Code-Behind}: 
Der er blevet brugt et \textbf{regular expression} til at tjekke, om den string brugeren angiver i de to TextBoxe for gæstens navne, er lovlige. 
Det er blevet valgt, at man må bruge hele det danske alfabet samt mellemrum, så navne såsom: Lars Peter Østergaard, er mulige.
Derefter tilføjes personen til listen, der vises i DataGridet til højre, og til sidst kaldes RefreshDataGrid, som man kan se på \myref{RefreshDatagrid}.

Her modtages der et DataGrid, som skal have dets Itemssource opdateret, og en ICollection, som er det data, der skal sættes ind i DataGridet. 
Det gøres ved at assigne dets ItemsSource til null, og derefter assigne det tilbage til den ICollection, der blev sendt med. 
Der kan laves en tilsvarende metode, som opdaterer andre WPF-controls.

\begin{lstlisting}[frame=single, caption=Refresh Datagrid, label=RefreshDatagrid]
private void RefreshDatagrid(DataGrid Grid, ICollection<Person> list)
{
    Grid.ItemsSource = null;
    Grid.ItemsSource = list;
}
\end{lstlisting}

\subsection{CreateLogbookWindow}

\textbf{Formål}: 
Dette vindue bruges til at udfylde logbogen for en sejltur.

\textbf{Brugergrænseflade}:  
Når vinduet åbnes indlæses der data fra den instans af RegularTrip som logbogen udfyldes for.
Som tidligere vinduer, bliver dataet afbilledet i de respektive grafiske elementer. 
De tomme felter skal selv udfyldes af brugeren. 

\textbf{Code-Behind}: 
Code-Behinden verificerer dataet inden det gemmes i databasen.
Herudover sættes referencen til logbogen i det pågældende RegularTrip.

\subsection{Undervisning}
