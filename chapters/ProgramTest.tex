\chapter{Program test}
I dette afsnit forklares der om teorien bag programtest samt udførelsen deraf. Foruden ses der nærmere på resultater af test og ses på hvor succesfuld programmet er. For at teste programmet udnyttes der i dette projekt en usability test. En sådan test lægger fokus på at teste tre primære emner hos brugeren: Funktionalitet, effektivitet og tilfredshed. 

For at kunne teste disse tre kriterier ses udvikles der først en profil af målgruppen. Idet dette program er lavet med hensigt af benyttelse i en sejlklub er denne gruppe rimelig stor. Målgruppen er hele sejlklubben, hvilket kan være rige som fattige og gamle som unge. Idet det er et computersystem er det eneste krav til målgruppen at de har basal kendskab til en computers egenskaber.

For at kunne udføre en test udvikles der således også en række mål for hvad brugeren ønsker at benytte programmet til. Ud fra disse mål kan der således opstilles scenarier, der kan testes. Programmet er ment som en hjælp med organisering, således er målene for programmet er relateret til klubbens organisatoriske opgaver. Nedenfor ses en liste af mål for den almindelige bruger hvorudfra der kan udvikles scenarier.
\begin{itemize}
  \item Følge ens status på duelighedsbevis.
  \item Se reservationer af både.
  \item Opret reservation af  både.
  \item Find andre medlemmer.
  \item Tilmeld sig begivenheder og undervisning.
  \item Følg status på begivenheder og undervisning(ændringer/aflysninger).
\end{itemize}
En administrator af programmet vil have få yderligere mål såsom opdatering af bådstatus og aflysning såvel som oprettelse af klubbegivenheder. Ud fra disse mål er lavet en række scenarier til usability test som er vedlagt i bilag. \fxnote{Bilaget er endnu ikke lavet}
\section{Funktionalitet}
I denne sektion ses der på produktets funktionalitet og udførelsen af programmets mål for den givne målgruppe. Funktionalitet ser på en overordnet succesrate af programmet og tester således for helt basale fejl i brugergrænsefladen der resultere i at opgaver ikke bliver løst. Funktionalitet kan udregnes ud fra hvor mange scenarier der er fejlet i forhold til udført. Funktionaliteten ønskes at ligge over 75 procent. 
\fxnote{indsæt succesrate diagram}
\section{Effektivitet}
Her ses der nærmere på den egentlige udførsel af hvert scenarie og ikke bare om det er udført eller ej. Effektivitet ser på hvor lang tid en bruger benytter på at blive bekendt med programmet og kunne løse opgaver. Denne værdi kan ses relativt i forhold til hver enkelt bruger, samt sættes op imod tidsforbrug af en expertbruger, en af programmets arkitekter, for at danne et indtryk af hvor let programmet er at gå til.
\fxnote{indsæt tidsforbrug diagram}
\section{Tilfredshed}
Til sidst laves der en tilfredshedsundersøgelse for at få brugerens meninger omkring programmets design. Dette kan gøres igennem et spørgeskema. Undersøgelsen skal give et indtryk af hvor godt brugeren synes programmede var i forhold til layout, placering af de krævede objekter i programmet, deres synlighed i programmet. Denne undersøgelse giver et indtryk af hvor godt programmet er designet, og derved hvor let forskellige funktioner i programmet er at tilgå. Et spørgeskema kan ses i bilag. \fxnote{indsæt bilag samt svar på spørgeskema}

\cbend