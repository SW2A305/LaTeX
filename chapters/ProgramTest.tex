\chapter{Test af program}\label{test_af_program}
I dette kapitel beskrives de to typer af tests, der er udført i forbindelse med projektet.
Under udviklingen af programmet har projektgruppens medlemmer testet funktionaliteterne løbende ved hjælp af blackbox testing.

\section{Unit test}
Til test af selve koden er der benyttet Unit tests.

Unit tests benyttes primært til at teste metoder, samt properties med logik.
Når man har skrevet en metode, så forventer man, som regel, et bestemt resultat ved et givent input.
Unit test af en metode fungerer ved, at man giver metoden, som skal testes, et input, og ser om man får det forventede resultat.

Der findes flere frameworks til unit testing.
Til projektet benyttes frameworket NUnit, som er et open source unit testing framework til .NET.
Det er ikke alt kode, som bliver testet, men relevante metoder og properties i udvalgte klasser.
Der er bl.a. skrevet unit tests til noget af mock data \ac{DAL}--implementationen, for at afprøve nogle formodninger, før hele programmet blev benyttet i testen.
De valgte tests på dét område blev valgt ud fra formodningen om, at fejl i de metoder kunne vise sig som anderledes fejl andetsteds i programmet, og dermed gøre det svært at lokalisere den egentlige fejl.

Noget af det mest gennemtestede kode, baseret på dets høje code coverage, er en hjælpeklasse, der importerer medlemmer fra en XML--fil, ind i det bagvedliggende persistenslag. 
Code coverage er en indikator for, hvor stor en procentdel af koden der testes.
Denne hjælpeklasse benyttes ikke i programmet, men er forblevet deri som kode, med henblik på, at den kunne blive nyttig i forbindelse med en udvidelse af programmet.

I \myref{unit_test_setup} ses en såkaldt SetUp--metode, som køres inden hver test udføres, og dermed gør det muligt at nulstille til en bestemt tilstand, så tests kan udføres på en ensartet, systematisk måde.
Til test af XML--import er der blevet benyttet et substitutionsframework, kaldet NSubstitute, der muliggør simulering af interfaces, når den bagvedliggende implementation ikke er vigtig for funktionaliteten.
Her bruges den specifikt til at simulerer den del af programmets \ac{DAL}, som håndterer medlemmerne i sejlklubben.

\newpage

\begin{lstlisting}[frame=single, caption=Eksempel på Unit test SetUp, label=unit_test_setup]
  [SetUp]
  public void Initialize()
  {
    _memberDalSubstitute = Substitute.For<ISailClubMemberDal>();
    _parser = new XmlMemberParser(_memberDalSubstitute);
    _sailClubMemberList = new List<SailClubMember>();
    
    // React to "creation" calls
    _memberDalSubstitute
          .Create(Arg.Any<SailClubMember>())
          .ReturnsForAnyArgs(true)
          .AndDoes(x => _sailClubMemberList.AddRange((SailClubMember[])x[0]));
  }
\end{lstlisting}

Når SetUp--metoden er skrevet, skrives de enkelte tests, ud fra devisen om, at hver test kun skal teste én ting.
Der er flere måder at navngive tests på.
Den metode, som gruppen valgte, består af tre dele, adskilt af \_ (underscore):

\begin{itemize}
  \item Første del beskriver den metode eller property, der testes.
  \item Anden del beskriver inputtet til metoden eller propertien.
  \item Tredje del beskriver det forventede udfald af testen.
\end{itemize}

Navngivningen er valgt på baggrund af, at gruppens medlemmer finder det lettere at se, hvor problemet ligger, hvis en eller flere tests fejler.

Et eksempel på en unit test kan ses på \myref{unit_test}.
Her testes metoden \texttt{ImportMembersFromXml()}, med input af et enkelt medlem, og med en forventning om, at det ene medlem bliver gemt.
Først kaldes metoden \texttt{InitializeXmlStrings1()}, som genererer en tekststreng med indholdet af det, der ville stå i en XML--fil med ét medlem.
Herefter kaldes \texttt{ImportMembersFromXml()} med det simulerede XML--data.
Endeligt benyttes NSubstitute til at verificere, at \texttt{Create()}--metoden i det angivne \ac{DAL} blev kaldt én gang, med et \texttt{SailClubMember} objekt.


\begin{lstlisting}[frame=single, caption=Eksempel på Unit test, label=unit_test]
  [Test]
  public void ImportMembersFromXml_InputtingOneMember_OneMemberSaved()
  {
    this.InitializeXmlStrings1();
    _parser.ImportMembersFromXml(_stream);
    
    _memberDalSubstitute.Received(1).Create(Arg.Any<SailClubMember>());
  }
\end{lstlisting}

\section{Brugertest}
I dette afsnit forklares fremgangsmåden for brugertests i forbindelse med programmets brugervenlighed.
Derudover ses der nærmere på, hvordan resultaterne bruges, hvorfra der kan dannes et overblik over programmets funktionalitet. 

For at teste programmet anvendes der, foruden unit tests i dette projekt, en række brugertests, som tager udgangspunkt i projektets UserStories, \myref{User_stories}.
Testene er inspireret af user acceptance Test, men da det ikke har været muligt at finde nogle af Sejlklubben Sundets medlemmer til at teste, er det ikke en konkret user acceptance test. 
Testen består i at få personer, som indgår i den endelige brugers segment til at prøve programmet, og udføre nogle opgaver som simulerer den reelle brug af programmet i sejlklubben \citep{UAT}.
Denne type test lægger fokus på uafhængige brugeres meninger om programmets brugervenlighed.
Testens form er inspireret af black-box testing, som også er blevet brugt i forbindelse med udviklingen af programmet, som normalt ellers udføres af udviklerne selv.

Da programmet er lavet med henblik på brug i en sejlklub, består målgruppen af mange forskellige befolkningssegmenter. 
Idet det er et computerprogram, er det eneste krav til målgruppen, at de er villige til at bruge en computer.

Ud fra programmets funktionaliteter dannes der nogle mål, som brugerne skal opnå igennem testen. 
Der defineres en række testscenarier med udgangspunkt i de mål, der blev sat for testen.
Disse mål er:
\begin{itemize}
  \item Følge sin undervisningsstatus.
  \item Se reservationer af både.
  \item Oprette reservation af både.
  \item Tilmelde sig begivenheder.
  \item Følge status på begivenheder og undervisning.
  \item Oprette logbog for fuldendt sejlads.
\end{itemize}

Ud fra disse mål er der skrevet en række korte opgaver, som berører bestemte funktionaliteter i programmet. 
Opgaverne kan findes i \myref{BrugerTestCases}.

Dette bilag vil blive vist til testpersoner af programmet, og de vil blive observeret under testen. 
Efter udførelsen af testen spørges der ind til testpersonens oplevelse af programmet med henblik på funktionalitet, brugergrænsefladens design og deres helhedsindtryk.

En opgave kunne eksempelvis være:
\begin{itemize}
	\item Log ind på systemet med følgende brugeroplysninger:
	\newline - Brugernavn: oskar
	\newline - Kodeord: lauridsen
	\item Foretag en booking af båden: Anna til næste lørdag, med følgende andre oplysninger: 
	\begin{itemize}
		\item Starttidspunkt: 20-07-2014 13:37
		\item Sluttidspunkt: 20-07-2014 20:42
		\item Besætning: Jens Hansen, Anders And\ldots
		\item Kaptajn: Anders And
		\item Formål: Eftermiddagssejlads
	\end{itemize}
	\item Log ud af systemet igen
\end{itemize}

Samtidigt udfyldes et skema med tidsforbrug af hver enkelt opgave, samt kommentarer omkring brugerens færden i programmet.

I de følgende afsnit beskrives der, hvordan testens data vil blive behandlet.

\subsection{Funktionalitet}
I testområdet funktionalitet undersøges der hvorvidt opgaverne stillet er mulige at løse for testpersonerne.
Dermed testes der på, om brugergrænsefladen hindrer programmets funktionalitet. 

\subsection{Effektivitet}
Effektivitet anses som værende forholdet mellem en ekspert og en nybegynders tid, for at udføre en given opgave \citep{UIEffeciency}.
Derfor tages der tid på udførelsen af, hver enkelt opgave som testpersonen udfører.
Eksperten vil være et medlem af projektgruppen, som ikke var observatør i nogen af de andre testpersoners test. 

\subsection{Tilfredshed}
Til sidst laves der en tilfredshedsundersøgelse for at få brugerens meninger omkring programmets design. 
Dette gøres igennem et spørgeskema. 
Undersøgelsen skal give et indtryk af, hvor godt brugeren synes programmet var i forhold til layout, placering og synlighed af de krævede elementer i programmet. 
Denne undersøgelse giver et indtryk af, hvor godt programmet er designet, og sværhedsgraden af forskellige funktioner. 
Spørgeskemaet kan ses i \myref{bilag:SporgeSkema}. \citep{UISatisfaction}


\section{Analyse af testresultater}

Grundet tidspres, fra bl.a. problemer med databaseimplementationen, var der kun tid til at teste på fire personer, foruden eksperten.
Personerne der udførte testene, bestod af elever på Aalborg Universitet, som ikke havde set eller på anden vis hørt om projektet før.
Resultaterne kan ikke konkluderes til at give et endegyldigt svar på om programmet kan bruges i Sejlklubben Sundet, da testpersonerne ingen kendskab har til sejlklubben og dennes behov. 
For at få et mere brugbart resultat fra testen, skulle testpersonerne være en repræsentativ gruppe af medlemmer fra sejlklubben, eller på anden vis være involveret med en sejlklub. 
På denne måde, ville testpersonen have en bedre forståelse for kravene til programmets funktionalitet.
Dog menes det, at testens resultater alligevel kan sige noget om brugen af programmet, både om navigering og de testede funktionaliteter.

\subsection{TestData}
På \myref{tab:TestTimeTable} kan der ses en tabel over tid brugt ved hver opgave for hver enkelt testperson sammenlignet med ekspertens tid.

% Table generated by Excel2LaTeX from sheet 'Sheet1'
\begin{table}[htbp]
    \colorlet{shadecolor}{gray!40}
    \rowcolors{1}{white}{shadecolor}
  \centering
    \begin{tabular}{r|rrrr|r|r|r}
    \textbf{Opgave} & T1     & T2     & T3     & T4     & \textbf{Gennemsnit} & \textbf{Ekspert} & \textbf{Forhold} \\ \hline
    1     & 00:15 & 00:32 & 00:24 & 00:31 & 00:25 & 00:15 & 1,7 \\
    2     & 04:42 & 04:00 & 02:04 & 02:27 & 03:18 & 01:35 & 2,1 \\
    3     & 01:08 & 02:13 & 01:02 & 01:10 & 01:23 & 00:30 & 2,8 \\
    4     & 00:49 & 00:32 & 00:32 & 00:55 & 00:42 & 00:29 & 1,4 \\
    5     & 00:29 & 00:24 & 00:27 & 00:30 & 00:27 & 00:13 & 2,1 \\
    6     & 01:16 & 00:57 & 00:57 & 01:02 & 01:03 & 00:50 & 1,3 \\
    7     & 02:00 & 00:35 & 00:30 & 00:40 & 00:56 & 00:14 & 4,0 \\
    8     & 02:02 & 04:20 & 04:57 & 02:10 & 03:22 & 00:56 & 3,6 \\
    9     & 00:32 & 00:26 & 00:19 & 00:42 & 00:29 & 00:10 & 3,0 \\ \hline
    \textbf{SUM} & 11:56 & 13:59 & 11:12 & 10:07 & 11:48 & 05:12 & 2,3 \\
    %\bottomrule
    \end{tabular}%    
    \caption{Tid brugt pr opgave under test. T[1-4] er udefrakommende testpersoner.\newline Forhold udregnes: $ \dfrac{Gennemsnit}{Ekspert} $}
  \label{tab:TestTimeTable}%
\end{table}%

Det lykkedes for testpersonerne at udføre alle opgaverne.

\subsection{Diskussion af test data}

Nogle opgaver var tydeligt sværere end andre for personer, som ikke havde kendskab til programmets brugergrænseflade. 
Særligt var opgave 2, 7, og 8 udfordrende.


Opgave 2 lød på at reservere en bestemt båd, med udleverede data.
Den tiltænkte løsning af opgaven er:
\begin{enumerate}
    \item Vælg tabben ``Både''.
    \item Vælg båden ``Anastasia''.
    \item Tryk på knappen ``Reserver båden'' (Dette åbner CreateBoatBookingWindow).
    \item Udfyld felterne med det udleverede data.
    \item Tryk på knappen ``Gem reservation''.
\end{enumerate}

T1 og T2 havde svært ved det 3. skridt i opgaven.
Fælles var, at de havde brugt over to minutter på netop dette skridt.
Efter udførelse af testen blev de bedt om at uddybe, hvad dette skyldtes.
De mente begge, at der var for meget støj på tabben ``Både'', og dette distraherede dem fra at finde den korrekte knap.
T3 og T4 havde i kontrast ingen problemer med at gennemgå dette skridt, og brugte hver under fem sekunder på at finde den korrekte knap. 
Eksperten brugte væsentligt mindre tid end T1 og T2, men sammenlignet med T3 er forskellen betydeligt mindre.

\textbf{Testpersonernes forslag til forbedring af funktionalitet i denne opgave:}
\begin{itemize}
    \item Ryd op i tabben ``Både''. 
    \item Herunder flyt alt logbogsrelateret til sin egen tab.
    \item Øg størrelsen på knappen ``Reserver båden''.
\end{itemize}

I opgave 7 skulle testpersonerne forfremme et hold i sejlerskolen. 
Den tiltænkte løsning af opgaven er:
\begin{enumerate}
    \item Vælg tabben ``Undervisning''
    \item Vælg holdet ``MasterRace''
    \item Tjek CheckBoxen ``Rediger hold''
    \item Tryk på knappen ``Forfrem hold''
    \item Tryk på knappen ``Log ud''
\end{enumerate}

T1 havde særlig svært ved regne ud, at man skal udføre det 3. skridt, før man kunne udføre det 4. skridt.
T1 mente, at der var for mange elementer i tabben ``Undervisning'', og at dette forvirrede i udførelsen af opgaven, da de fjernede opmærksomheden fra CheckBoxen i toppen af tabben.

\textbf{Testpersonernes forslag til forbedring af funktionalitet i denne opgave:}
\begin{itemize}
    \item Flyt rediger hold funktionaliteten til sit eget vindue, ved tryk på en knap.
    \item Der er ingen respons fra programmet efter det 4. skridt, tilføj et popup vindue som giver besked.
\end{itemize}


I opgave 8 blev testpersonerne bedt om at udfylde logbogen for en fuldført tur, som brugeren, de var logget ind som, havde reserveret, og derfor skulle udføre logbogen for.

Den tiltænkte løsning af opgaven er:
\begin{enumerate}
    \item Log ind
    \item Marker sejlturen i det højre DataGrid på forsiden.
    \item Tryk på knappen ``Opret logbog''
    \item Tryk på knappen ``Ændre besætning''
    \item Fjern ``Kasper Eriksen''
    \item Ændre den ``Faktisk afgang'' til 10:00
    \item Tryk på ``Ja'' under ``Er båden skadet?''
    \item Angiv vejrforhold og skadesrapport
    \item Tryk på knappen ``Udfør''
\end{enumerate}

Alle fire testpersoner havde svært ved skridt 2 og 3.
Særligt T2 og T3 brugte lang tid på dette problem.
Testpersonerne besøgte tabben ``Både'' for at udføre dette. 
Eksperten havde ingen problemer med denne opgave og gennemførte opgaven på under et minut.

\textbf{Testpersonernes forslag til forbedring af funktionalitet i denne opgave:}
\begin{itemize}
    \item Giv tydelig besked og instrukser efter log in, om at en logbog mangler at blive udfyldt.
    \item Forøg størrelsen af den forklarende tekst på forsiden til hver af de to DataGrids.
    \item Øg afstanden fra velkomstbeskeden til de to DataGrids.
    \item Mindske højden af de to DataGrids, da de alligevel er hovedsageligt tomme. 
    \item Dediker en tab til læsning og udfyldelse af logbøger. 
\end{itemize}

\subsection{Testpersonernes samlede vurdering}

I spørgeskemaet blev testpersonerne bedt om at vurdere flere centrale skærmvinduer.
De blev bedt om at vurdere, deres evne til at overskue hvert af de angivne skærmvinduers funktionaliteter.
Dette skulle angives som en karakter fra et til fem, hvor et er meget uoverskueligt og fem er meget overskueligt.
Dette kan ses i skemaet \myref{tab:vurderingtest}.

% Table generated by Excel2LaTeX from sheet 'Sheet1'
\begin{table}[htbp]
    \colorlet{shadecolor}{gray!40}
    \rowcolors{1}{white}{shadecolor}
    \centering
    \begin{tabular}{l|rrrr|r}
        \textbf{Vindue} & T1 & T2 & T3 & T4 & \textbf{Gennemsnit} \\
        \midrule
        Forside & 5 & 4 & 4 & 3 & 4,00 \\
        Undervisning & 1 & 2 & 3 & 2 & 2,00 \\
        Medlemmer & 4 & 5 & 4 & 2 & 3,75 \\
        Både & 3 & 4 & 3 & 3 & 3,25 \\
        Begivenheder & 4 & 4 & 3 & 2 & 3,25 \\
        Logbogsvinduer & 3 & 2 & 2 & 4 & 2,75 \\
        CreateCrewWindow & 5 & 5 & 4 & 3 & 4,25 \\
        CreateBoatBookingWindow & 5 & 5 & 4 & 4 & 4,50 \\ \hline
        \textbf{Samlet Vurdering} & 3,75 & 3,88 & 3,38 & 2,88 & 3,47 \\
    \end{tabular}%
        \caption{Testpersonernes vurdering af centrale skærmvinduer}
    \label{tab:vurderingtest}%
\end{table}%

\subsubsection*{Kommentarer fra testpersonerne}
Foruden \myref{tab:vurderingtest} argumenterede testpersonerne også for deres vurdering.

\textbf{Forside}: 
Generelt kunne testpersonerne lide formålet med forsiden, som er at give brugeren et hurtigt overblik efter login. 
Dog mente flere, at layoutet kunne have været udført bedre.
Herunder større afstand fra velkomstteksten til de to DataGrids, større overskrifter ved hver DataGrid, og lave de to DataGrids mindre da de hovedsageligt er tomme. 

\textbf{Undervisning}: 
Generel utilfredshed omkring denne UserControl.
Specifikt var der klager vedrørende, brugen af CheckBoxes til at tjekke elevers fremmøde.
Her mente flere, at CreateCrewWindow-vinduet kunne have været anvendt. 
Derudover tilføjede den højre side af vinduet, der havde et layout som lignede CreateCrewWindow-vinduet, forvirring. 
Der var også negative kommentarer omhandlende CheckBoxen ``Rediger hold''. 
Flere fandt denne unødvendig, mens en anden mente, at det indhold denne låste op for, burde have sit eget dedikerede popup vindue.

\textbf{Medlemmer}:
Testpersonerne skulle ikke bruge dette vindue i opgaverne.
Derfor fik de vist dette vindue efter testens gennemførsel, således de kunne kommentere tabbens indhold. 
Der var generelt positive kommentarer, dog mente T4, at den store mængde data virkede overvældende. 
T3 og T4 mente at søgebaren ikke var klart nok markeret, og foreslog at der kunne tilføjes et ``forstørrelsesglas''--ikon.

\textbf{Både}:
Flere testpersoner udtrykte, at denne UserControl var uoverskuelig, da der vises for meget information efter valg af en båd.
T1, T3 og T4 forslog at flytte alt logbogsrelateret til sin egen tab.
Derudover kunne størrelsen af skrift forstørres, således den ville være nemmere at læse. 

\textbf{Begivenheder}:
Kommentarer til denne tab var få, T3 foreslog at ændre overskriften ``Agenda'' til ``Begivenheder'', over ListBoxen.
Flere mente at brugen af ListBoxen burde have været erstattet med et DataGrid, som er anvendt mange andre steder i programmet. 

\textbf{Logbogsvinduerne}:
Kritikken af disse var primært rettet mod navigeringen til vinduet. 
Der var der uenighed imellem testpersonernes vurdering af vinduets design.
Én mente at elementerne i det tospaltede layout var uoverskueligt, særligt pga. rækkefølgen af disse. 
En anden udtrykte positive tanker om layoutet.

\textbf{CreateCrewWindow}:
Der var næsten kun positive kommentarer om dette vindue.
Vinduet blev omtalt som værende overskueligt, dog mente de det venstre DataGrid burde være større.
T3 foreslog, at dette kunne have samme størrelse som det til højre. 
Her blev søgefunktionen igen overset, og der blev igen foreslået tilføjelsen af et ``forstørrelsesglas''--ikon.

\textbf{CreateBoatBookingWindow}:
Der var meget få kritikpunkter til dette vindue.
T4 mente at knapperne ``Gem reservation'' og ``Annuller reservation'' var for brede og burde være placeret således, de lignede dem fra Windows vinduer såsom ``egenskaber''. 
Dvs. hvor de er placeret nederst i højre hjørne, på linje med hinanden. 

\subsection{Konklusion af brugertest}
Brugertesten har givet os en større indsigt i programmets layout.
Målene for layoutet var minimalisme og konsistens, men dette er ikke opnået, grundet problemer i arbejdsprocessen, se \myref{sec:faktiske-arbejdsproces}. Dette uddybes i \myref{chap:diskussion}.
 
På baggrund af brugertesten etableres det, at nogle dele af brugergrænsefladen er bedre konstrueret end andre.
De bedste vurderes til at være vinduerne: \textbf{CreateCrewWindow} og \textbf{CreateBoatBookingWindow}. 
Disse vinduer har begge en høj overskuelighed, dette er et af principperne fra minimalisme, altså at fjerne det unødvendige. 

Ud fra testpersonernes kommentarer kunne de resterende vinduers og UserControls layout forbedres. 
Særligt skulle tabben ``Undervisning'' og tabben ``Både'' forbedres.
Konkrete forslag til disse forbedringer er givet af testpersonerne, disse udtrykker projektgruppen enighed i. 

Da projektgruppen har en begrænset mængde tid, bliver disse rettelser ikke implementeret. 
