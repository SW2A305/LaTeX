\chapter{Test af program}
I dette kapitel, beskrives de to typer af tests der er udført i forbindelse med projektet.

\section{Unit test}


\section{Brugertest}
I dette afsnit forklares hvordan brugertests udføres i forbindelse med programmets brugervenlighed.
Derudover ses der nærmere på hvordan resultaterne bruges, hvorfra der kan dannes et overblik over programmets funktionalitet. 
For at teste programmet anvendes der, foruden unit tests i dette projekt, en række brugertests, som tager udgangspunkt i projektets UserStories\myref{User_stories}
Denne type test ligger fokus på uafhængige brugers mening om programmets brugervenlighed.

Da programmet er lavet med henblik på benyttelse i en sejlklub, er målgruppen stor. 
Idet det er et computersystem, er det eneste krav til målgruppen, at det er villige til at bruge en computer.

Ud fra programmets funktionaliteter dannes der nogle mål som brugerene skal opnå igennem testen. 
Der defineres en række testscenarier med udgangspunkt i de mål der blev sat for testen.
Disse mål er:
\begin{itemize}
  \item Følge sin undervisningsstatus.
  \item Se reservationer af både.
  \item Oprette reservation af både.
  \item Tilmelde sig begivenheder.
  \item Følg status på begivenheder og undervisning(ændringer/aflysninger).
  \item Oprette logbog for fuldendt sejlads.
\end{itemize}

Ud fra disse mål kan der skrives en række korte opgaver, som berører ét bestemt område i programmet. 
Opgaverne kan findes i \myref{BrugerTestCases}.

Dette bilag vil blive vist til testpersoner af programmet, og de vil blive observeret mens de forsøger at udføre den opgave de har fået. 
Efter udførslen af opgaven de er blevet stillet, spørges der ind til brugerens oplevelse af programmet med henblik på funktionalitet, om brugergrænsefladen var forvirrende eller guidede testpersonerne i den rigtige retning og eventuelt andre meninger fra testpersonerne.

En opgave kunne eksempelvis være:
\begin{itemize}
	\item Log ind på systemet med følgende brugeroplysninger: oskar og lauridsen
	\item Foretag en booking af båden: Anna til næste lørdag, med følgende andre oplysninger: 
	\begin{itemize}
		\item Starttidspunkt: 20-07-2014 13:37
		\item Sluttidspunkt: 20-07-2014 20:42
		\item Besætning: Jens Hansen, Anders And\ldots
		\item Kaptajn: Anders And
		\item Formål: Eftermiddagssejlads
	\end{itemize}
	\item Log ud af systemet igen
\end{itemize}

Samtidigt udfyldes et skema med tidsforbrug af hver enkelt opgave, samt kommentarer omkring brugerens færden i programmet, hvis noget er bemærkelsesværdigt. 

I de følgende afsnit beskrives der hvordan disse data vil blive behandlet.

\section{Funktionalitet}
I denne sektion ses der på produktets funktionalitet og udførelsen af programmets mål for den givne målgruppe.
Funktionalitet ser på en overordnet succesrate af programmet og tester således for helt basale fejl i brugergrænsefladen, der resultere i at opgaver ikke bliver løst. 

\section{Effektivitet}
Effektivitet anses som værende forholdet mellem en erfaren bruger og en nybegynders tid for at udføre en given opgave.\citep{UIEffeciency}
Derfor tages der tid på udførelsen af, hver enkelt opgave som testpersonen udfører.

\section{Tilfredshed}
Til sidst laves der en tilfredshedsundersøgelse for at få brugerens meninger omkring programmets design. 
Dette kan gøres igennem et spørgeskema. 
Undersøgelsen skal give et indtryk af, hvor godt brugeren synes programmet var i forhold til layout, placering af de krævede objekter i programmet og deres synlighed i programmet. 
Denne undersøgelse giver et indtryk af, hvor godt programmet er designet, og derved hvor let forskellige funktioner i programmet er at tilgå. 
Et spørgeskema kan ses i \myref{bilag:SporgeSkema}. \citep{UISatisfaction}
