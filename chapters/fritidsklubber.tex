\chapter{Fritidsklubber} \label{chap:Fritidsklubber}

I følgende afsnit vil forskellige fritidsklubber blive undersøgt, for at afdække de respektive administrative poster, som kunne forekomme i sådanne klubber. 
Afsnittet er skrevet ud fra gruppens egne erfaringer, hvilket gøres, da det ikke har været muligt, at finde eksterne kilder omhandlende administrationen af de respektive typer af fritidsklubber.  \fxnote{Søren: tilføj : på trods af utallige forsøg}
Det er taget i betragtning, at disse erfaringer kan være specifikke for den enkelte klub, hvor erfaringen stammer fra. 
Dog vurderes der, at de forekommende administrative opgaver er generelle opgaver, og altså ikke enkelttilfælde. 

Begrebet fritidsklubber dækker over klubber, hvor voksne og unge kan bruge deres fritid, hvad end det er socialt, sportslig eller begge. 
Fritidsklubberne er valgt ud fra det faktum, at de ofte er drevet af frivilligt arbejde og har faciliteter der udlejes. 
Denne type frivilligt arbejde, er individuel fra klub til klub, hvad end det er undervisning, håndtering af det økonomiske eller noget helt andet. \fxnote{Nikolaj: "noget helt andet" talesprog?}

Dog er der nogle administrative opgaver, som går igen blandt fritidsklubberne.
Det er primært opgaver, som omhandler medlemshåndtering, da klubber har en medlemsbase, som skal administreres. \fxnote{Troels: Forslag til erstatningssætning: Det er primært opgaver som omhander administraion af medlemmer.}
De generelle administrative opgaver omkring medlemshåndtering kan være følgende:
\begin{itemize}
	\item Kontingentbetaling
	\item Undervisning/træning
	\item Tunneringsbetaling
\end{itemize}

%\fxnote{Kan være der her skal inkluderes en liste af mere generelle ting?(Medlemskontigent, undervisning osv. og så
% fjerne det fra de enkelte afsnit, på den måde undgås at dette blot skrives på variende måder}
%\subsection{Sejlklub}\label{subsec:sejlklub}

\section{Sportsklubber} \label{Sportsklubber}

I sportsklubber kan der forekomme mange forskellige administrative opgaver. 
Det er forskelligt fra sportsklub til sportsklub, hvilke opgaver der er tale om, men der er nogle fællestræk, som går igen blandt klubberne. 
Disse fællestræk er bl.a.:
\begin{itemize}
	\item Udlejning/udlån af aktiver, f.eks. baner og sportsudstyr. 
	\item Arrangering og pointtildeling ved sportsarrangementer.
	\item Arrangering af kørsel til og fra arrangementer.
	\item Kommunikation til forældre og sportsudøvere.
\end{itemize}

Sportsklubber kan også have specielle forhold, som spiller ind, grundet de forskellige sportsgrene de udbyder, f.eks. ved en skytteklub gælder der specielle regler for våbenlicenser og våbentransport.
Golfklubber har også specielle forhold omkring deres baner, da de, rent arealmæssigt, er meget større end andre sportsbaner. \fxnote{Troels: Fjern dette eksempel, eller tilføj flere?(Søren): Jeg stemmer for vi bare fjerne golf eksemplet. Det er desuden et dårlig eksempel, at de bare har større baner.}


%\subsection{Fodboldklub} \label{Fodbold}

%I en fodboldklub er der mange administrative og organisatoriske opgaver, som kunne have gavn af et elektronisk
%system til håndtering af disse opgaver frem for et manuelt system. Eksempler på sådanne administrative opgaver
%kan være:

%\begin{itemize}
%  \item Holde styr på baner, hvilket hold som spiller/træner hvor og hvornår
%  \item Kørsel til og fra arrangementer og lignende
%  \item Pointgivning ved lokale sportsarrangementer
%\end{itemize}

%Små lokale klubber kan også have vask af trøjer og andet udstyr gående på skift blandt medlemmerne, hvor
%forældrene hurtigt vil kunne se hvis tur det er via et administrationssystem.\citep{fodbold1}\citep{fodbold2}


%\subsection{Badminton-- og tennisklub}

%En badmintonklub har ligesom ved fodbold, nogle baner som klubben har til rådighed. Dog skiller badminton sig
%ud ved, at det ofte foregår i lokale sportshaller, hvor andre sportsklubber også har bane, f.eks. håndbold.
%Tennis er her meget ens med badminton, den største forskel værende, at de typisk har deres egne baner samt,
%det ofte foregår udendørs. Disse baner kan typisk lejes, hvilket også skal håndteres. Administrationen
%indebærer blandt andet følgende opgaver:

%\begin{itemize}
%  \item Arrangering af sportsarrangementer
%  \item Udlejning af baner
%\end{itemize}


%\subsection{Skytteklub}

%I en skytteklub kan der foregå mange forskellige typer skydning. Der kan forekomme op til flere administrative
%opgaver, som der i mindre klubber udføres manuelt. Sådanne administrative opgaver kan være følgende:

%\begin{itemize}
%  \item Administration af skydebaner og evt. reservation
%  \item Kørsel til og fra stævner og andre arrangementer
%  \item Håndtering af våbenlicenser internt i foreningen
%\end{itemize}


%\subsection{Golf}

%Golf skiller sig ud fra de tidligere nævnte fritidsklubber ved, at golf foregår på et meget større område.
%Ligeledes er der på de græsarealer et bestemt antal huller, og klubben kan have flere baner af ofte 9 eller 18
%huller.

%Da der er flere grupper af spillere på samme bane samtidigt, skal klubben i stedet holde styr på hvem der
%starter hvornår på 1. hul. Herudover, kan det være centralt for klubben at have en måde at
%registrere om gruppen vil have golfbiler med, og endda caddier, hvis det er noget golfklubben også tilbyder.

%Det vil sige, at der i stedet for ressourceplanlægning, er mere tale om skemalægning. Altså hvem der spiller
%hvornår, og med tilvalg såsom golfbiler eller caddier.


\section{Haller}

Sportshaller lægger lokale til mange forskellige sportsklubber.
Fodboldklubber bruger omklædningen og de udendørs baner til træning. 
Badmintonklubber bruger de indendørs baner, som de skal dele med andre aktiviteter f.eks. håndbold, indendørs fodbold, indendørs hockey osv.
Sportshaller har også ofte lokale arrangementer, f.eks. foredrag, ungdomsklub osv.
Haller har ofte også en kiosk eller café, hvor sportsudøvere eller arrangementsgæster kan komme og få noget at spise og drikke. \fxnote{Troels: Er dette relevant for projektet?}
Haller kan også have et motionscenter, hvor idrætsklubber og almindelige personer kan købe adgang\citep{spt_hal}. 
Eksempler på administrative opgaver kunne være:

\begin{itemize}
  \item Administration af haludlejning og omklædning \fxnote{Slet hal og skriv:: udlejning istedet?? (Søren)}
  \item Information fra kiosken
  \item Informationsdeling til sportsklubber og andre interesserede
\end{itemize}


\section{Ungdomsklub}

En ungdomsklub ses som et sted, hvor unge kan bruge deres fritid på sociale aktiviteter.
Det varierer, hvordan ungdomsklubber drives. \fxnote{Troels: Det varierer fra klub til klub hvordan den drives.}
Nogle drives af frivillige, hvor andre drives af skolevæsnet, f.eks. ungdomsskoler\citep{ung1}.
Ofte holder ungdomsklubber til i skolebygninger. 
Hvad end de drives af frivillige eller skolevæsnet, har de stadig nogle administrative opgaver, som skal løses. 
Eksempler på sådanne administrative opgaver kunne være:

\begin{itemize}
  \item Arrangementsplanlægning
  \item Bestilling af varer (hvis der sælges sodavand, slik o.l.)
\end{itemize}
\fxnote{Hvilken relevans har Ungdomsklubber i henhold til projektet?? - De har ikke rigtig noget til fælles med sejlklubber, andet en begivenheder. - Det var ment som en tilføjelse til fritidsklubber, da det måske kan virke som om vi har set lidt snævert på fritidsklubber (Thomas)}


\section{Sejlklub}

En sejlklub kan, alt efter størrelsen, have flere administrative opgaver.
Disse opgaver kan bl.a. bestå af administration af deres både, herunder hvem der har reserveret en båd, hvornår båden er ledig osv.
Nogle sejlklubber har også sejlundervisning, hvor der bl.a. skal kunne holdes styr på hvilke lektioner, eleverne har haft. 
Andre eksempler på administrative opgaver i en sejlklub kunne være:

\begin{itemize}
  \item Af- og tilmelding af undervisning
  \item Reservering og udlån af både \fxnote{Troels: Tilføj se hvem og hvornår der reservers både? Se og skriv logbøger (vi næver dette i det initierende problem.)}
  \item Oversigt over hvem der skylder penge for lån af båd
\end{itemize}

\section{Afgrænsning}

Ud fra ovenstående fritidsklubber afgrænses der til fremover udelukkende at beskæftige sig med sejlklubber. 
Denne afgrænsning finder sted, da det vurderes, at der er potentiale for at gøre arbejdet i en sejlklub nemmere. \fxnote{Tilføj: Grundet de mange administrative udfordringer. (Søren)}
Ligeledes findes sejlklubber specielle, da nogle sejlklubber har en undervisningsfunktion, og derfor har specielle administrative funktioner.
