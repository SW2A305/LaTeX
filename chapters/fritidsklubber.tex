\chapter{Analyse af fritidsklubber} \label{Fritidsklubber}

I følgende afsnit vil vi kigge nærmere på forskellige fritidsklubber, for at afdække de administrative poster som kunne forekomme. Fritidsklubber dækker over klubber, hvor unge og voksne hvor unge og voksne kan bruge deres fritid, om det er socialt, sportslig eller begge. Fritidsklubberne, der omhandler sport, er valgt ud fra den sportsgren som de repræsenterer; At det er populære og kendte sportsgrene. De er også valgt ud fra det faktum, at de ofte er drevet af frivillig arbejde. Denne type frivillig arbejde, er meget individuel fra klub til klub, om det er undervisning, håndtering af det økonomiske eller noget helt andet. 

%\fxnote{Kan være der her skal inkluderes en liste af mere generelle ting?(Medlemskontigent, undervisning osv. og så fjerne det fra de enkelte afsnit, på den måde undgås at dette blot skrives på variende måder}
%\subsection{Sejlklub}\label{subsec:sejlklub}

\subsection{Fodboldklib} \label{Fodbold}
I en fodboldklub er der mange administrative og organisatoriske opgaver, som kunne have gavn af et elektronisk system
til håndtering af administrative opgaver frem for et manuelt system. Eksempler på sådanne
administrative opgaver kunne være:

\begin{itemize}
\item Holde styr på baner, hvilket hold som spiller/træner hvor og hvornår
\item Medlemshåndtering; alder, spillestatus (skader, niveau ol.)
\item Holdsammensætning og taktik
\item Medlemsbetaling
\item Kørsel til og fra arrangementer og lignende
\item Pointgivning ved lokale sportsarrangementer
\end{itemize}

Små lokale klubber kan også have vask af trøjer og andet udstyr gående på skift blandt medlemmerne, hvor forældre hurtigt vil kunne se hvis tur det er via et administrations system. 

\subsection{Badminton- og tennis-klub}

En badmintonklub har ligesom ved fodbold, nogle baner som klubben har til rådighed. Dog skiller badminton sig ud ved, at det ofte foregår i lokale sportshaller, hvor andre sportsklubber også har bane, f.eks. håndbold. Tennis er her meget ens, den største forskel værende, at de typisk har deres egne baner samt ofte foregår på udendørs baner. Disse baner kan typisk lejes, hvilket også skal håndteres. Administrationen indebærer blandt andet følgende opgaver:

\begin{itemize}
\item Håndtering af medlemmer
\item Udskrivning af regninger
\item Arrangering af sportsarrangementer
\item Udlejning af baner
\end{itemize}

\subsection{Skytteklub}

I en skytteklub kan der foregå mange forskellige typer skydning og alle aldersgrupper kan være med. Der kan forekomme optil flere administrative opgaver, som der i mindre klubber kan forekomme at gøres manuelt. Sådanne administrative opgaver kan være følgende:

\begin{itemize}
\item Administration af skydebaner og evt. reservation
\item Medlemshåndtering inkl. udskrivning af regninger
\item Kørsel til og fra stævner og andre arrangementer
\item Håndtering af våbenlicenser internt i foreningen
\end{itemize}

I nogle skytteklubber kan man også få sit våben opbevaret i deres våbenskab, sammen med skytteforeningens egne våben. Da våbenene godt kan blive blandet sammen, så kan det være svært at holde styr på, hvem der ejer hvilket våben.


\subsection{Golf}

Golf skiller sig ud fra de tidligere nævnte fritidsklubber ved at golf foregår på meget store græsarealer. Ligeledes er der på de græsarealer et bestemt antal huller, og klubben kan have flere baner af ofte 9 eller 18 huller. 

Da der er flere grupper af spillere på samme bane samtidigt, skal klubben i stedet holde styr på hvem der starter hvornår på 1. hul. Herudover, kan det være centralt for klubben at have en måde at registrere om gruppen vil have golfbiler med, og endda caddier, hvis det er noget golfklubben også tilbyder.

Det vil sige, at der i stedet for ressourceplanlægning, er mere tale om skemalægning. Altså hvem der spiller hvornår, og med tilvalg såsom golfbiler eller caddier.


\subsection{Haller}

Sportshaller lægger lokale til mange forskellige sportsklubber. Fodbold bruger omklædningen og de udendørsbaner til træning, badminton bruger de indendørs baner som de skal dele med f.eks. håndbold, indendørs fodbold, indendørs hockey og mange flere. Sportshaller har også ofte lokale arrangementer, f.eks. foredrag, ungdomsklub osv. 
Haller har også ofte en kiosk eller cafe, hvor sportsudøvere eller arrangementsgæster kan komme og få noget at spise og drikke. 
Haller kan også have et motionscenter, hvor idrætsklubber og almindelige personer kan købe adgang\citep{spt_hal}.
Eksempler på administrative opgaver kunne være:

\begin{itemize}
\item Administration af baner og omklædning.
\item Administration af andre typer arrangementer.
\item Information fra kiosken.
\item Informationsdeling til sportsklubber og andre interesserede.
\end{itemize}

\subsection{Sejlklub}

En sejlklub kan, alt efter størrelsen, have flere administrative poster. Disse poster kunne bl.a. bestå af administration af deres både, hvem har reserveret en båd, hvornår er båden ledig osv. Nogle sejlklubber har også sejlundervisning, hvor der bl.a. skal kunne holdes styr på elevers niveau. Andre eksempler på administrative opgaver i en sejlklub kunne være: 

\begin{itemize}
\item Af- og tilmelding af undervisning
\item Reservering og udlån af både
\item Oversigt over hvem der skylder penge for lån af båd
\item Medlemshåndtering
\item Medlemsbetaling
\end{itemize}

\subsection{Ungdomsklub}

En ungdomsklub ses som en sted, hvor unge kan bruge deres fritid på sociale aktiviteter. Det er meget forskelligt hvordan ungdomsklubber drives, nogle drives af frivillige hvor andre drives af skolevæsnet, f.eks. ungdomsskoler. Ofte holde ungdomsklubber til i skolebygninger. Om de drives af frivillige eller skolevæsnet, så har de stadig nogle adminimistrative opgaver som skal løses. Eksempler på sådanne administrative opgaver kunne være:

\begin{itemize}
\item Arrantementsplanlægning
\item Bestilling af varer (hvis der sælges sodavand, slik o.l.)
\item Medlemsbetaling
\end{itemize}

\section{Afgrænsning}

Ud fra de fundne fritidsklubber, der kan man se, at der er nogle træk som går igen. Dette er bl.a. medlemshåndtering og medlemsbetaling, som stort set er en generel administrativ opgve i alle ovenstående fritidsklubber. Nogle klubber har også specielle administrative poster, som f.eks. skytteklubber, der har med våbenhåndtering af gøre.

Ud fra ovenstående fritidsklubber afgrænses der til at udelukkende beskæftiges med sejlklubber. Denne afgrænsning finder sted, da det vurderes, at der er et potentiale til at gøre arbejdet i en sejlklub nemmere og ligeledes findes sejlklubber specielle, da nogle sejlklubber har undervisningsfunktion, og derfor har specielle administrative funktioner. 

