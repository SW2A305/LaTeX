\chapter{Fritidsklubber} \label{chap:Fritidsklubber}

I følgende afsnit vil der blive kigget nærmere på forskellige fritidsklubber, for at afdække de administrative poster
som kunne forekomme. Afsnittet er skrevet ud fra gruppens egne erfaringer, hvilket gøres da det ikke har været muligt, at finde eksterne kilder omhandlende administrationen af de respektive typer af fritidsklubber. 
Det er taget i betragtning, at disse erfaringer
kan være specifikke for den enkelte klub hvor erfaringen stammer fra. 
Dog vurderes der at de forekommende administrative opgaver, er generelle opgaver, og altså ikke enkelttilfælde. 
Fritidsklubber dækker over klubber, hvor unge og voksne kan bruge deres fritid, om det er socialt, sportslig eller begge. 
Fritidsklubberne er valgt ud fra det faktum, at de ofte er drevet af frivilligt arbejde og har faciliteter der udlejes. 
Denne type frivilligt arbejde, er meget individuel fra klub til klub, om det er undervisning, håndtering af det økonomiske eller noget helt andet.

Dog er der nogle administrative opgaver, som går igen blandt fritidsklubberne. Det er primært opgaver, som omhandler medlemshåndtering, da klubber har en medlemsbase, som skal administreres. De generelle administrative opgaver omkring medlemshåndtering kan være følgende:
\begin{itemize}
	\item Generel medlemshåndtering
	\item Medlemsbetaling
\end{itemize}

%\fxnote{Kan være der her skal inkluderes en liste af mere generelle ting?(Medlemskontigent, undervisning osv. og så
% fjerne det fra de enkelte afsnit, på den måde undgås at dette blot skrives på variende måder}
%\subsection{Sejlklub}\label{subsec:sejlklub}

\section{Sportsklubber} \label{Sportsklubber}

I sportsklubber kan der forekomme mange forskellige administrative opgaver. Det er forskelligt fra sportsklub til sportsklub, hvilke opgaver der er tale om, men der er nogle fællestræk, som går igen blandt klubberne. Disse fællestræk er bl.a.:
\begin{itemize}
	\item Udlejning af materiel, f.eks. baner, sportsudstyr o.l.
	\item Arrangering og pointgivning ved sportsarrangementer
	\item Arrangering af kørsel til og fra arrangementer
\end{itemize}

Sportsklubber kan også have specielle forhold, som spiller ind, f.eks. ved en skytteklub gælder der specielle regler ved våbenlicenser og våbentransport. Golfklubber har også specielle forhold omkring deres baner da de, rent arealmæssigt, er meget større end almindelige sportsbaner. 


%\subsection{Fodboldklub} \label{Fodbold}

%I en fodboldklub er der mange administrative og organisatoriske opgaver, som kunne have gavn af et elektronisk
%system til håndtering af disse opgaver frem for et manuelt system. Eksempler på sådanne administrative opgaver
%kan være:

%\begin{itemize}
%  \item Holde styr på baner, hvilket hold som spiller/træner hvor og hvornår
%  \item Kørsel til og fra arrangementer og lignende
%  \item Pointgivning ved lokale sportsarrangementer
%\end{itemize}

%Små lokale klubber kan også have vask af trøjer og andet udstyr gående på skift blandt medlemmerne, hvor
%forældrene hurtigt vil kunne se hvis tur det er via et administrationssystem.\citep{fodbold1}\citep{fodbold2}


%\subsection{Badminton-- og tennisklub}

%En badmintonklub har ligesom ved fodbold, nogle baner som klubben har til rådighed. Dog skiller badminton sig
%ud ved, at det ofte foregår i lokale sportshaller, hvor andre sportsklubber også har bane, f.eks. håndbold.
%Tennis er her meget ens med badminton, den største forskel værende, at de typisk har deres egne baner samt,
%det ofte foregår udendørs. Disse baner kan typisk lejes, hvilket også skal håndteres. Administrationen
%indebærer blandt andet følgende opgaver:

%\begin{itemize}
%  \item Arrangering af sportsarrangementer
%  \item Udlejning af baner
%\end{itemize}


%\subsection{Skytteklub}

%I en skytteklub kan der foregå mange forskellige typer skydning. Der kan forekomme op til flere administrative
%opgaver, som der i mindre klubber udføres manuelt. Sådanne administrative opgaver kan være følgende:

%\begin{itemize}
%  \item Administration af skydebaner og evt. reservation
%  \item Kørsel til og fra stævner og andre arrangementer
%  \item Håndtering af våbenlicenser internt i foreningen
%\end{itemize}


%\subsection{Golf}

%Golf skiller sig ud fra de tidligere nævnte fritidsklubber ved, at golf foregår på et meget større område.
%Ligeledes er der på de græsarealer et bestemt antal huller, og klubben kan have flere baner af ofte 9 eller 18
%huller.

%Da der er flere grupper af spillere på samme bane samtidigt, skal klubben i stedet holde styr på hvem der
%starter hvornår på 1. hul. Herudover, kan det være centralt for klubben at have en måde at
%registrere om gruppen vil have golfbiler med, og endda caddier, hvis det er noget golfklubben også tilbyder.

%Det vil sige, at der i stedet for ressourceplanlægning, er mere tale om skemalægning. Altså hvem der spiller
%hvornår, og med tilvalg såsom golfbiler eller caddier.


\section{Haller}

Sportshaller lægger lokale til mange forskellige sportsklubber. Fodbold bruger omklædningen og de
udendørs baner til træning, badminton bruger de indendørs baner, som de skal dele med f.eks. håndbold,
indendørs fodbold, indendørs hockey og mange flere. Sportshaller har også ofte lokale arrangementer, f.eks.
foredrag, ungdomsklub osv. Haller har også ofte en kiosk eller café, hvor sportsudøvere eller
arrangementsgæster kan komme og få noget at spise og drikke. Haller kan også have et motionscenter, hvor
idrætsklubber og almindelige personer kan købe adgang\citep{spt_hal}. Eksempler på administrative opgaver
kunne være:

\begin{itemize}
  \item Administration af haludlejning og omklædning 
  \item Information fra kiosken
  \item Informationsdeling til sportsklubber og andre interesserede
\end{itemize}


\section{Ungdomsklub}

En ungdomsklub ses som en sted, hvor unge kan bruge deres fritid på sociale aktiviteter. Det er meget
forskelligt, hvordan ungdomsklubber drives. Nogle drives af frivillige, hvor andre drives af skolevæsnet\citep{ung1},
f.eks. ungdomsskoler. Ofte holder ungdomsklubber til i skolebygninger. Hvad end de drives af frivillige eller
skolevæsnet, har de stadig nogle administrative opgaver som skal løses. Eksempler på sådanne administrative
opgaver kunne være:

\begin{itemize}
  \item Arrangementsplanlægning
  \item Bestilling af varer (hvis der sælges sodavand, slik o.l.)
\end{itemize}

\section{Sejlklub}

En sejlklub kan, alt efter størrelsen, have flere administrative opgaver. Disse opgaver kunne bl.a. bestå af
administration af deres både, hvem har reserveret en båd, hvornår er båden ledig osv. Nogle sejlklubber har
også sejlundervisning, hvor der bl.a. skal kunne holdes styr på elevers niveau. Andre eksempler på
administrative opgaver i en sejlklub kunne være:

\begin{itemize}
  \item Af- og tilmelding af undervisning
  \item Reservering og udlån af både
  \item Oversigt over hvem der skylder penge for lån af båd
\end{itemize}

\section{Afgrænsning}

Ud fra de fundne fritidsklubber kan man se, at der er nogle træk som går igen. Dette er bl.a. medlemshåndtering og
medlemsbetaling, som stort set er en generel administrativ opgave i alle ovenstående fritidsklubber. Udlån/udlejning af materiel er også en generelt administrativ opgave som går igen.
Nogle klubber har
også specielle administrative poster, som f.eks. skytteklubber, der har med våbenhåndtering af gøre.

Ud fra ovenstående fritidsklubber afgrænses der til udelukkende at beskæftiges med sejlklubber. Denne afgrænsning finder
sted, da det vurderes, at der er et potentiale til at gøre arbejdet i en sejlklub nemmere og ligeledes findes
sejlklubber specielle, da nogle sejlklubber har en undervisningsfunktion, og derfor har specielle administrative
funktioner.
