\chapter{Fritidsklubber} \label{chap:Fritidsklubber}

I følgende afsnit vil forskellige fritidsklubber blive undersøgt, for at afdække de respektive administrative poster, som kunne forekomme i sådanne klubber. 
Afsnittet er skrevet ud fra gruppens egne erfaringer, hvilket gøres, da det ikke har været muligt, at finde eksterne kilder omhandlende administrationen af de respektive typer af fritidsklubber.
Det er taget i betragtning, at disse erfaringer kan være specifikke for den enkelte klub, hvor erfaringen stammer fra. 
Dog vurderes det, at de forekommende administrative opgaver er generelle opgaver, og altså ikke enkelttilfælde, og derfor godt kan anvendes i denne sammenhæng.

Begrebet fritidsklubber dækker over klubber, hvor voksne og unge kan bruge deres fritid, hvad end det er socialt, sportsligt eller begge. 
Fritidsklubberne er valgt ud fra det faktum, at de ofte er drevet af frivilligt arbejde og har faciliteter der udlejes. 
Denne type frivilligt arbejde er individuel fra klub til klub, hvad end det er undervisning, håndtering af det økonomiske eller noget helt tredje. 
Dog er der nogle administrative opgaver, som går igen blandt fritidsklubberne.
Det er primært opgaver som omhandler administration af medlemmer, og deres gøremål i klubsammenhæng.
De generelle administrative opgaver omkring medlemshåndtering kan være følgende:
\begin{itemize}
	\item Kontingentbetaling
	\item Undervisning/træning
	\item Turneringsbetaling
\end{itemize}

\section{Sportsklubber} \label{Sportsklubber}

I sportsklubber kan der forekomme mange forskellige administrative opgaver. 
Det er forskelligt fra sportsklub til sportsklub, hvilke opgaver der er tale om, men der er nogle fællestræk, som går igen blandt klubberne. 
Disse fællestræk er bl.a.:
\begin{itemize}
	\item Udlejning/udlån af aktiver, f.eks. baner og sportsudstyr. 
	\item Arrangering og pointtildeling ved sportsarrangementer.
	\item Arrangering af kørsel til og fra arrangementer.
	\item Kommunikation til forældre og sportsudøvere.
\end{itemize}

Sportsklubber kan også have specielle forhold, som har indflydelse på, hvilke administrative opgaver klubben har, grundet de forskellige sportsgrene de udbyder, f.eks. ved en skytteklub gælder der specielle regler for våbenlicenser og våbentransport.

\section{Sportscentre}

Sportscentre lægger lokale til mange forskellige sportsklubber.
Fodboldklubber bruger omklædningen og de udendørs baner til træning. 
Badmintonklubber bruger de indendørs baner, som de skal dele med andre aktiviteter f.eks. håndbold, indendørs fodbold, hockey osv.
Sportscentre har også ofte lokale arrangementer, f.eks. foredrag, ungdomsklub osv.
Sportscentre kan også have et motionscenter, hvor idrætsklubber og almindelige personer kan købe adgang \citep{spt_hal}. 
Eksempler på administrative opgaver kunne være:

\begin{itemize}
  \item Administration af udlejning og omklædning.
  \item Information fra kiosken.
  \item Informationsdeling til sportsklubber og andre interesserede.
\end{itemize}


\section{Ungdomsklub}

En ungdomsklub ses som et sted, hvor unge kan bruge deres fritid på sociale aktiviteter.
Det varierer, hvordan ungdomsklubber drives. 
Nogle drives af frivillige, hvor andre drives af skolevæsnet, f.eks. ungdomsskoler\citep{ung1}.
Ofte holder ungdomsklubber til i skolebygninger. 
Hvad end de drives af frivillige eller skolevæsnet, har de stadig nogle administrative opgaver, som skal løses. 
Eksempler på sådanne administrative opgaver kunne være:

\begin{itemize}
  \item Arrangementsplanlægning.
  \item Bestilling af varer (hvis der sælges sodavand, slik o.l.).
\end{itemize}


\section{Sejlklub}

En sejlklub kan have flere administrative opgaver. 
Disse opgaver kan bl.a. bestå af administration af klubbens både, herunder hvem der har reserveret en båd, hvornår båden er ledig osv.
Nogle sejlklubber har også sejlundervisning, hvor der bl.a. skal kunne holdes styr på hvilke lektioner, eleverne har haft. 
Andre eksempler på administrative opgaver i en sejlklub kunne være:

\begin{itemize}
  \item Af- og tilmelding af undervisning.
  \item Reservering og udlån af både.
  \item Oversigt over hvem der skylder penge for lån af både.
\end{itemize}

\section{Afgrænsning}

Ud fra ovenstående fritidsklubber afgrænses der til fremover udelukkende at beskæftige sig med sejlklubber. 
Denne afgrænsning finder sted, da det vurderes, at der er potentiale for at gøre arbejdet i en sejlklub nemmere, grundet de administrative udfordringer.
Ligeledes findes sejlklubber specielle, da nogle sejlklubber har en undervisningsfunktion, og derfor har specielle administrative funktioner.
