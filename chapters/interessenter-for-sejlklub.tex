\chapter{Interessenter for en sejlklub}\label{chap:interessent-analyse-ved-sejlklubber}

I denne interessentanalyse ses der på mennesker og grupper, der har interesse i en specifik softwareløsning,
konkret til sejlklubber med sejlerskole. Der vil herudover blive set på, hvad deres interesse er i projektet,
samt hvilken indflydelse en løsning ville kunne have på deres tilværelse.

Der vil for interessenternes vedkommende blive set på, hvad de hver især kunne få gavn af i forhold til
projektet. Det er værd at notere, at hver enkelt klub har forskellige behov, så dette er en generel forståelse
af interessenternes behov. Forståelsen for de forskellige grupper af interessenter er dannet ud fra et
interview med den tidligere skolechef fra sejlklubben Sundet. Interviewet kan findes i
\myref{bilag:interview}.


\section{Interessenter for sejlklubbers administrative opgaver}


\subsection{Dansk Sejlunion}

Dansk Sejlunion er et forbund, som blev dannet i 1913, og har ikke nogen direkte grund til at være
interesseret i projektet. \fxnote{Jeg ved ikke om der skal skrives at de ikke har direkte interesse i
projektet, man kan nemt sidde og tænke: ``Hvorfor er de så med i analysen?'' --Thomas; Burde måske omformuleres
til at de ikke direkte kommer til at bruge vores løsning -- Caspar} Et af deres mål er, som forbund, at hjælpe
sejlklubber med service, rådgivning mm. og derfor menes det, at Dansk Sejlunion også vil være interesseret i
et system, der vil kunne hjælpe de frivillige i deres arbejde.

Deres mission er at være det nationale samlingspunkt for alle sejlere. Dansk Sejlunion er tilsluttet Danmarks
Idrætsforbund, International Sailing Federation og andre lignende organisationer inden for sejlsport.
\citep{Sejlsportdk} \fxnote{er de sidste linjer ikke mere nice-to-know end need-to-know?}


\subsection{Medlemmer}

Hver sejlklub bestemmer selv deres opbygning organisatorisk af hvilke type medlemmer de har i klubben.
\fxnote{Lyder: ``Hver sejlklub bestemmer selv deres organisatoriske opbygning af hvilke typer medlemmer...''
bedre?} Der vil blive fortalt mere om dette i \myref{chap:organisation}. \fxnote{måske: Dette vil blive
undersøgt i organisationsafsnittet, i stedet for}

Medlemmerne i sejlklubben er de personer, som bruger sejlklubbens faciliteter. \fxnote{er det ikke lidt en
sjov definition på hvem der er medlem? Burde måske være mere konkret som: Dem som betaler kontingent er
medlemmer} Herudover også her klubben får deres indtægter til at kunne holde sig i gang, og investere i
eventuelt nye både el. \fxnote{Sidste sætning skal rettes men kan ikke gennemskue hvad der rigtigt skal stå?}
Det er derfor vigtigt, at klubberne gør medlemmernes tid ved sejlklubben så god som mulig. Et system der kunne
hjælpe medlemmerne med at finde informationer, leje både og som har medlemmernes interesse i højsædet, vil
være i deres interesse. De førnævnte funktioner i et system til en sejlklub, er netop funktioner, der
sandsynligvis ville blive brugt mest af medlemmer.

Medlemmer i en sejlklub kan være en meget bred gruppe fra den yngre befolkningsgruppe, til den noget ældre.
Det er derfor vigtigt at lave et design, som er intuitivt i anvendelse, for alle brugere.

%Interessentgruppen ``medlemmer'' omhandler medlemmer af sejlklubber. Det er meget forskelligt fra sejlklub til
%sejlklub hvordan medlemmer kategoriseres eller om de overhovedet gøres dette. F.eks. så har ``Sejlklubben
%Sundet'' kategoriseret deres medlemmer således: Voksen-, Bådejer-, Gaste-, Mini-kølbåd-, Ungdoms-, Passiv- og
%Støttemedlem.
%[http://www.sundet.dk/vedtaegter/Vedtaegter%20for%20Sundet%2027nov12.pdf]

%Til sammenligning har Vestre Baadelaug kategoriseret deres medlemmer efter følgende: Aktive-, passive- og
%æresmedlemmer.
%[http://www.aalborglystbaadehavn.dk/UserFiles/file/VB_filer/Statiske_filer/Vedtaegter_2008-9_web.pdf].

%Fælles for medlemmerne er, at de gerne vil have informationer fra klubberne, så de ved hvilke begivenheder der finder
%%sted. Det kan være kapsejladser, foredrag, o.lign. Der er mange informationer, medlemmer fra en sejlklub kan modtage.
%\section{Gæster}:
%Gæster har ikke nogen primær funktion når det kommer til udlån, de bådklubber der tilbyder udlejning af både,
%%forbeholder denne funktion til medlemmer, som har bestået et førerkursus accepteret af den givne bådklub.


\subsection{Undervisere}

En sejlklub med sejlerskole skal have undervisere til sejlerskolen, og disse kan også drage nytte af et system
til at hjælpe med administrative opgaver. Underviserne vil ved hjælp af et system, eventuelt kunne sende
afbuds--e--mails eller SMS--beskeder ud til alle på undervisningsholdet. Det vil muligvis kunne gøre det lettere
for underviseren at holde styr på den enkelte elevs fremskridt ved undervisningen mm., hvilket medfører at
underviserne bør have andre funktioner i et system til en sejlklub end medlemmerne ville have. Underviserne
ville eventuelt kunne stå for at registrere hvem der har et førerbevis i klubben, ol. Underviserne kan også
være frivillige i sejlklubben, men de er beskrevet for sig selv, da de har mere specifikke opgaver i
sejlklubben. Det antages, at hvis et system ville kunne hjælpe undervisere med deres opgaver, så de bruger
mere af deres arbejdstid på at undervise, ville det give en positiv respons.


\subsection{De frivillige i sejlklubben}

Der er, udover undervisere i en sejlklub, også andre frivillige der arbejder i sejlklubben. Dette kan være
lige fra en formand, til en sekretær eller måske endda nogen som gør rent i sejlklubben. Disse frivillige kan
alle have forskellig gavn af et system, der kan håndtere deres anliggender i sejlklubben, og er derfor taget
med som interessenter. Nogle af dem kan have til opgave at holde styr på ind- og udbetalinger for medlemmerne,
hvilket et medlemssystem ville kunne hjælpe med. En sekretær ville kunne give nyheder eller referater videre
til klubbens medlemmer igennem sådan et system. Så der er mange forskellige muligheder et administrativt
system ved en sejlklub ville kunne hjælpe med, og herudover også mange mennesker der ville kunne få gavn af en
løsning.


\subsection{Interessentkonklusion?} \fxnote{Ved ikke om det er den rette titel, men der skal i hvert fald være
en adskillelse af ovenstående og nedenstående afsnit, da det er to forskellige dele.}

Disse 4 interessenter, har hver deres interesse for projektet, hvor underviserne og de andre frivillige
smelter en smule sammen. Underviserne, samt de andre frivillige, vil gerne have det gjort lettere at udføre
deres frivillige arbejde. Medlemmerne vil gerne have informationer fra klubberne angående begivenheder, der
måtte foregå, samt muligheder for at deltage i disse, eller leje sejlbåde. Dansk Sejlunion er en mindre
direkte interessent, men kan dog have interesse i at hjælpe nye eller mindre sejlklubber med at opsætte et
godt administrativt netværk, for at danne en velfungerende sejlklub, der giver gode oplevelser for
medlemmerne.

