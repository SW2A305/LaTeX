\chapter{Interessenter for sejlklub}

I denne interessentanalyse, ses der på mennesker og grupper der har interesse i en specifik software løsning, konkret til sejlklubber med sejlerskole.


I dette afsnit vil der blive set på diverse interessenter med relevans for sejlklubber. Der vil for interessenternes
vedkommende blive set på hvad, de hver især kunne få gavn af i henhold til projektet. Det er værd at notere at hver
enkelte klub har forskellige behov, men der tages udgangspunkt i en sejlklub. Denne klub vil blive sammenlignet organisatorisk med to andre klubber, og herfra dannes der en konklusion, på sejlklubber som organisation.

\section{Interessenter for sejlklubbers administrative opgaver}

\subsection{Dansk sejlunion}

Dansk sejlunion er et forbund, som blev dannet i 1913.
%[http://www.sejlsport.dk/mere/dansk-sejlunion/historie] 

Deres mission er at være det nationale samlingspunkt for alle sejlere. Dansk sejlunion er tilsluttet Danmarks
Idrætsforbund, International Sailing Federation og andre lignende organisationer inden for sejlsport. Dansk
sejlunion tilbyder også services i form af rådgivning og aktiviteter til klubber, sejlklasser og andre
samarbejdspartnere.
%[http://www.sejlsport.dk/mere/dansk-sejlunion/strategi-og-politik/vision-og-vaerdier]


%\section{Andre fritidsklubber}
%
%I denne sammenhæng er ``Andre fritidsklubber'' fritidsklubber som har samme lignende funktion som en sejlklub.
%Der kan drages paralleller til en skytteforening; en skytteforening har skydebaner som kan administreres og
%hvis man ikke selv medbringer våben, så kan man ofte også låne et, hvor der ligeledes kan ske adminstration. I
%skydning er konkurrencerne ofte delt op efter typen af skydning, alder og skydestil \fxnote{andet ord?}, hvilket
%medfører at medlemmerne med fordel kunne skrives ind i et system som automatisk kan holde styr på dem og se
%hvilke konkurrencer de kvalificerer sig til.


\subsection{Medlemmer}

Interessentgruppen ``medlemmer'' omhandler medlemmer af sejlklubber. Det er meget forskelligt fra sejlklub til
sejlklub hvordan medlemmer kategoriseres eller om de overhovedet gøres dette. F.eks. så har ``Sejlklubben
Sundet'' kategoriseret deres medlemmer således: Voksen-, Bådejer-, Gaste-, Mini-kølbåd-, Ungdoms-, Passiv- og
Støttemedlem.
%[http://www.sundet.dk/vedtaegter/Vedtaegter%20for%20Sundet%2027nov12.pdf]

Til sammenligning har Vestre Baadelaug kategoriseret deres medlemmer efter følgende: Aktive-, passive- og
æresmedlemmer.
%[http://www.aalborglystbaadehavn.dk/UserFiles/file/VB_filer/Statiske_filer/Vedtaegter_2008-9_web.pdf].

Fælles for medlemmerne er, at de gerne vil have informationer fra klubberne, så de ved hvilke begivenheder der finder sted. Det kan være kapsejladser, foredrag, o.lign. Der er mange informationer, medlemmer fra en sejlklub kan modtage.
%\section{Gæster}:
%Gæster har ikke nogen primær funktion når det kommer til udlån, de bådklubber der tilbyder udlejning af både, forbeholder denne funktion til medlemmer, som har bestået et førerkursus accepteret af den givne bådklub.

%Forældre(irrelevant)?:


\subsection{Undervisere}

Undervisererne udnytter sejlerklubbens både i faste tidsrum til undervisning. Deres funktion i
klubben gør at der er specifikke tidspunkter hvorpå bådene ikke kan udlejes. Underviserne kunne benyttes til
det formål, at gøre udlejning af både lettere. Undervisere ville løbende kunne opdatere status for hvilke
medlemmer der er kvalificeret til at føre en båd, og derved hjælpe med det administrative aspekt af udlejning
af både.

Disse tre interessenter, har hver deres interesse for projektet. Underviserne på sejlskolerne, vil gerne have det gjort lettere at udføre deres frivillige arbejde. Medlemmerne vil gerne have informationer fra klubberne ang. begivenheder der måtte foregå, og Dansk Sejlunion kan have interesse i at hjælpe nye eller mindre sejlklubber med at istandsætte et godt administrativt netværk.