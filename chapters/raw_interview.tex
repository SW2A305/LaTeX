\section{Interview}
Jacob: I skal huske at det her baseret på muligvis forældet viden.
Alle: Jaja.

Jacob: Hvad mener i med at adminstrere i sejlklubben? Det er jo et bredt spørgsmål.

Marc: Det er primært hvordan i holder styr på de ting som ligger under sig.

Jacob: Vi har et hjemmestrikket medlemsystem. Det var jo den gang ikk'. Der ligger alt det sædvanlige ikk': Navn, adresse, telefonnummer. Så ligger der fuldt medlemsnummer og så ligger der ens uddannelse, altså sejlskole, hvor langt man er kommet i forløbet på sejlskolen og om man har førebrevet, duelighedsbevis. Så ligger der om man er bådeejer, det koster nemlig flere penge. Der ligger vidst også noget om landpladsadminstration, alt hvad der hedder vand, plads i vandet, der organiserer havnen. Plads på land det arrangerer klubben. De klubber har forskellige landområder vi disponerer over. Det er så det generelle, det vi har på medlemmerne.
Så kan man, f.eks. hvis man som medlem har været... Hvis man skal betale for et eller andet, hvis man har lånt en båd og skal betale for det, så man kan også et skærmbillede til det. Jeg kan ikke lige huske om man klikker direkte fra medlem eller man skal indføre medlemsnummeret der, men der er et sted hvor man kan gå ind og sige: Her er en der skal betale noget. Han skal betale for så og så mange dages leje af en båd og en motor osv. osv. og så printer vi en girokort ud. Det samme typisk skærmbillede bruger vi til en som skal betale for en landplads og hans båd fylder så og så mange kvadratmeter *døk?* Så kan man skrive her ind, som jeg gjorde i weekenden, lånt en af klubbens slibemaskiner med støvsuger og det koster så og så meget pr. dag, så meget af det ligger i et *pr. access?*-baseret tussegammelt hjemmestrikket system som ligger på computeren nede i klubben.

Søren: Ligger det kun lokal eller hvadt?

Jacob: Ja, det ligger kun der lokalt og der er forskellige record udtræksningsmuligheder os' ikk: Gi' mig en liste over alle elever på andet år, gi' mig en liste over alle bådejere eller gi' mig en liste over hvilken rækkefølgen både skal sættes op i og sådan noget, så det er der også. SÅ har vi selvfølgelig senere, men det kan i se på nettet, så har vi jo indført forskellige... Altså meget af det information som vi kan trække det bliver så på forskellig vis smidt op på nettet. Det jeg sagde lige før med liste over hvornår både skal op den ligger på nettet, og den kan alle og enhver gå ind og se, så i kan gå ind og se hvad min båd hedder og hvornår den skal i vandet. Det ligger bare som en PDF. 
Nå, hvad har i på jeres medlemmer? Der er sagt, altså vi har typisk om de har eller ikke har førebrevet, om hvor langt de er kommet i skolen, hvis vi altså husker at vedligeholde det, det er ikke altid vi gør det. Det er vel sådan cirka det. 

Hvor mange både har vi til rådighed?

3:32

