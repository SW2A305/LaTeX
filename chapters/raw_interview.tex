\chapter{Interview transskribering}\label{bilag:interview-transkribering}
Jacob: I skal huske at det her baseret på muligvis forældet viden.
Alle: Jaja.

Jacob: Hvad mener i med at adminstrere i sejlklubben? Det er jo et bredt spørgsmål.

Marc: Det er primært hvordan i holder styr på de ting som ligger under sig.

Jacob: Vi har et hjemmestrikket medlemsystem. Det var jo den gang ikk'. Der ligger alt det sædvanlige ikk': Navn, adresse, telefonnummer. Så ligger der fuldt medlemsnummer og så ligger der ens uddannelse, altså sejlskole, hvor langt man er kommet i forløbet på sejlskolen og om man har førebrevet, duelighedsbevis. Så ligger der om man er bådeejer, det koster nemlig flere penge. Der ligger vidst også noget om landpladsadminstration, alt hvad der hedder vand, plads i vandet, der organiserer havnen. Plads på land det arrangerer klubben. De klubber har forskellige landområder vi disponerer over. Det er så det generelle, det vi har på medlemmerne.
Så kan man, f.eks. hvis man som medlem har været... Hvis man skal betale for et eller andet, hvis man har lånt en båd og skal betale for det, så man kan også et skærmbillede til det. Jeg kan ikke lige huske om man klikker direkte fra medlem eller man skal indføre medlemsnummeret der, men der er et sted hvor man kan gå ind og sige: Her er en der skal betale noget. Han skal betale for så og så mange dages leje af en båd og en motor osv. osv. og så printer vi en girokort ud. Det samme typisk skærmbillede bruger vi til en som skal betale for en landplads og hans båd fylder så og så mange kvadratmeter *døk?* Så kan man skrive her ind, som jeg gjorde i weekenden, lånt en af klubbens slibemaskiner med støvsuger og det koster så og så meget pr. dag, så meget af det ligger i et *pr. access?*-baseret tussegammelt hjemmestrikket system som ligger på computeren nede i klubben.

Søren: Ligger det kun lokal eller hvadt?

Jacob: Ja, det ligger kun der lokalt og der er forskellige record udtræksningsmuligheder os' ikk: Gi' mig en liste over alle elever på andet år, gi' mig en liste over alle bådejere eller gi' mig en liste over hvilken rækkefølgen både skal sættes op i og sådan noget, så det er der også. SÅ har vi selvfølgelig senere, men det kan i se på nettet, så har vi jo indført forskellige... Altså meget af det information som vi kan trække det bliver så på forskellig vis smidt op på nettet. Det jeg sagde lige før med liste over hvornår både skal op den ligger på nettet, og den kan alle og enhver gå ind og se, så i kan gå ind og se hvad min båd hedder og hvornår den skal i vandet. Det ligger bare som en PDF. 
Nå, hvad har i på jeres medlemmer? Der er sagt, altså vi har typisk om de har eller ikke har førebrevet, om hvor langt de er kommet i skolen, hvis vi altså husker at vedligeholde det, det er ikke altid vi gør det. Det er vel sådan cirka det. 

Hvor mange både har vi til rådighed? 

Ja det svinger meget. Den gang jeg var skolechef der havde vi 2 gaffelrigger, 3 drabanter og en spækhugger. Nu er konfigurationen ved at ændre sig lidt så nu hedder det stadig to gaffelriggere og så hedder det vidst nok to eller tre spækhuggere, er de ved at lave det om til og så nogle nye som hedder J80, sådan nogle satans små flyvepap. Jeg tror de unge mennesker kommer til at få nogle forskrækkelser når de skal ud og sejle dem første gang hvis de ikke har sejlet før, men det er jo deres problem. Det er jo en politisk diskussion. Og så har vi en enkelt yngling. Det var en som blev til overs engang, den er bare blevet foræret til klubben. Den er til at læren til at lege i og så har faktisk en fire... men de er ikke rigtig taget med der, vi har en fire til seks Mini12'er, hedder de. Det er til handicapafdelingen. De ligner gammeldags Americas Cup både som de så ud den gang de lignede sejlbåde, skaleret ned i en to-tre meters længde, så kan man sidde nede i dem og styre med hænderne eller fødderne eller hvad man har at bruge, så den bruger handicapafdelingen til når handicappede er ude og sejle. 

Et EDB-system?

Ja, i den forstand at vi har overblik over hvilke både vi har. Så er der kommet det der online reservationssystem, men den situation jeg kendte, som jeg tror stadig ligger bagved, fordi der kan man kun reservere en båd. Det er jo at der inde i skolestuen, dvs. det er der hvor der ligger redningsveste og grej til alle bådene og sådan der ligger der en stor bog hvor man går ind og skriver "3. maj, sådan og sådan, Jacob, besætning, telefonnummer på Jacob, hvornår sejler vi, hvornår tror vi at vi kommer hjem, hvor tror vi nok vi tager hen og så et eller andet sted, en kolonne som hedder: det var her vi kom hjem og hvis der er nogen kommentar, så det er sådan en stor bog af den tykke der, den fylder sådan der *viser hvor står bogen er*. 
Så er der på hver båd en havariprotokol, dvs. at hvis der er et eller andet med båden, så tager man den rigtige bog frem og skriver: Jeg var ude og sejle med båden og jeg smadrede ind i Oslobåden eller et eller andet, der var noget som knækkede, sejlede revnede eller hvad det nu var og hvis man ikke kan reparere det selv så skriver man det der. Hvis man føler eller ved at det er noget som bare skal repareres i en helveds fart, så er man stærkt opfodret til at ringe til skolechefen eller den der har ansvar for båden. Hver båd har en bådchef der er ansvarlig på den båd, så hvis man kan se at det her kræver mere end jeg lige kan klare, det kræver noget værktøj, reservedele, det kræver måske syn af en bådbygger, så skal man så ringe til den pågældende person som så vil sætte det i værk. Hvis man kan reparere det selv så prøver man også på at gøre det, men bare lige skriv et notat om at der var noget eller hvis det er noget som skal laves men det kan godt sejle uden det laves så skriv det lige. 

Så er der skolen: Hver båd hver aften har sådan et fortrykt ark: Fører, elever, sejladsnummer, dato, afkrydsning, hvem og hvad. Ud fra hver, mulighed for bemærkning: Skete der et eller andet, hvad gjorde vi, er der nogen som ikke dukker op, er der gæster med så bliver de også bare skrevet på der nedenunder den faste liste, og hvis det er en anden fører skriver man også det på, så krydser man af *noget noget noget 7:49* og det bruget man tildels til at holde øje med hvem der kommer, hvem der ikke kommer, hvem som husker at melde afbud, og glemmer det, det er fyfy at glemme at melde afbud, og tælle op antal gange man har sejlet, det er jo også vigtigt, altså i forhold til *pro-et-eller andet 8:07*, det var sejladsprotokollen til skolen. 

Så har vi selvfølgelig reservationssystemet, som flytter lidt der, men der er stadig sådan et andet slags reservationssystem som handler om når skolen arrangerer noget. Når skolen siger nu på første weekend i juni der er der 24-timers sejlads, det er en skide god oplevelse. Er der nogen elever der gerne vil være med så hænger der gerne sådan et stykke papir nede i gangen hvor der står 24-timer sejlads, her kan førerne skrive sig på og her kan eleverne skrive sig på og så må man håbe at der er nogen som skriver sig på. Så det er sådan en: Her er et tilbud om en ekstra tur af en eller anden slags, skriv jer lige på her. Det foregår også på papir. Så det kræver altid at man kommer der ned og gør noget, skrive sig på, sørge for overblik. Så det der *pullik 9:11* der hedder "Hvad er jeres største problem", det største problem jeg oplevede var simpelthen det her med at holde styr på... Som adminstrativ ansvarlig, så er der bare en fandens masse papir som man skal holde styr på. Man skal ind og tælle op på den der mødeliste, man skal... Den gang skulle man hvis nogen ringede ned og sagde kunne jeg reservere en båd på det og det tidspunkt, så skulle man ud og finde... Vi havde gerne sådan en kalender hængende nede på opslagstavlen med bådene og datoerne, så kunne man så blive krydset af, enten gjorde folk det selv eller så bad de os om at gøre det. Havde anden kalender bl.a. sedel som hang der med reservation til onsdags-kap-sejlads, sådan at folk "Jeg vil gerne ud og sejle onsdags-kap-sejladsen i næste uge" så skriver man sig lige på, "jeg har reserveret båden" og den der første når frem med en kuglepen har vundet og enten skriver man "jeg har besætning" eller "jeg har plads til to elever" eller hvad det nu er og så må folk finde ud af det på den måde. Det kræver igen at man kommer ned og kigger på opslagstavlen og forstår det der. Der var en regel omkring *onsdags 10:18* med at man kun kan reservere en ude ud i forvejen. Det der med at man bare lige, nogen af de gamle, der er altid nogen som sejler *onsdags 10:25* rundt i vores både de tager blokreservation hen over hele sommeren, nænæ. Man kan reservere en ude i forvejen. Det er jo en politik, som man har. Det er meget papirarbejde. Når man skal finde ud af hvor meget der skal betales, så skal man ind i den store og så skal man ind og se hvem har egentlig lejet en båd og så gå igennem, og så skal man jo prøve på... Typisk så gør man det måske tre-fire gange i løbet af en sommer, maks. Dvs. at man skal finde alle de gange hvor den person har haft fat i båden, og hvad er det: En aften, en dag, en weekend, hvad fanden er det for noget? 

Marc: Krydsreferere om der er elever med? 

Jacob: Ja det kan man... Det er vel den nye politik, så skal man også finde ud af om der er elever med. Det tror jeg nok vi havde en... [politik]. Så indførte vi en da det kom så indførte vi med at man lige kunne lave et kryds på besætningen og sige at man har en elev med eller skrive et sted at man har en elev med, så skal man også lige finde ud af det. Så gik vi ikke mere ned i det end at vi stolede på, at hvis man skrev at man havde en elev med, så stolede vi på det. *eksempel:* Nå det var så Anette, hun har jo sejlet der, nå er han blevet stillet ind: fint. Og så gå igennem sådan 2-3 sider, der bliver hurtigt fuldt op på sådan en side, så fandens besvær, derfor gad vi heller ikke gøre det til sidst. 
Og så blev der skrevet girokort ud som blev puttet ind på den *et eller andet pågældende 12:00* og sendt til den pågældende. Nogen af os kunne godt finde ud af at fiske girokortet over i en PDF og så bare for det sendt på mail, men som regel blev det noget med en kuvert. Ikke nogen elektronisk opkræven her. 
Hvis vi går sådan lidt længere ud... Så omkring det hele med skolen, men det er nok lidt voldsomt, der synes jeg at det havde været et kæmpe puslespil det med at få styr på hvilke elever skal være hvor og hvilke dage og hvordan får vi placeret dem på både osv. Men problemet var ofte lige så stort... Skyldtes lige så meget at folk bare ikke fik meldt tilbage når vi skrev: Nu må i godt lige få sendt jeres ønsker. Når vi skriver til folk: Nu skal vi lige have af vide: Om du har tænkt dig at sejle til sommer, om du sidder over et år eller hvad du gør, hvilke dage du kan sejle. Og hvis de ikke kommer så sidder man bare der... Man kan ikke lave planen. Det synes jeg at vi brugte meget tid på, vi havde ikke rigtig noget stytte til at lave den der, det blev sådan et Excel-regneark til sidst, med båd og elever og så blev det bare hængt op på opslagstavlen: Sådan her ser det ud. Det var sådan lidt... Det virkede.
Som elev og som medlem synes jeg at det største var det besværelige med man kan bare ikke få overblik over en skid hjemmefra. Altså hvilke både er ledige? Det kan så se med det nye system som de har lavet, men jeg kunne virkelig godt tænke mig at sejle 24-timers sejlads. Der skal man altså have fat i *et eller andet 13:58* nede i Sundet og bede om: Er den der indkaldelse kommet op og hænge. Jeg vil gerne ud og sejle onsdagsmatch, der kan man så gå igen. Jeg vil gerne faktisk finde ud af om der er en ledig plads på en onsdagsmatchbåd. Altså, en ting er om den er reserveret, en anden ting er om der lige er en er en... Kan jeg lige springe på som besætning her. Så den der manglende afgang til information når man sidder der hjemme og gerne vil et eller andet og så det der med at det hele er papirbaseret. Skulle gå igennem en stor protokol for at finde ud af hvad folk skal betale for et eller andet, det må da kunne gøres smartere. Min vision var den gang noget integration mellem de forskellige ting, mellem brug af både til skole, til fritidsbrug, til kapsejlads, til skolearrangeret fritidsbrug. Altså der er forskel på at jeg beslutter tage ud søndag og sejle med nogle venner og så at skolen siger at den søndag er der et arrangement som skolen gerne vil ha' at kommer ud, så der laver skolen nærmest en *for-et-eller-andet 15:17*, som så først bliver oplyst måske tre dage før at man finder ud af at der kommer noget, så bliver båden frigivet. Så der er det igen med at man som medlem kan gå ind og se: Kan man komme til der. For det er forskellig måder at reservere og bruge en båd på som godt kan være lidt indviklet at finde rundt i. 

Som medlem kan man leje og låne et fartøj?

Ja det kan de sagtens. Det ved i godt. Kravene er førebeviset. Omkostninger: Vi sendter et girokort. Adminstreres: Ja det gør de jo stortset ved at vi har de der papir, protokoller hvor man skriver sig i. 

Søren: Det der førerbevis det får man bare igennem skolen på de to år som det tager eller er det kun duelighedsbeviset?

Jacob: Altså det tager to år på skolen, det er på den måde vi organiseret det på. Det er et valg man har truffet i klubberne der nede. Jamen det tager to år og faktisk har de tre klubber en lidt forskellig policy omkring det, men i Sundet har det været sådan, *der siges et eller andet 16:35*, det var sådan at det første år sejlede man med yoghurtbærger, undskyld plastikfiberbåde og det andet år sejlede man gaffelriggere. Og det ud fra sådan rent propertion, altså glasfiber, sådan en drabant der, den skal men for det første være rigtigt ond ved den før den vælter og for det andet så er den forholdsvis nem at sejle og man får meget klar og hurtig besked hvis man gør noget forkert, båden giver meget hurtig tilbagemelding. Gaffelriggeren er  svær, den er tung, den er længere om at reagere, den gir' ikke så hurtig og klar besked om hvad der foregår, dvs. at man skal være mere erfaren for at forstå hvad den fortæller en. Så den prøver vi på ikke at skræmme førsteårseleverne i. Alene mængden at torvværk det ku' godt få folk til at blive bekymrede når de sådan skal finde ud af det. Set enkelt er det en fantastisk båd at undervise i, den har et stort åbent... Man kan gå rundt nede i den. Man sidder ikke sådan klemt sammen på smalle bænke, man kan gå omkring, fedt. Så det tager to år. Så har vi, nu glemte jeg at sige før med Spækhuggeren, jeg sagde vi havde en Spækhugger, vi har skolen 2 år. Så har vi det der hedder... Det er så blevet en kapsejladsskole, sammen med de andre klubber i havnen, så folk der er blevet uddannet fra sejlklubberne, kan så tilmelde sig kapsejladsskolen, dvs., det er mest spækhugger vi gør det i, dvs. så kommer de... Så for det noget teoriundervisning og noget praksisundervisning i kapsejlads på forskellige niveauer kulminerende med at man så kan deltage i almindelig kapsejlads. Det er ikke noget man skal, lige så snart man har førerbeviset, så kan du gå ud og sejle kapsejlads, det er slet ikke det, men der er mange der godt vil have den der uddannelse, forstå hvordan man *et eller andet med start 18:33*, hvordan lægger man taktik, hvordan får man samarbejdet til at fungere, til kapsejlads skal det gå temmelig stærkt, så det har vi så sammen med de andre, det er så os klubben, så spækhuggeren, den ene og muligvis også den næste, er mere eller mindre fastreserveret til kapsejladsskolen mandag, tirsdag og onsdag tror jeg det er. Så det er jo også en anden fast ting, altså hvor vi har... Faktisk bliver de jo... Den protokol de udfylder er den samme som skoleprotokollen, den er fortrykt med navne på, afkrydsning, du er på dette kapsejladshold, på den båd om mandagen. Så må du krydse af når du kommer. Og det her med at vi hvem der er på båden det er jo i forhold til skolen, i forhold til skolens regler for om man er dygtig nok når man har gjort nok, og det er helt generelt at vi vil vide hvem som er på båden altid, når den er ude og sejle, det er rent sikkerhedsmæssigt. Så det har det dobbelte formål. 

Søren: Ja det var...

Jacob: Har jeg fået svaret på alle jeres spørgsmål? Jeg kørte bare der ud af. 

Alle: Ja det tror jeg.

Jacob: Hvad gør i nu?









