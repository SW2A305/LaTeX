\chapter{Arbejdsproces}\label{chap:arbejdsproces}

I dette afsnit vil projektets proces for udviklingen af programmet beskrives, som baseres på et begreb kaldet ``agile software development''.

\section{Agile software development}\label{sec:agile-software-development}
Når agile software development benyttes i udviklingsprocessen ligges der fokus på at lave et fungerende program.
Dette anvendes for at holde programmets omfang realistisk, ved tankegangen:
``begynd ikke på noget der ikke er tid til eller kun kan udvikles delvist''.
Med dette menes der, at hver enkelt funktion skal fungere fejlfrit, før arbejdet med den næste påbegyndes. 
Der tilstræbes altså at have et mindre og fejlfrit program, fremfor et større fejlfyldt program.

Agile software development ligger yderligere fokus på ``co-location'', således at arbejde bliver udført på en arbejdsplads med ens kollegaer eller medstuderende, i stedet for at hver person sidder for sig selv, heri ligger ``pair programming''. 
Ved denne metode programmerer to programmører på en enkelt computer, med en fører og en observatør, for øget kodekvalitet.\citep{agile_software}

\section{Faktiske arbejdsproces}\label{sec:faktiske-arbejdsproces}
Pair programming er delvist blevet anvendt i programmeringsprocessen, dvs. at noget af kildekoden er skrevet med denne programmeringsteknik.
Således er mindre dele af kode skrevet af enkeltpersoner, som efterfølgende er gennemgået af andre gruppemedlemmer.
Tankegangen med at en funktionalitet skulle være færdig før der blev startet på ny, blev overholdt i starten af udviklingsprocessen.
Der gik dog ikke lang tid, før der blev fraveget fra dette, da det faldt gruppen svært at opdele de enkelte funktioner i små dele, som kunne laves af forskellige personer eller pair programming grupper.
I stedet for blev arbejdet med andre funktioner påbegyndt, så alle i gruppen havde noget at lave.  
