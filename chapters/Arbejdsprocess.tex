\chapter{Arbejdsprocess}

I dette afsnit vil projektets udviklingsproces beskrives, hvilket metoder der er blevet benyttet, hvordan de er udført samt hvordan udviklingen er planlagt. Udviklingsmæssigt baserer vi vores proces på to primære begreber værende, agile software development og test driven development(TDD).

\section{Agile software development}
Agile software development benyttes som et grundlag for programmeringen af produktet. 
Her ligges der fokus på at lave et fungerende program.
Dette anvendes i udviklingen for at holde programmet realistisk, begynd ikke på noget der ikke er tid til eller kun kan udvikles delvist. 
Hellere have et mindre og godt program, end et større fejlfyldt program.

Agile software development ligger yderligere fokus på ``co-location'', således at arbejde bliver udført på en arbejdsplads med ens kollegaer, medstuderende, snarere end en isoleret process, heri ligger ``pair programming''. Denne metode er baseret på at to programmører programmere på en enkelt computer, her med en fører og en observatør, for øget kodekvalitet. 
Denne metode er blevet delvist anvendt, dvs. somme tider er pair-programming anvendt andre gange ikke. 
Således er der skrevet mindre dele af kode isoleret, som så efterfølgende er gennemgået og samlet sammen.\fxnote{Marc, find din kilde}


