\chapter{Arbejdsprocess}
I dette afsnit vil udviklingsprocessen beskrives, hvilket metoder der er blevet benyttet, hvordan de er udført samt hvordan udviklingen er planlagt. Udviklingsmæssigt basere vi vores process på to primære begreber værende, agile software developlment, test driven development(TDD).

\section{Agile software development}
Agile software development benyttes som et grundlag for programmeringen af produktet. Her ligges der eksempelvis fokus på et virkende produkt frem for en teoretisk snak. Dette anvendes i udviklingen for at holde programmet realistisk, begynd ikke på noget der ikke er tid til eller kun kan laves delvist, hellere have et mindre godt program, end et større fejlfyldt program.

Agile software development ligger yderligere fokus på 'co-location', således at arbejde bliver udført på en arbejdsplads med ens kollegaer, medstuderende, snarere end en isoleret process. Heri ligger 'pair programming', denne metode er baseret på at to programmører programmere på en enkelt computer, her med en fører og en observatør, for øget kodekvalitet. Denne metode er blevet delvist anvendt, mens der til tider har været benyttet pair-programming, er metoden blevet brugt løst for at optimere arbejdsprocessen, således er der skrevet mindre dele af kode isoleret, som så efterfølgende er gennemgået og samlet sammen.

Noget tredje Agile software development også ligger fokus som der benyttes på er en dynamisk udviklingsprocess over planlagt udvikling. Hermed menes der naturligvis ikke at planlægning ikke benyttes, men at hvis der opstår problemer eller andre uventede dilemmaer undervejs i udviklingsprocessen, så håndteres dette med det samme, og udsættes på baggrund af at udføre hvad tidsplanen siger. 

\section{Test driven development}
TDD baserer sig på test til udvikling af kode. TDD anvender en simpel iterativ arbejdsprocess for udvikling af kode (Insert image?). Processen baserer sig på 5 skridt, add a test, run all test and see if the new one fails, write some code, run tests, refacter code og så gentages processen. Denne udviklings process er blevet delvist benyttet i denne rapports udvikling. TDD's første skridt er at producere en test hvorefter at se den fejle. Disse to skridt springes over i denne rapports udvikling, da det findes formålt at teste noget der endnu ikke er kodet, alt der kodes bliver dog testet, hver klasse, method m.m. For at tests ikke bliver uoverskueligt store er der foruden et andet princip rapporten også gør brug af, 'keep the unit small', for at reducere tid benyttet på debugging, samt at øge læsbarheden af tests, programmeres der i mange små methods og classer, snarere end enkelte store.