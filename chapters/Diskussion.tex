\chapter{Diskussion}

I diskussionen vil alle elementer fra problemløsningen blive diskuteret. \fxnote{Troels: Her sammenligner vi krav og hvorvidt de er opfyldt. Mere beskrivende metatekst ellers er den useless.}

Et af kravene til produktet var medlemshåndtering, som er implementeret i klasserne Person, SailClubMember og StudentMember. \fxnote{Troels: At vi har klasserne gør os ikke i stand til at sige at systemet håndterer medlemmer, kun hvis der er persistens, som der står i næste linje.}
Derudover lagres objekterne i databasen. 

Medlemmerne udgør desuden grundlaget for login-systemet, da der herigennem kan bestemmes hvilken bruger der logger ind, og hvilke funktioner i programmet, de skal have lov til at tilgå.  
Denne løsning ses som værende tilstrækkelig, da det er muligt at registrere, hvem som er logget ind og dermed give dem adgang til de respektive funktionaliteter. 

Et vigtigt punkt i programmet var at sørge for, at man kunne oprette en reservation af en båd. 
Dette punkt er blevet gennemført, da der i programmet kan oprettes en reservation. 
Desuden er der indført logik, som sikrer at båden skal være ledig og ikke være skadet, før reservationen kan gennemføres. \fxnote{Findes dette tjek??? - skadet(Søren) Troels: Ja! }
Yderligere er der til en reservation også knyttet en logbog, som skal udfyldes, efter sejladsen er gennemført. 
Udfyldelsen af logbøger blev beskrevet i interviewet med Jacob Nørbjerg, som værende besværligt i Sejlklubben Sundet, da deres logbøger skal udfyldes på papir. 
Det menes derfor, at det udviklede program kan gøre det lettere for klubben at holde styr på logbøger, ved at digitalisere dem, holde referencer til hvilken sejltur de hører til, sørge for at ikke alle kan udfylde logbøgerne og ikke mindst indeholde de krævede informationer. 
Dette anses for at være en god løsning, sammenlignet med at skulle finde den korrekte, fysiske logbog i klubhuset for den pågældende båd og derefter udfylde alle informationerne i hånden. 

Et andet vigtigt krav til programmet var at gøre det muligt at holde styr på undervisningen, både fra elevernes side og undervisernes side. 
I programmet er der blevet konstrueret en del, som specifikt tager sig af undervisningen, hvorfra underviserne kan notere hvilke hold og elever, der har gennemført specifikke opgaver.
Tilsvarende kan eleverne se deres egne fremskridt, og hvornår de har deres undervisningslektioner.
Programmets undervisningsdel fungerer, dog er det specifikke pensum, som Sejlklubben Sundet selv står for at udvælge, ikke implementeret. 
Det menes, at skiftet fra det overfladiske pensum, som programmet er skabt med, til det faktiske pensum som Sejlklubben Sundet anvender, ville være relativt simpelt. 
Derfor kan programmet også bruges i andre sejlklubber med undervisning, som anvender et andet pensum.\fxnote{Det kunne dog udvides ved også at holde styr på antallet af gange en elev har udført en given opgave, og ikke kun notere om eleven ahr duført opgaven før.(Søren)}

Begivenheder findes også på listen af produktkrav.
Funktionaliteten for begivenhedshåndtering er blevet implementeret, og det er muligt for brugere selv at oprette, samt tilmelde sig begivenheder.\fxnote{Thomas: Kun oprette hvis de er admin}
Desuden kan en bruger, når de logger ind på systemet, se kommende begivenheder og hvilke andre medlemmer der er tilmeldt, hvis begivenheden kræver tilmelding. 
Dette ses som en god løsning, da medlemmer hurtigt kan få et overblik over kommende begivenheder i klubben.
Dog er det ikke muligt kun at se hvilke begivenheder man selv er tilmeldt, hvilket anses for at være en mangel. 

Der var valgt at bruge et persistenslag til lagring af data. 
Først faldt valget på \acl{EF}. 
Grundet problemer beskrevet i \myref{subsec:Pwoblem} blev der skiftet til SQLite, som også endte med at give problemer. 
Ser man på tidsforbruget brugt på databaserne i betragtning, ville en simplere løsning have været bedre.\fxnote{Thomas: 'Hvis man tager tidsforbruget på databaserne i betragtning...', synes det lyder bedre}
Sådan en løsning kunne have været med flade filer eller et program helt uden persistenslag, som udelukkende kørte ved brug af Mockdata.  

Ved problemløsningens start blev det besluttet, at brugergrænsefladen skulle tilstræbe principperne minimalisme og konsistens i layout.
Grundet den spredte arbejdsproces, som blev beskrevet i \myref{sec:faktiske-arbejdsproces}, blev disse principper ikke overholdt, da gruppemedlemmernes forståelse for implementationen af principperne ikke stemte overens med hinanden.
Dette har haft en negativ indvirkning på programmets layout, da skærmvinduerne ikke er konsistente.
Ved brugertest kom dette til udtryk, da testpersonerne havde svært ved at finde rundt i enkelte områder af programmet. \fxnote{Troels: Dette var ikke nødvendigvis fordi vores vinduer ikke var konsitente med hinanden, men nærmere fordi nogle aspekter af vores layout ikke gav mening?}
Testpersonerne bestod ikke af sejlklubmedlemmer, men det menes stadig deres testresultater og feedback kan bruges til at forbedre programmet.
Dette konkluderes, da Sejlklubben Sundets medlemmer består af et bredt befolkningssegment, som også inkluderer testpersonerne.
Testen viste, at der var visse mangler ved programmets layout, dog ytrede testpersonerne generelt tilfredshed med programmet.
Det vurderes dog, at der stadig skal laves ændringer til programmets brugergrænseflade i tilfælde af en publicering i kommerciel kontekst. 

Det er altså muligt for Sejlklubben Sundet at benytte programmet på nuværende tidspunkt, men det anbefales at testpersonernes forslag til ændringer bliver implementeret, før det tages i brug.
Løsningen kan anvendes af andre sejlklubber med sejlerskole og bådudlån, efter de nødvendige rettelser er implementeret. \fxnote{Troels: Hvis vi tjekker alle user stories igennem, så er der en del mangler, syndes det er meget modigt at gå ud fra at programmet er klart, efter de ændring der er forslået}

Ses der på programmet i et større perspektiv, vil det være svært at bruge programmet til andre formål end sejlklubsrelaterede. \fxnote{Nikolaj: Lidt mærkelig formulering, måske der bare skal stå: ``Det vurderes at det vil være svært at bruge...''}
Det er grundet den tætte forbindelse mellem modellerne og selve brugergrænsefladens Code-Behind. 
Derfor kan programmet ikke ses som værende mulig at anvende i andre former for fritidsklubber. 

Sammenlignes problemformuleringens målsætninger med den færdige løsning, kan der argumenteres for, at der er blevet udviklet et brugbart IT-system. 
Dette konkluderes, da systemet digitaliserer en stor del af den dokumenthåndtering, som foregår i klubben, såsom undervisningsfremskridt for hver elev, booking af både, udfyldelse af logbøger, oprettelser af begivenheder og fremvisning af begivenheder.
Programmet er god til at håndtere dokumentationen, som påkræves af Sejlklubben Sundet vedrørende sejlture og undervisning.
Desuden verificerer programmet brugerinput, og dette kan mindske fejlraten på informationerne.