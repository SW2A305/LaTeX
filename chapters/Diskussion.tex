\chapter{Diskussion}\label{chap:diskussion}

I diskussionen vil alle elementer fra problemløsningen blive diskuteret, herunder i hvor høj grad produktkravene er opfyldt.


\section*{Medlemshåndtering og Brugerlogin}

Et af kravene til produktet var medlemshåndtering, som er implementeret ved hjælp af klasserne Person, SailClubMember og StudentMember, da instanser af klasserne lagres i databasen; hvert med et unikt ID.

Medlemmerne udgør desuden grundlaget for login-systemet, da der herigennem kan bestemmes hvilken bruger der logger ind, og hvilke funktioner i programmet, de skal have lov til at tilgå.  
Denne løsning ses som værende tilstrækkelig, da det er muligt at registrere, hvem som er logget ind og dermed give dem adgang til de respektive funktionaliteter. 


\section*{Bådreservation og logbog}

Et vigtigt punkt i programmet var at sørge for, at man kunne oprette en reservation af en båd. 
Dette punkt er blevet gennemført, da der i programmet kan oprettes en reservation. 
Desuden er der indført logik, som sikrer at båden skal være ledig og ikke være skadet, før reservationen kan gennemføres.
Yderligere er der til en reservation også knyttet en logbog, som skal udfyldes, efter sejladsen er gennemført. 
Udfyldelsen af logbøger blev beskrevet i interviewet med Jacob Nørbjerg, som værende besværligt i Sejlklubben Sundet, da deres logbøger skal udfyldes på papir. 
Det menes derfor, at det udviklede program kan gøre det lettere for klubben at holde styr på logbøger, ved at digitalisere dem, holde referencer til hvilken sejltur de hører til, sørge for at ikke alle kan udfylde logbøgerne og ikke mindst indeholde de krævede informationer. 
Dette anses for at være en god løsning, sammenlignet med at skulle finde den korrekte, fysiske logbog i klubhuset for den pågældende båd og derefter udfylde alle informationerne i hånden. 


\section*{Undervisningsorganisering}

Et andet vigtigt krav til programmet var at gøre det muligt at holde styr på undervisningen, både fra elevernes side og undervisernes side. 
I programmet er der blevet konstrueret en del, som specifikt tager sig af undervisningen, hvorfra underviserne kan notere hvilke hold og elever, der har gennemført specifikke opgaver.
Tilsvarende kan eleverne se deres egne fremskridt, og hvornår de har deres undervisningslektioner.
Programmets undervisningsdel fungerer, dog er det specifikke pensum, som Sejlklubben Sundet selv står for at udvælge, ikke implementeret. 
Det menes, at skiftet fra det overfladiske pensum, som programmet er skabt med, til det faktiske pensum som Sejlklubben Sundet anvender, ville være relativt simpelt. 
Derfor kan programmet også bruges i andre sejlklubber med undervisning, som anvender et andet pensum. 
Det kunne dog udvides, ved også at holde styr på antallet af gange en elev har udført en given opgave, og ikke kun notere om eleven har udført opgaven før.


\section*{Begivenhedsadministration og visning}

Begivenheder findes også på listen af produktkrav.
Funktionaliteten for begivenhedshåndtering er blevet implementeret, og det er muligt for medlemmer at tilmelde sig begivenheder og som administrator oprette, redigere og slette begivenheder. 
Desuden kan en bruger, når de logger ind på systemet, se kommende begivenheder og hvilke andre medlemmer der er tilmeldt, hvis begivenheden kræver tilmelding. 
Dette ses som en god løsning, da medlemmer hurtigt kan få et overblik over kommende begivenheder i klubben.
Dog er det ikke muligt kun at se hvilke begivenheder man selv er tilmeldt, hvilket anses for at være en mangel. 


\section*{Persistenslag}

Der var valgt at bruge et persistenslag til lagring af data. 
Først faldt valget på \acl{EF}. 
Grundet problemer beskrevet i \myref{subsec:Pwoblem} blev der skiftet til SQLite, som også endte med at give problemer. 
Hvis man tager tidsforbruget på udviklingen af persistenslagene i betragtning, ville en simplere løsning have været bedre, da der så havde været mere tid til at udvikle og teste resten af programmet.
Sådan en løsning kunne have været med flade filer eller et program helt uden persistenslag, som udelukkende kørte ved brug af Mockdata.  

Da der var foretaget flere databaseskift under udviklingen var det et godt valg at bruge et \acl{DAL}, da dette simplificerede skiftet. 
Uden et \ac{DAL} ville dette skift have været besværligt, da ethvert databasekald skulle skrives om. 

\section*{Brugergrænseflade}

Ved problemløsningens start blev det besluttet, at brugergrænsefladen skulle tilstræbe principperne minimalisme og konsistens i layout.
Grundet den spredte arbejdsproces, som blev beskrevet i \myref{sec:faktiske-arbejdsproces}, blev disse principper ikke overholdt, da gruppemedlemmernes forståelse for implementationen af principperne ikke stemte overens med hinanden.
Dette har haft en negativ indvirkning på programmets layout, da skærmvinduerne ikke er konsistente.
Ved brugertest kom dette bl.a. til udtryk, da testpersonerne havde svært ved at finde rundt i enkelte områder af programmet og ikke ved andre. 
Testpersonerne bestod ikke af sejlklubmedlemmer, men det menes stadig deres testresultater og feedback kan bruges til at forbedre programmet.
Dette konkluderes, da Sejlklubben Sundets medlemmer består af et bredt befolkningssegment, som også inkluderer testpersonerne. 
Testen viste, at der var visse mangler ved programmets layout, dog ytrede testpersonerne generelt tilfredshed med programmet.
Det vurderes derfor, at der stadig skal laves ændringer til programmets brugergrænseflade i tilfælde af en publicering i kommerciel kontekst. 

\section*{Modeldesign}
I det udviklede modellag var der flere designfejl som vil blive uddybet nedenunder. 
Da der ikke var tid til at refaktorisere programmet, blev disse designfejl ikke udbedret. 
\begin{itemize}
	\item SailTrip- og RegularTripklassen
	\item Lectureklassen
	\item StudentMember booleans
\end{itemize}

\textbf{SailTrip- og RegularTripklassen}: 
Disse to klasser burde have været samlet, da indtil flere felter er placeret på subklassen i stedet for superklassen. 

\textbf{Lectureklassen}: 
Denne klasse burde have en relation til SailTripklassen. 

\textbf{StudentMember booleans}:
Disse booleans burde have været af typen \texttt{int}, da elever godt kunne have lavet den samme øvelse flere gange. 

\section*{Opsummering}

Det er altså muligt for Sejlklubben Sundet at benytte programmet på nuværende tidspunkt, men det anbefales at testpersonernes forslag til ændringer bliver implementeret, før det tages i brug.
Løsningen kan anvendes af andre sejlklubber med sejlerskole og bådudlån, som ikke har brug for flere funktionaliteter end de udviklede, efter de nødvendige rettelser er implementeret. 

Det vurderes at det vil være svært at bruge programmet til andre formål end sejlklubsrelaterede. 
Det er grundet den tætte forbindelse mellem modellerne og selve brugergrænsefladens Code-Behind. 
Derfor kan programmet ikke ses som værende mulig at anvende i andre former for fritidsklubber. 

Sammenlignes problemformuleringens målsætninger med den færdige løsning, kan der argumenteres for, at der er blevet udviklet et brugbart management system. 
Dette konkluderes, da systemet digitaliserer en stor del af den dokumenthåndtering, som foregår i klubben, såsom undervisningsfremskridt for hver elev, booking af både, udfyldelse af logbøger, oprettelser af begivenheder og fremvisning af begivenheder.
Programmet er i stand til at håndtere dokumentationen, som påkræves af Sejlklubben Sundet vedrørende sejlture og undervisning.
Desuden verificerer programmet brugerinput, og dette kan mindske fejlraten på informationerne.