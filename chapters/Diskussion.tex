\chapter{Diskussion}
I vores produktkrav var det er punkt vedrørende medlemmer, hvilket er blevet tilfredstillende implementeret i programmet. Medlemmer har fået deres egen klasse,\textbf{SailClubMember}, som indeholder de relevante informationer om et medlem. Medlemmerne udgør desuden grundlaget for login-systemet, da der herigennem kan bestemmes hvilken bruger der logger ind, og hvilke funktioner i programmet de skal have lov til at tilgå. Login-systemet fungerer desuden med at brugerne har brugernavne og kodeord. Kodeordene er krypteret med en hashingalgoritme, hvilket er gjort for at gøre dem ulæslige i tilfælde af at databasen skulle blive aflæst af andre. 

Et vigtigt punkt i programmet var at sørge for at man kunne oprette en booking af en båd. Dette punkt er blevet gennemført, da der i programmet kan oprettes en booking. Desuden er det indført logik som sikrer at båden skal være ledig og ikke være skadet før at bookingen kan gennemføres. Yderligere er der til en reservation også knyttet en logbog, som skal udfyldes efter sejladsen er gennemført. Dette blev fremført i interviewet med Jacob Nørbjerg, som værende en besværligt i Sejlklubben Sundet, da deres logbøger skal udfyldes på papir. Der menes derfor at programmet her kan gøre det lettere at klubben at holde styr på logbøger, ved at digitalisere dem, holde referencer til hvilken sejltur de hører til, sørge for at ikke alle kan udfylde logbøgerne og selvfølgelig indeholde de krævede informationer. 