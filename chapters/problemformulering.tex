\chapter{Problemformulering}\label{chap:problemformulering-new}

Indtil nu er problemet om fritidsklubbers administrative opgaver blevet belyst og analyseret vha. Laudon og Laudons model beskrevet i \myref{chap:struktur-af-problemanalyse}. 
Det er nu, ud fra denne analyse, muligt at komme frem til en problemformulering, som der vil arbejdes ud fra i løsningsdelen henimod en løsning til fritidsklubber og ikke mindst sejlklubber.

Der blev i \myref{chap:Fritidsklubber}, omhandlende fritidsklubber, afgrænset til sejlklubber, da disse viste sig at have flere specifikke administrative opgaver, sammenlignet med andre fritidsklubber. 
Denne afgrænsning blev foretaget da et system, der kan håndtere generelle samt specifikke opgaver til en sejlklub, kan bruges af andre former for fritidsklubber ved at tilpasse specifikke funktioner.
Ved at analysere videre på de direkte forhold vedrørende en sejlklub, blev der dannet en forståelse for, hvilke problemer en sejlklub med sejlerskole har, som kan løses vha. et administrations system.


\section{Interessenterne for sejlklubber}

Der var forskellige interessenter for sejlklubberne, og de havde varierende mængder af interesse i projektet.

Medlemmerne, underviserne og de frivillige vil alle være personer, der skal bruge det udviklede system. 
De har derfor en mere direkte indvirkning på, hvordan system skal opbygges. 
Hvis systemet ikke har de funktioner, som de efterspørger, vil det ikke være den gode løsning, som de gerne vil have. 
Dette gælder for alle de tre nævnte interessenter. 
Det skal dog understreges, at de ikke vil bruge samme dele af systemet. 
Dette skyldes, at medlemmerne ikke skal kunne oprette undervisningsdage mm., da det kun er underviserne, der skal have tilladelse til dette. 

\section{Organisation}

I forbindelse med organisationsafsnittet blev der undersøgt, hvordan sejlklubberne håndterer forskellige opgaver i klubben, samt hvilke opgaver de beskæftiger sig med.

Det viste sig, at klubberne har individuelle medlemstyper, og det kan derfor være relevant at lade klubberne selv oprette forskellige medlemstyper i systemet. 
Desuden efterspørges det, at man kan tilkoble sig systemet hjemmefra, for således at kunne få informationer om begivenheder og undervisning i klubben, og måske endda tilmelde sig disse, uden at tage turen ned til klubben.

Grundet den store mængde af information, der skal nedskrives, i forbindelse med en sejlads, giver det mening at hjælpe med at organisere denne opgave, samt at gøre det lettere at registrere informationerne for diverse frivillige og undervisere. 
Desuden kunne det hjælpe, hvis hvert medlem havde en saldo over udgifter ved sejlklubben, således det er nemmere at håndtere brugerbetaling.

Følgende er en liste over opgaver, som kan dækkes af et system for sejlklubberne:

\begin{itemize}
  \item Tilkobling hjemmefra via internettet.
  \item Mulighed for at få informationer vedr. begivenheder, samt at tilmelde sig disse.
  \item Organisering af information der nedskrives i forbindelse med en sejlads.
  \item Organisering af betalinger, samlet for det enkelte medlem, samt mulighed for online betaling.
  \item Booking af både.
  \item Administrering af undervisning.
\end{itemize}


\section{Teknologi}

Sejlklubben Sundet, viste sig at have flere forhold at organisere end de andre klubber der er undersøgt, og derfor afgrænses projektet til konkret at designe et IT-system til Sejlklubben Sundet. 
Dette gøres, da hvis et system kan hjælpe Sundet, er det blevet konkluderet, at det også kan hjælpe andre sejlklubber, der har færre forhold at holde styr på.
Man har fundet frem til at systemet, der efterspørges, er af typen management system, som blev beskrevet i \myref{subsec:management-systemer}.\fxnote{Er det ikke bedre med: Der er blevet fundet frem til... i stedet for 'man'? - Thomas}
Der findes allerede systemer, som kan dække Sundets administrative behov, bortset fra håndteringen af undervisning.
Der mangler altså et program på markedet, som kan håndtere en sejlklubs sejlerskole.
Heri ligger projektets eksistensberettigelse.\fxnote{Sundet vs. Sejlklubben Sundet - Thomas}

\subsection*{Problemformulering}
\subsubsection*{Ud fra denne afgrænsning er flg. problemformulering formuleret:}

\begin{center}
  \begin{tabular}{|p{14cm}|}
    \textit{Det er et problem at frivillige i fritidsklubber med specielle udlejningsmuligheder, så som Sejlklubben Sundet, benytter unødvendig arbejdskraft på fysisk dokumenthåndtering vedrørende udlånte faciliteter, undervisning og begivenhedsorganisation. 
    Hvordan kan et system hjælpe med at danne overblik over sådanne opgaver?}
  \end{tabular}
\end{center}
\fxnote{Nikolaj: bør det ikke nærmere være: "Hvordan kan der udvikles et system som kan hjælpe med at danne overblik over sådanne opgaver?}


\section{Afgrænsning for problemløsning}

Da dette projekt er udarbejdet i løbet af 2. semester på Aalborg Universitet, er der ikke uanede mængder af tid.\fxnote{Lyder lidt talesprogsagtigt 'uanede mængder', måske skrive noget med begrænset tid. - Thomas} 
Der afgrænses derfor fra at udvikle hele management systemet til sejlklubben Sundet, til i stedet at lave enkelte dele af systemet.

Grundet de manglende ressourcer og tid, vil der altså ikke blive lavet et system, der kører over internettet, men i stedet et system, som kan håndtere de forskellige emner lokalt. 
Denne afgrænsning finder sted, da hvis funktionerne for klubben kan fungerer på computeren, skal det tilkobles en server, for at kunne tilgåes fra flere forskellige computere. \fxnote{Hvad menes der med ``computeren'' er det den nede i klubben, eller enhver?}
Derfor ser projektgruppen altså funktionerne for Sejlklubben Sundet som værende de vigtige ting at udvikle, selvom systemet mister noget af sin effektivitet uden internetopkoblingen.\fxnote{Nikolaj: Omformulering: " selvom systemet mister sin brugbarhed uden internetopkoblingen}
Dette ses som værende acceptabelt, da systemet ikke skal tages i brug af klubben, da dette er et projekt lavet i forbindelse med uddannelsen på Softwareingeniørstudiet ved Aalborg Universitet.

De resterende kapitler i rapporten, vil beskæftige sig med udviklingen af et management system, henvendt imod Sejlklubben Sundet, baseret på problemformuleringen. 
Som det første vil der komme en kravspecifikation for systemet, på baggrund af denne afgrænsning.
