\chapter{Problemformulering}\label{chap:problemformulering}


Igennem problemanalysen er der på baggrund af kontekstmodellen, blevet udledt delkonklusioner vedr. interessenter, organisation og teknologi.

Ud fra indsamlet data konkluderes der, at sejlklubber i stor grad har meget til fælles, med andre klubber dedikerede til udøvelse af en given hobby. Det ses i analysen, af de klubber som rapporten har set nærmere på, at der er et generelt behov for administration af en ressource, som klubben har til rådighed. En stor del af de mindre fritidsklubber er drevet af en større del af frivillige, hvilket har en betydning for den tidsmæssige investering, der kan sættes i projektet.\fxnote{hvilket projekt henvises der til?} Således ville et effektivt system til håndtering af opgaver som kontingentbetaling, medlemsinformation eller lignende være til gavn for klubbens generelle arbejdskraft.\fxnote{hvad er generel arbejdskraft?}

Der konkluderes således, at der ses på et generelt problem. Der tilstræbes en overordnet løsning, der løser så mange problemer som muligt, og derefter en specialisering til hver enkelt klubs specielle krav.\fixme{Dette går da imod det vi har skrevet tidligere, se sidste linje under 4.4 Konklusion af organisationsanalyse. Hvis dette præciseres senere så er dette måske lidt forvirende.} Da en specialiseret løsning til hver enkelte klub, ikke er tidsmæssigt muligt, fokuseres der på en overordnet løsning med specialisering i sejlklubber, som så kan videreudvikles til at indeholde yderligere specialer foruden de generelle kriterier. Men dette er ikke muligt inden for projektets tidsramme, derfor vil projektet fokusere på en enkelt organisation. 

I forbindelse med organisationsanalyserne er der blevet undersøgt tre sejlklubber. De tre klubber har mange ting til fælles i deres organisation.

De har en bestyrelsesformand, nogle gange en næstformand. Formanden er valgt fra generalforsamlingen og står for
dagsordnerne ved generalforsamlingerne, og fungerer som en leder for klubben. Herudover er der sekretærer, samt nogle
der står for klubbernes økonomi. Udover disse stillinger, er der de frivillige hjælpere, som hjælper med diverse opgaver
i klubben, nogle af dem er sejlinstruktører, andre hjælper med at holde styr på bådene og holde dem i stand. Der er også
forskellige typer medlemmer i klubberne. Nogle af dem kan være elever i sejlerskolen, hvis klubben har sådan en, der kan
desuden også være flere typer medlemmer. Fælles for medlemmerne er, at de betaler kontingenter for at være medlem, og for
at kunne benytte sig af klubbens faciliteter. Det er herudover meget forskelligt fra klub til klub, hvilke frynsegoder
der kan være ved at hjælpe til i klubben.

Efter denne undersøgelse er der dannet en forståelse af hvordan bådklubberne er opbygget, og hvordan de håndterer
forskellige opgaver i form af nedskrivninger i bøger og på opslagstavler.

Endvidere blev der set på teknologier på markedet, som det er nu. Ud fra teknologi- og organisationsanalysen kan det
uddrages, at sejlklubber med fordel kan anvende et IT-system til at administrere deres administrative opgaver. Der blev
også fundet frem til nogle features, en sejlklub kunne bruge i et sådant IT-system. Endvidere kan der, ud fra de to
nævnte afsnit, tænkes på en overordnet struktur for et program, som kunne hjælpe sejlklubber og andre fritidsklubber med
at varetage deres administrative opgaver.

Sejlklubben Sundet, viste sig at have mange flere forhold at bogføre\fixme{Skulle vi undgå dette ord grundet betydningen i forhold til banker eller er det bevidst? -Troels} m.m. end de andre klubber, der er undersøgt. Derfor afgrænses projektet til konkret at designe et IT-system for Sundet. Dette gøres, da hvis et system kan hjælpe Sundet, vil der på baggrund af analysen kunne konkluderes, at det også kan hjælpe andre sejlklubber, med færre forhold at holde styr på. 

\subsubsection*{Ud fra denne afgrænsing er flg. problemformulering formuleret:}
\begin{center}
\begin{tabular}{|p{14cm}|}
\textit{Det er et problem, at sejlklubben Sundet bruger unødige ressourcer på fysisk bogføring. Kan der laves et system, som kan håndtere alle de data som bliver bogført, nedsætter recourceforbruget og samtidigt er nemt at bruge? }
\end{tabular}
\end{center}
\subsubsection*{Ud fra denne afgrænsing er flg. problemformulering formuleret:}
\begin{center}
\begin{tabular}{|p{14cm}|}
\textit{Det er et problem at frivillige, i sejlklubber som Sundet, benytter unødvendig arbejdskraft på fysisk dokumenthåndtering. Hvordan kan et system reducere mængden af arbejdskraft brugt? }
\end{tabular}
\end{center}

