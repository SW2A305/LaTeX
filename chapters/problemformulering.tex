\chapter{Problemformulering}\label{chap:problemformulering-new}

Indtil nu er problemet om fritidsklubbers administrative opgaver blevet belyst og analyseret vha. Laudon og
Laudons model beskrevet i \myref{chap:struktur-af-problemanalyse}. Det er nu ud fra denne analyse muligt at
komme frem til en problemformulering, som vil hjælpe til at finde en løsning for fritidsklubber og ikke mindst
sejlklubberne.

Der blev i \myref{chap:Fritidsklubber}, omhandlende fritidsklubber, afgrænset til sejlklubber, da disse viste sig
at have flere specifikke administrative opgaver, sammenlignet med andre fritidsklubber. Denne afgrænsning blev foretaget
da et system der kan håndtere generelle, samt specifikke opgaver til en sejlklub, kan bruges af andre former for
fritidsklubber ved at tilføje, fjerne og modificere specifikke funktioner. Derfor ved at analysere videre på de
direkte forhold vedrørende en sejlklub, ville der kunne dannes en forståelse for, hvilke problemer en sejlklub
med sejlerskole har, som kan løses vha. et administrativt system.


\section{Interessenterne for sejlklubber}

Der var forskellige interessenter for sejlklubberne, og de havde varierende mængder af interesse i projektet.
Dansk Sejlunion er den som skiller sig mest ud. De kan have interesse i projektet for at kunne hjælpe klubber
med at få et godt administrativt system, så klubberne kan give deres medlemmer en bedre oplevelse og service.

Medlemmerne, underviserne og de frivillige vil alle være personer der skal bruge det udviklede system. De har
derfor en mere direkte indvirkning på, hvordan system skal opbygges. Hvis systemet ikke har de funktioner som
de efterspørger, vil det ikke være den gode løsning, som de gerne vil have. Dette gælder for alle de tre
nævnte interessenter. Det skal dog understreges, at de ikke vil bruge samme dele af systemet. Dette
skyldes, at medlemmerne ikke skal kunne oprette undervisningsdage mm., det er kun underviserne. Dette er bl.a.
en af grundene til, at de er delt op i de tre forskellige grupper.


\section{Organisation}

I forbindelse med organisationsafsnittet blev der undersøgt, hvordan sejlklubberne håndterer forskellige
opgaver i klubben, samt hvilke opgaver de beskæftiger sig med.

Det viste sig, at klubberne har individuelle medlemstyper, og det kan derfor være relevant at lade klubberne
selv oprette forskellige medlemstyper i systemet. Desuden efterspørges det, at man kan tilkoble sig systemet
hjemmefra, for således at kunne få informationer om begivenheder og undervisning i klubben, og måske endda
tilmelde sig disse, uden at tage turen ned til klubben.

Grundet den store mængde af information, der skal nedskrives, i forbindelse med en sejlads, giver det mening at
hjælpe med at organisere denne opgave, samt at gøre det lettere at registrere informationerne for diverse
frivillige og undervisere. Desuden kunne det hjælpe hvis hvert medlem havde en saldo over udgifter ved sejlklubben, således
det er nemmere at håndtere brugerbetaling.

Følgende er altså en liste over emner som kan dækkes af et system for sejlklubberne:

\begin{itemize}
  \item Tilkobling hjemmefra, via internettet.
  \item Mulighed for at få informationer vedr. begivenheder, samt at tilmelde sig disse.
  \item Organisering af informationer der foretages i form af sejladser.
  \item Organisering af betalinger, samlet for det enkelte medlem, samt mulighed for online betaling.
\end{itemize}

Man har fundet frem til, at systemet, der efterspørges, er et af typen management systemer, som blev beskrevet i
\myref{subsec:management-systemer}. For at kunne implementere disse funktioner til et
management system, skal der være forskellige teknologier til rådighed. Disse blev beskrevet i
\myref{chap:teknologi-analyse}.


\section{Teknologi}

For at kunne tilkoble sig systemet igennem internettet, skal man have opsat en server til systemet. Serveren
bør ikke være lokaliseret i klubhuset, men man kan købe sig til forskellige servere i datacentre rundt om i
verden. For at systemet skal kunne fungere optimalt for de frivillige, er det stadig vigtigt med en computer i
klubhuset, og der skal derfor være en internetopkobling i klubhuset.

Foruden en serveropsætning er det vigtigt at se på designet af systemet, for at de frivillige kan udnytte det
optimalt. Her tænkes der på brugergrænsefladen. Som nævnt i \myref{chap:interessent-analyse-ved-sejlklubber}
er det en blandet gruppe af mennesker, der kan være frivillige i en sejlklub. Derfor bør brugergrænsefladen
være nem at finde rundt i, således de fleste vil kunne bruge systemet uden større problemer. Dette
forventes at blive gjort ved brugen af Windows Presentation Foundation.\fxnote{Skal dette være der inden diskussionen mellem WPF, WinForms og Webpage?}

Sejlklubben Sundet, viste sig at have mange flere forhold at organisere end de andre klubber der er undersøgt, og
derfor afgrænses projektet til konkret at designe et IT-system til Sundet. Dette gøres da hvis et system kan
hjælpe Sundet, vil der på baggrund af analysen kunne konkluderes, at det også kan hjælpe andre sejlklubber,
med færre forhold at holde styr på.\fxnote{Kan ikke helt finde ud af starten af sætning}

\subsubsection*{Ud fra denne afgrænsning er flg. problemformulering formuleret:}

\begin{center}
  \begin{tabular}{|p{14cm}|}
    \textit{Det er et problem at frivillige i fritidsklubber med specielle udlejningsmuligheder, så som Sundet, benytter unødvendig arbejdskraft på fysisk dokumenthåndtering af faciliteternes udlån, undervisning og begivenhedsorganisation. Hvordan kan et system hjælpe med at danne overblik over sådanne opgaver?}
  \end{tabular}
\end{center}


\section{Afgrænsning for problemløsning}

Da dette projekt er foretaget i løbet af forårssemestret på Aalborg Universitet, er der ikke uanede mængder af tid. Der
afgrænses derfor fra at udvikle hele management systemet til sejlklubben Sundet, til i stedet at lave enkelte
dele af systemet.

Grundet de manglende ressourcer og tid, vil der altså ikke blive lavet et system, der kører over internettet,
men i stedet et system, som kan håndtere de forskellige emner lokalt. 
Denne afgrænsning finder sted, da hvis
funktionerne for klubben kan fungerer på computeren, skal det blot tilkobles en server, for dermed at kunne
virke på flere forskellige computere.\fxnote{Underlig sætning, er lidt i tvivl om hvad der menes}
Derfor ser projektgruppen altså funktionerne for sejlklubben som værende de vigtige ting at udvikle, selvom systemet mister meget af sin effektivitet uden internetopkoblingen. 
Men dette ses som værende okay, da systemet ikke skal tages i brug af klubben, da dette er et projekt lavet i
forbindelse med uddannelsen på Softwareingeniørstudiet ved Aalborg Universitet.

Hvis tiden er til det i slutningen af projektet, vil der muligvis ændres i systemet så det virker igennem en
internetopkobling.\fxnote{Skal sådanne gisninger omkring projektforløbet med?}

De resterende kapitler i rapporten, vil beskæftige sig med udviklingen af et management system, henvendt imod
sejlklubben Sundet, baseret på problemformuleringen. Som det første vil der komme en kravspecifikation for
systemet, baseret på baggrund af denne afgrænsning.
