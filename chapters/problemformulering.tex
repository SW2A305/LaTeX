\chapter{Problemformulering}\label{chap:problemformulering}

Indtil nu er problemet om fritidsklubbers administrative opgaver blevet belyst og analyseret vha. Laudon og Laudons model beskrevet i \myref{sec:struktur-af-problemanalyse}. Det er nu ud fra denne analyse muligt at komme frem til en problemformulering, som vil hjælpe til at finde en løsning for fritidsklubber og ikke mindst sejlklubberne.

Igennem problemanalysen er der på baggrund af kontekstmodellen, blevet udledt delkonklusioner vedr. interessenter, organisation og teknologi.

Ud fra indsamlet data konkluderes der, at sejlklubber i stor grad har meget til fælles, med andre klubber dedikerede til udøvelse af en given hobby. Det ses i analysen, af de klubber som rapporten har set nærmere på, at der er et generelt behov for administration af en ressource, som klubben har til rådighed. En stor del af de mindre fritidsklubber er drevet af en større del af frivillige, hvilket har en betydning for den tidsmæssige investering, der kan sættes i projektet.\fxnote{hvilket projekt henvises der til?} Således ville et effektivt system til håndtering af opgaver som kontingentbetaling, medlemsinformation eller lignende være til gavn for klubbens generelle arbejdskraft.\fxnote{hvad er generel arbejdskraft?}

Der konkluderes således, at der ses på et generelt problem. Der tilstræbes en overordnet løsning, der løser så mange problemer som muligt, og derefter en specialisering til hver enkelt klubs specielle krav.\fixme{Dette går da imod det vi har skrevet tidligere, se sidste linje under 4.4 Konklusion af organisationsanalyse. Hvis dette præciseres senere så er dette måske lidt forvirende.} Da en specialiseret løsning til hver enkelte klub, ikke er tidsmæssigt muligt, fokuseres der på en overordnet løsning med specialisering i sejlklubber, som så kan videreudvikles til at indeholde yderligere specialer foruden de generelle kriterier. Men dette er ikke muligt inden for projektets tidsramme, derfor vil projektet fokusere på en enkelt organisation.

Medlemmerne, underviserne og de frivillige ville alle være personer der skal bruge det udviklede system. De har derfor en mere direkte indvirkning på hvordan system skal opbygges. Hvis systemet ikke har de funktioner som de efterspørger, vil det ikke være den gode løsning, som de gerne vil have. Dette gælder for alle de tre nævnte interessenter. Det skal dog understreges at de ikke ville skulle bruge samme dele af systemet. Dette skyldes at medlemmerne ikke skal kunne oprette undervisnings dage mm, det er kun underviserne. Dette er bl.a. grunden til at de delt op i de tre forskellige grupper. 

De har en bestyrelsesformand, nogle gange en næstformand. Formanden er valgt fra generalforsamlingen og står for
dagsordnerne ved generalforsamlingerne, og fungerer som en leder for klubben. Herudover er der sekretærer, samt nogle
der står for klubbernes økonomi. Udover disse stillinger, er der de frivillige hjælpere, som hjælper med diverse opgaver
i klubben, nogle af dem er sejlinstruktører, andre hjælper med at holde styr på bådene og holde dem i stand. Der er også
forskellige typer medlemmer i klubberne. Nogle af dem kan være elever i sejlerskolen, hvis klubben har sådan en, der kan
desuden også være flere typer medlemmer. Fælles for medlemmerne er, at de betaler kontingenter for at være medlem, og for
at kunne benytte sig af klubbens faciliteter. Det er herudover meget forskelligt fra klub til klub, hvilke frynsegoder
der kan være ved at hjælpe til i klubben.

I forbindelse med organisations afsnittet blev der undersøgt hvordan sejlklubberne håndterer forskellige opgaver i klubben, samt hvilke opgaver de beskæftiger sig med. 

Endvidere blev der set på teknologier på markedet, som det er nu. Ud fra teknologi- og organisationsanalysen kan det
uddrages, at sejlklubber med fordel kan anvende et IT-system til at administrere deres administrative opgaver. Der blev
også fundet frem til nogle features, en sejlklub kunne bruge i et sådant IT-system. Endvidere kan der, ud fra de to
nævnte afsnit, tænkes på en overordnet struktur for et program, som kunne hjælpe sejlklubber og andre fritidsklubber med
at varetage deres administrative opgaver.

Sejlklubben Sundet, viste sig at have mange flere forhold at bogføre\fixme{Skulle vi undgå dette ord grundet betydningen i forhold til banker eller er det bevidst? -Troels} m.m. end de andre klubber, der er undersøgt. Derfor afgrænses projektet til konkret at designe et IT-system for Sundet. Dette gøres, da hvis et system kan hjælpe Sundet, vil der på baggrund af analysen kunne konkluderes, at det også kan hjælpe andre sejlklubber, med færre forhold at holde styr på. 

Følgende er altså en liste over emner som kan dækkes af et system for sejlklubberne:

\begin{itemize}
	\item Tilkobling hjemmefra, via internettet.
	\item Mulighed for at få informationer vedr. begivenheder, samt at tilmelde sig disse.
	\item Organisering af informationer der foretages ifm. sejladser.
	\item Organisering af betalinger, samlet for det enkelte medlem, samt mulighed for online betaling.
\end{itemize}

Man har fundet frem til at systemet der efterspørges er et af typen management systemer, som blev beskrevet i (KILDE TIL FUCKING MANAGEMENT SYSTEMER INDSÆTTES HER)
For at kunne implementere disse funktioner til et management system, skal der være forskellige teknologier tilrådighed. Disse blev beskrevet i \myref{chap:teknologi-analyse}.

\section{Teknologi}

For at kunne tilkoble sig systemet igennem internettet, skal man have opsat en server til systemet. Serveren bør ikke være lokaliseret i klubhuset, men man kan købe sig til forskellige servere i datacentre rundt om i verden. For at systemet skal kunne fungerer optimalt for de frivillige er det stadig vigtigt med en computer i klubhuset, og der skal derfor være en internetopkobling i klubhuset.

Foruden en server opsætning er det vigtigt at se på designet af systemet, for at de frivillige kan udnytte det optimalt. Her tænkes der på brugergrænsefladen. Som nævnt i \myref{chap:interessent-analyse-ved-sejlklubber} er det en blandet gruppe af mennesker der kan være frivillige i en sejlklub. Derfor bør brugergrænsefladen være så nem at finde rundt i, således de fleste vil kunne bruge systemet uden større problemer. Dette forventes at blive gjort ved brugen af Windows forms.

Sejlklubben Sundet, viste sig at have mange flere forhold at organisere end de andre klubber der er undersøgt, derfor afgrænses projektet til konkret at designe et IT-system for Sundet. Dette gøres da hvis et system kan hjælpe Sundet, vil der på baggrund af analysen kunne konkluderes, at det også kan hjælpe andre sejlklubber, med færre forhold at holde styr på. 

\subsubsection*{Ud fra denne afgrænsning er flg. problemformulering formuleret:}
\begin{center}
\begin{tabular}{|p{14cm}|}
\textit{Det er et problem, at sejlklubben Sundet bruger unødige ressourcer på fysisk bogføring. Kan der laves et system, som kan håndtere alle de data som bliver bogført, nedsætter recourceforbruget og samtidigt er nemt at bruge? }
\end{tabular}
\end{center}
\subsubsection*{Ud fra denne afgrænsning er flg. problemformulering formuleret:}
\begin{center}
\begin{tabular}{|p{14cm}|}
\textit{Det er et problem at frivillige i fritidsklubber med specielle udlejningsmuligheder, så som Sundet, benytter unødvendig arbejdskraft på fysisk dokumenthåndtering af faciliteternes udlån, undervisning og begivenhedsorganisation. Hvordan kan et system hjælpe med at danne overblik over sådanne opgaver? }
\end{tabular}
\end{center}


\section{Afgrænsning for problemløsning}

Da dette projekt er foretaget i løbet af forårs semestret på AAU er der ikke uanede mængder af tid, der afgrænses derfor fra at udvikle hele management systemet til sejlklubben Sundet, og derimod til at lave enkelte dele af systemet i stedet. 

Grundet de manglende ressourcer og tid, vil der altså ikke blive foretaget et system der kører over internettet, men i stedet et system som kan håndtere de forskellige emner lokalt. Denne afgrænsning finder sted, da hvis funktionerne for klubben kan fungerer på computeren, skal det blot tilkobles en server, for dermed at kunne virke på flere forskellige computere. Derfor ser projektgruppen altså funktionerne for sejlklubben som værende de vigtige ting at udvikle, selvom systemet mister meget af sin effektivitet uden internetopkoblingen. Men dette ses som værende okay, da systemet ikke skal tages i brug af klubben, da dette er et projekt lavet i forbindelse med uddannelsen på Software ingeniør studiet ved AAU.

Hvis tiden er til det i slutningen af projektet, vil der muligvis ændres i systemet så det virker igennem en internetopkobling.

De resterende kapitler i rapporten, vil beskæftige sig med udviklingen af et management system, henvendt imod sejlklubben Sundet, baseret på problemformuleringen. Som det første vil der komme en kravspecifikation for systemet, baseret på baggrund af denne afgrænsning.


