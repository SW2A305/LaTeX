\chapter{Informationer påkrævet af sejlklubben Sundet}\label{bilag:sundet}

\section{Data på sejlladser og medlemmer}

Dette bilag har til formål at beskrive de informationer Sejlklubben Sundet påkræver af de sejlere, der sejler fra deres klub. 
Herudover er der også beskrevet hvilke informationer sejlerskolen Sundet skal bruge om deres elever for at kunne lade dem bestå.

Når en skolesejllads skal foretages skal følgende informationer skrives ned i skoleprotokollen:

\begin{itemize}
	\item Navne på underviseren og eleverne
	\item Ugedag
	\item Dato
	\item Fremmøde, i form af X for mødt, A for afbud
	\item Navne på gæster
	\item Afgang og ankomsttider
	\item Vindforhold
	\item Evt. Kommentarer, normalt henvisninger til skadesrapporterne, hvis skader er opstået
\end{itemize}

Alle disse informationer skrives for hver lektion og arkiveres i klubhuset, dvs. det er ikke elektronisk.

Når der sejles uden for skolen, dvs. fornøjelsesture eller kapsejladser, skal lignende information udfyldes. 
Disse informationer ses her:

\begin{itemize}
	\item Bådnavn/id
	\item Navn og telefonnummer på føreren af båden
	\item Navne på alle i besætningen, evt. telefonnummer
	\item Startid og dato
	\item Forventet ankomstdato og tid
	\item Formål med turen, med indikation af område for sejladsen
	\item Reel ankomstdato og tid
	\item Vind og vejrforhold
	\item Kommentarer
\end{itemize}

Hvis der sker skader på en båd under en sejllads skrives disse ned i skadesrapporten. 
Føreren skal også læse op i denne bog før en sejllads, for at sikre at man er klar over eventuelle skader på båden.
I skadesrapporten noteres flg. oplysninger:

\begin{itemize}
	\item Dato
	\item Tekst der beskriver uheldet, eller skaden
	\item Her kan også skrives evt. svar fra bådchefen(Personen der sørger for båden)
\end{itemize}

Til at holde styr på medlemmernes oplysninger, bruger Sundet et Microsoft Access baseret system. Her er der personlige oplysninger på eleverne og medlemmerne, desuden kan der printes girokort ud til håndtering af kontingenterne. Det bruges også til at holde styr på bådene i havnen som er ejet af nogle af klubbens medlemmer, samt deres lokationer. Dette system kan kun tilgås lokalt på computeren lokaliseret i klubhuset.

\section{Informationer og andre events}

Når Sundet skal give informationer ud til medlemmerne, sættes informationerne op på en opslagstavle.
Disse informationer kan f.eks. være tilmeldinger til diverse begivenheder såsom 24-timers sejladser eller onsdagsmatcher. 
Dvs. for at kunne tilmelde sig begivenheder i klubben skal man skrive sig på det bestemte tilmeldingsskema man finder i klubhuset. 
Det er ikke sikkert at medlemmer har tid eller når at opdage tilmeldingsfristerne på diverse begivenheder, hvilket kan resultere i et lavere fremmøde, og mindre aktivitet i klubben.
