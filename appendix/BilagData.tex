\chapter{Informationer påkrævet af sejlklubben Sundet}\label{bilag:sundet}

\section{Data på sejlladser og medlemmer}

Dette bilag har til formål at beskrive de information Sundet påkræver af de sejlere der sejler fra deres klub. Herudover er der også informationer omkring hvilke informationer sejlerskolen ved Sundet skal bruge på deres elever for at kunne lade dem bestå.

Når en skolesejllads skal foretages skal følgende informationer skrives ned skoleprotokollen:

\begin{itemize}
	\item Navne på læren og eleverne
	\item Ugedag
	\item Dato
	\item Fremmøde, i form af X for mødt, A for afbud
	\item Navne på gæster
	\item Afgang og ankomsttider
	\item Vind forhold
	\item Evt. Kommentarer, normalt henvisninger til skadesrapporterne hvis sådanne er opstået
\end{itemize}

Alle disse informationer skrives for hver skolegang, og arkiveres i klubhuset, dvs. det er ikke elektronisk.

Når der sejles uden for skolen, dvs. fornøjelsesture, eller kapsejlladser skal lignende information udfyldes, dog med visse ændringer:

\begin{itemize}
	\item Bådnavn/id
	\item Navn, telefonnummer for føren af båden
	\item Navne på alle i besætningen, evt. telefonnummer
	\item Startid og dato
	\item Forventet ankomstdato og tid
	\item Formål med turen, med indikation af område for sejladset
	\item Reelle ankomstdato og tid
	\item Vind og verjforhold
	\item Kommentarer
\end{itemize}

Hvis der sker skader på en båd under en sejllads skrives disse ned i skadesrapporten. Føren skal også læse op i denne bog før en sejllads for at sikre at man er klar over evt. skader på båden, som man skal være klar over. I denne bog skrives flg. oplysninger:

\begin{itemize}
	\item Dato
	\item Tekst der beskriver uheldet, eller skaden
	\item Her kan også skrives evt. svar fra bådchefen(Personen der sørger for båden)
\end{itemize}

Til at holde styr på medlemmernes oplysninger, bruger Sundet et Microsoft Access baseret system. Her er der personlige oplysninger på eleverne og medlemmerne, samt kan der printes girokort ud til håndterring af kontigenterne. Det bruges også til at holde styr på bådene i havnen som er ejet af nogle af klubbens medlemmer, samt deres lokationer. Dette system kan kun tilkobles lokalt på computeren lokaliseret i klubhuset.

\section{Informationer og andre events}

Når Sundet skal give informationer ud til medlemmerne, sættes disse op på en opslagstavle. Disse informationer kan f.eks. være tilmeldinger til diversebegivenheder såsom 24-timers sejlladser eller onsdagsmatcher. Dvs. for at kunne tilmelde sig begivenheder i klubben skal man skrive sig på det bestemte tilmeldingsskema man finder i klubhuset. Det er ikke sikkert at medlemmer har tid eller når at opdage tilmeldingsfristerne på diverse begivenheder, hvilket kan resultere i et lavere fremmøde, og mindre aktivitet i klubben.
