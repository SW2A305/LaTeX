\chapter{Bruger tests af program}\label{BrugerTestCases}
\cbstart
Du skal nu teste programmet vi har udviklet i forbindelse med vores P2-projekt på AAU. Programmet er lavet som et management system til en sejlklub, så de kan holde styr på deres både, deres medlemmer, deres undervisningsforløb, og desuden deres logbøger de holder for alle klubbens både. I den forbindelse vil du blive bedt om at udføre forskellige opgaver i vores program. Et eksempel kunne være at vi beder dig logge ind med et givent brugernavn og kodeord, og herefter finde frem til medlemmet Troels Kroegh, i medlemsoversigten.

Hvis der er noget du kommer i tvivl om, f.eks. hvad du skal i den givne opgave, så kan du spørge, og vi vil vurdere om det vil ødelægge testen og få svaret på dit spørgsmål.


\section{Opgave 1}

Log ind med følgende brugeroplysninger: INDSÆT BRUGERNAVN og INDSÆT PASSWORD

Når du har gjort dette bedes du finde tilmelde dig begivenheden " Pizza Fest" som finder sted den 20/6-14  klokken 19.00.

\section{Opgave 2}

Forbliv logget ind til denne opgave. Du bedes nu booke en båd med følgende oplysninger:

\begin{itemize}
	\item Bådnavn: INDSÆT BÅDNAVN
	\item Besætning: INDSÆT MEDLEMMER TIL CREW
	\item Kaptajn: INDSÆT MEDLEM TIL KAPTAJN
	\item Formål: " Vi skal ud og fiske på havet."
	\item Afgang: " 20/6-14 klokken 09:00 " 
	\item Ankomst " 20/6-14 klokken 17.00 "
\end{itemize}

Log herefter ud af programmet, så du er klar til næste opgave, som du finder på næste side.

\newpage
 
\section{Opgave 3}

Log ind med følgende brugeroplysninger: INDSÆT BRUGERNAVN og INDSÆT PASSWORD

Dette medlem er administrator da medlemmet er en af sejlerskolens undervisere.
Du har lige været ude og undervise eleverne og skal nu registrere at dit hold har fuldført et nyt punkt på deres liste over mål i forbindelse med uddannelsen.

Du bedes 

\begin{itemize}
\item Finde holdet: INDSÆT HOLDNAVN
\item Vælg lektion: INDSÆT LEKTION
\item Afkryds at alle medlemmer på holdet har fuldført deres natsejlads
\item Gem herefter lektionsinformationerne
\end{itemize}


\section{Opgave 4}

Forbliv logget ind til denne opgave.

Gør følgende:
\begin{itemize}
\item Vælg hold: INDSÆT HOLDNAVN
\item Opret lektion for holdet der starter kl. 19.00 og slutter 21.00 den 1. august 2014
\end{itemize}


\section{Opgave 5}

Forbliv logget ind til denne opgave.

Her skal du:
\begin{itemize}
		\item Opret et nyt hold
		\item Navngiv holdet: TestHold
		\item Holdet skal bestå af følgende medlemmer: INDSÆT MEDLEMMER TIL HOLD
		\item Holdet skal være et andenårs hold 		
		\item Gem nu holdet i programmet
\end{itemize}

\section{Opgave 6}

Forbliv logget ind til denne opgave.

Hold INDSÆT HOLDNAVN har færdiggjort alle læringsområderne, så de kan nu få et bådførerbevis.

For at opnå dette, gør følgende
\begin{itemize}
\item Vælg hold: INDSÆT HOLDNAVN
\item Forfrem holdet
\end{itemize}


\section{Opgave 7}

Forbliv logget ind til denne opgave.

Du bedes:
\begin{itemize}
\item Vælg hold: INDSÆT HOLDNAVN
\item Slet holdet
\end{itemize}


\section{Opgave 8}

Inden du logger ud fra administratorbrugeren bedes du oprette en ny begivenhed.

Begivenheden skal foregå den 7/6-14, og skal hedde, Bålfest ved molen. Beskrivelsen kan være hvad som helst, alle kan tilmelde sig. Log herefter ud igen.


\section{Opgave 9}

Log ind med følgende brugeroplysninger: INDSÆT BRUGERNAVN og INDSÆT PASSWORD

Dette medlem en elev. Du er færdig med de to års sejlskole og vil være sikker på at du kan få de bådførerbevis.

Tjek derfor at alle læringsområderne er tjekket af.

Log herefter ud af programmet.


\section{Opgave 10}

Log ind med følgende brugeroplysninger: INDSÆT BRUGERNAVN og INDSÆT PASSWORD

Du er nu kommet hjem fra en sejltur og skal udfylde din logbog for turen.

Turen foregik den 18/6-14, og havde formålet: "Sejler til Sverige og hjem igen."

Find logbogen frem og udfyld felterne, efter egen fantasi. (Du får ikke oplyst informationerne her, du skal altså selv finde på noget at skrive. Det behøver ikke give mening.)

Når du har gjort dette vil vi bede dig finde frem til din nu udfyldte logbog. 

\section{Testen er nu slut}


Tak fordi du ville være med i vores test, vi har nogle opfølgende spørgsmål som vil blive stillet til dig af observatøren fra vores gruppe.

\fxnote{Her skal være bedre beskrivelse af både opret event samt undervisning, jeg kender ikke nok til funktionerne i har lavet.}

\cbend