\chapter{Brugertests af program}\label{BrugerTestCases}


Velkommen til testen af programmet McSnttt udviklet af Software gruppen SW2A305.
Programmet er beregnet som et managementsystem til en sejlklub.\fxnote{Nikolaj: Lyder mærkeligt}
Før du påbegynder hver opgave, bedes du gennemlæse den, når du er klar så giv venligst os besked.
Dette er således, vi kan tage tid på din udførelse af opgaverne, dog skal du ikke føle dig presset til at skynde dig.

\section{Opgave 1}

Log ind med følgende brugeroplysninger: 
\newline - Brugernavn: ``marcus''
\newline - Kodeord: ``husmand''

Når du har gjort dette, bedes du tilmelde dig begivenheden ``PizzaParty'', som finder sted den 20/6-14 klokken 19.00.

\section{Opgave 2}

Forbliv logget ind til denne opgave. Du bedes nu booke en båd med følgende oplysninger:

\begin{itemize}
	\item Bådnavn: Anastasia
	\item Afgang: ``20/6-14 klokken 09:00''
	\item Ankomst ``20/6-14 klokken 17.00''
	\item Besætning: 
	\begin{itemize}
		\item Marcus Husmand
		\item Lars Olsen
		\item Bodil Kjær
		\item Anne Frank
		\item Kasper Eriksen
		\item Anders And
	\end{itemize}
	\item Kaptajn: Anders And
	\item Formål: ``Vi skal ud og fiske på havet.''

\end{itemize}

Log herefter ud af programmet, så du er klar til næste opgave.
 
\section{Opgave 3}

Log ind med følgende brugeroplysninger: 
\newline - Brugernavn: ``oskar''
\newline - Kodeord: ``lauridsen''

Dette medlem er administrator, da medlemmet er en af sejlerskolens undervisere.
Du har lige været ude og undervise eleverne og skal nu registrere, at dit hold har fuldført et nyt punkt på deres liste over mål i forbindelse med uddannelsen.

Du bedes :

\begin{itemize}
\item Finde holdet: ``Svenskerne''
\item Vælg lektion: 05/09/2014 kl. 15:00 
\item Afkryds at alle medlemmer var mødt op, og alle på denne undervisning, lavede Drabant sejllads.
\item Gem herefter lektionsinformationerne
\end{itemize}


\section{Opgave 4}

Forbliv logget ind til denne opgave.

Gør følgende:
\begin{itemize}
\item Sikre du har valgt holdet: ``Svenskerne''
\item Opret lektion for holdet der starter kl. 19.00 og slutter kl. 21.00 den 1. august 2014
\end{itemize}

Log nu ud af programmet.

\section{Opgave 5} \fxnote{Troels: Denne opgave er ikke på punktform.}

Log ind med følgende brugeroplysninger: 
\newline - Brugernavn: ``michelle''
\newline - Kodeord: ``kristensen''

Dette medlem er en elev. Du er færdig med de to års sejlskole og vil være sikker på at du kan få dit bådførerbevis.

Tjek derfor at alle læringsområderne er tjekket af.

Log herefter ud af programmet.


\section{Opgave 6} \fxnote{Troels: Denne opgave er ikke på punktform.}

Log ind med følgende brugeroplysninger: 
\newline - Brugernavn: ``oskar''
\newline - Kodeord: ``lauridsen''

Nu bedes du oprette en ny begivenhed.

Begivenheden skal foregå den 7/6-14, klokken 21:00, og skal hedde, ``Bålfest ved molen''. Beskrivelsen kan være hvad som helst, tilmelding påkrævet, tilmeld dig herefter begivenheden.


\section{Opgave 7}

Forbliv logget ind til denne opgave.

Vi skal nu forfremme et hold i sejlerskolen.

\begin{itemize}
\item Vælg hold: ``MasterRace''.
\item Forfrem holdet og tildel dem deres bådførerbevis.
\item log herefter ud af programmet.
\end{itemize}

\section{Opgave 8}

Log ind med følgende brugeroplysninger: 
\newline - Brugernavn: ``marcus''
\newline - Kodeord: ``husmand''

Du er nu kommet hjem fra en sejltur, på båden ``Anna'', og skal udfylde din logbog for turen.

\begin{itemize}
	\item Åben opret logbådsvinduet for turen som forgik d. 08-05-2014, med formålet: ``Generobre Skåne''
	\item Fjern besætningsmedlemmet ``Kasper Eriksen''
	\item Ændre faktiske ankomst til: 09-05-2014 kl. 10:00
	\item Angiv båden som værende beskadiget.
	\item Angiv skadesrapporten som: ``Masteræb delvist flænset''
	\item Angiv vejrforholdene som: ``2 m/s fra vest''
	\item Tryk ``Udfør''
\end{itemize}

Forbliv logget ind.

\section{Opgave 9}

Du bedes nu tjekke, at din logbog var udført korrekt.
Du skal nu finde logbogen i systemet (Husk det var båden ``Anna'').

Når du har tjekket at du havde angivet vejrforholdene korrekt (``2 m/s fra vest''), er du færdig.

Log nu ud af programmet.

\section{Testen er nu slut}

Tak fordi du ville være med i vores test, vi har nogle opfølgende spørgsmål, som vil blive stillet til dig af observatøren fra vores gruppe.

