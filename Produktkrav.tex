\chapter{Produktkrav}
I problemanalysen blev det gjort klart at der er et problem på området. Problemet kommer fra at der ikke er noget EDB system til at håndtere papirarbejde og gøre det overskueligt, i dette afsnit vil det blive set på hvilke krav et sådant system har.

\section{Funktionelle krav} \label{funktionelleKrav}
Papirarbejde indebære logbøger, manifester, bådreservationer samt hvem der er ombord, tilmeldinger til begivenheder m.m. Som base for at alt dette kan fungere kræves der en metode til at sætte sig selv på en aktivitet, for at gøre dette muligt kræves der således to ting, en database af medlemmer, samt en form for begivenhedskalender. Foruden dette kræves der information om de forskellige begivenheder, en log over hvem der har betalt for at være med, eventuelt hvem der mødte op. Der kræves også et system til udlejning af både samt undervisning hvor nævnte både også benyttes.

Produktet har krav til følgende elementer i sejlklubben:
\begin{itemize}
  \item Medlemmer
  \item Begivenhedsinformation
  \item Begivenhedskalender
  \item Bådreservation
  \item Oversigt og kommunikation for undervisning
\end{itemize}

%Dette afsnit skal uddybes meget mere, har skrevet et eksempel på hvad jeg mener afsnittet skal indeholde
\section{Datalogiske krav}(
Der ligger mere bag de ovenstående elementer en \ref{funktionelleKrav} giver udtryk for, disse vil her blive set nærmere på fra en datalogisk vinkel.
\subsection{Medlemmer}
For at kunne integrere medlemmer i programmet ville skal der være en overordnet klasse der danner base for alle medlemmer. Yderligere specifikationer kan så laves igennem nedarvinger fra de forskellige medlemmer, (elev, administrator, kasser, underviser og lign.)
\subsection{Begivenhedsinformation}

\subsection{Begivenhedskalender}

\subsection{Bådreservation}

\subsection{Oversigt og kommunikation for undervisning}